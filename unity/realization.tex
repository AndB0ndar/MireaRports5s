\section{Разработка}

Разработка игры ведётся на движке Unity.
Для создания основных игровых механик,
таких как управление автомобилем, стрельба и генерация врагов,
используются скрипты на языке C\#.

\subsection{Управление автомобилем}

Ниже представлен код, отвечающий за управление автомобилем:
\lstinputlisting[language=C,
	caption=\leftline{Управление автомобилем}]{code/Player/Move.cs}

Этот скрипт (\texttt{Move.cs}) является основным для управления автомобилем.
Он обрабатывает передвижение автомобиля, вращение колёс, гравитацию и стрельбу.
В методе \textbf{Start()} инициализируются компоненты автомобиля,
такие как колёса и точка проверки земли. Метод \textbf{Update()}
обрабатывает ввод от игрока для движения, прыжков и стрельбы.
Функция \textbf{Moving(float moveInput)} реализует передвижение
автомобиля вперёд и назад, а \textbf{RotateWheels(float direction)}
вращает колёса автомобиля в зависимости от направления движения.
Метод \textbf{Gravity()} обрабатывает гравитацию и прыжки автомобиля,
а функция \textbf{Shoot()} реализует стрельбу из автомобиля.

Код для стрельбы из автомобиля:
\lstinputlisting[language=C,
	caption=\leftline{Стрельба}]{code/Player/Shooting.cs}

Скрипт \texttt{Shooting.cs} отвечает за стрельбу.
В методе \textbf{Update()} обрабатывается нажатие кнопки для стрельбы.
Функция \textbf{Shoot()} создаёт пулю и задаёт её направление и скорость.

Код для управления жизнью игрока:
\lstinputlisting[language=C,
	caption=\leftline{Управление жизнью игрока}]{code/Player/Health.cs}

Скрипт \texttt{Health.cs} отвечает за управление здоровьем игрока.
Метод \textbf{TakeDamage(int damageAmount)} уменьшает здоровье игрока
при получении урона, а метод \textbf{Heal(int healAmount)} восстанавливает
здоровье игрока. Функция \textbf{Die()} обрабатывает смерть игрока
и перезагрузку уровня.

Код для скрипта пули:
\lstinputlisting[language=C,
	caption=\leftline{Скрипт для пули}]{code/Player/Bullet.cs}

Скрипт \texttt{Bullet.cs} управляет поведением пули.
Метод \textbf{Update()} уничтожает пулю,
если она достигает определённой позиции.
Функция \textbf{OnCollisionEnter(Collision collision)}
обрабатывает столкновение пули с врагами или землёй,
а метод \textbf{OnTriggerEnter(Collider other)}
обрабатывает столкновение пули с игроком.

\subsection{Враги}                                                             
                                                                               
Код для генерации врагов:                                                      
\lstinputlisting[language=C,
	caption=\leftline{Генерация врагов}]{code/Enemy/GenMobs.cs}

Скрипт \texttt{GenMobs.cs} отвечает за генерацию врагов на сцене.
В методе \textbf{Start()} вызывается функция \textbf{Spawn()},
которая создаёт первого врага при запуске сцены.
В методе \textbf{Update()} происходит отслеживание времени
с момента последнего сброса врага.
Если прошло достаточно времени (\texttt{dropInterval}),
и на сцене меньше максимального числа врагов (\texttt{max\_count\_mobs}),
и случайное число удовлетворяет условию (вероятность сброса),
то вызывается функция \textbf{Spawn()}, создающая нового врага.

Код для движения врагов:                                                           
\lstinputlisting[language=C,
	caption=\leftline{Движение врагов}]{code/Enemy/FlyMob.cs}

Скрипт \texttt{FlyMob.cs} отвечает за движение летающего врага.
В методе \textbf{Start()} сохраняется начальная позиция по оси Y.
В методе \textbf{Update()} происходит плавное вертикальное движение
объекта вверх-вниз по синусоидальной траектории.
Если враг достигает определённой позиции по оси Z (\texttt{deadline}),
он уничтожается.

Код управления сбросом бомб:                                                  
\lstinputlisting[language=C,
	caption=\leftline{Управление сбросом бомб}]{code/Enemy/BallDrop.cs}

Скрипт \texttt{BallDrop.cs} отвечает за сброс бомб врагами.
В методе \textbf{Update()} отслеживается время
с момента последнего сброса бомбы.
Если прошло достаточно времени (\texttt{dropInterval}),
вызывается функция \textbf{DropBall()},
которая создаёт бомбу в позиции сброса.

Код взаимодействия бомбы с игроком:                                            
\lstinputlisting[language=C,
	caption=\leftline{Взаимодействие бомбы с игроком}]{code/Enemy/BallBehavior.cs}

Скрипт \texttt{BallBehavior.cs} управляет поведением бомбы.
В методе \textbf{Update()} проверяется позиция бомбы.
Если она достигает определённой высоты (\texttt{dead\_y}), она уничтожается.
В методе \textbf{OnCollisionEnter(Collision collision)}
проверяется столкновение бомбы с землёй,
а в методе \textbf{OnTriggerEnter(Collider other)} --- столкновение с игроком.
При столкновении с игроком, игроку наносится урон, и бомба уничтожается.

\subsection{Финишная линия}

Код для обработки финиша:
\lstinputlisting[language=C
	, caption=\leftline{Генерация врагов}]
{code/Place/finishLine.cs}

Этот скрипт (\texttt{finishLine.cs}) отвечает за обработку пересечения
финишной линии игроком. В методе \textbf{OnTriggerEnter(Collider other)}
проверяется, столкнулся ли объект, пересекающий финишную линию,
с игроком (\texttt{Player}).
Если да, то загружается сцена с меню перезапуска уровня (\texttt{RestartMenu}).

\subsection{Пользовательский интерфейс}                                           
                                                                               
Код для отображения жизней пользователя:                                       
\lstinputlisting[language=C,
	caption=\leftline{Отображение жизней пользователя}]{code/UI/HealthSlider.cs}

Скрипт \texttt{HealthSlider.cs} управляет отображением текущего уровня
здоровья игрока на полосе здоровья (\texttt{slider}).
В методе \textbf{Start()} подписывается на событие изменения здоровья
из класса \texttt{Health}, чтобы обновлять отображение.
В методе \textbf{UpdateSlider(int currentHealth, int maxHealth)}
обновляются значение полосы здоровья и текст текущего уровня здоровья.

Код для обработки меню смерти игрока:                                          
\lstinputlisting[language=C,
	caption=\leftline{Меню смерти игрока}]{code/UI/MenuDeadManager.cs}

Скрипт \texttt{MenuDeadManager.cs} отвечает за обработку действий
в меню смерти игрока. Функция \textbf{Restart()} перезапускает игру,
загружая сцену "'Game"', а \textbf{Quit()} завершает приложение.

Код для обработки главного меню:                                               
\lstinputlisting[language=C,
	caption=\leftline{Главное меню}]{code/UI/MenuManager.cs}

Скрипт \texttt{MenuManager.cs} управляет главным меню игры.
Функция \textbf{Play()} запускает игру, загружая сцену "'Game"',
а \textbf{Quit()} завершает приложение.
Функции \textbf{Settings()} и \textbf{ExitSetting()} отвечают
за переключение между основным и настройочным экранами.

Код для обработки финишного меню:                                              
\lstinputlisting[language=C,
	caption=\leftline{Финишное меню}]{code/UI/MenuRestartManager.cs}

Скрипт \texttt{FinishMenuManager.cs} обрабатывает действия
в финишном меню игры. Функция \textbf{Restart()} перезапускает игру,
загружая сцену "'Game"', а \textbf{Quit()} завершает приложение.

