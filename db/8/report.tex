\section*{\LARGE Введение}
\addcontentsline{toc}{section}{Введение}

Данная практическая работа направлена на закрепление полученных навыков
в области разработки программного обеспечения и баз данных.
В рамках этой работы студентам предлагается создать веб-приложение,
предназначенное для управления расписанием в высших учебных заведениях.

\begin{itemize}
	\item Тема проекта: Расписание в ВУЗе
	\item Вид приложения: Веб-интерфейс
	\item СУБД: MongoDB
	\item Требования к БД:
	\begin{itemize}
		\item Необходим кластер, состоящий минимум из 3 узлов
		\item Для создания необходимо использовать Docker
		\item Не менее 5 таблиц в базе данных
		\item Не менее 1 связи Один-Ко-Многим
		\item Не менее 1 связи Один-К-Одному
		\item Не менее 1 связи Один-К-Одному
	\end{itemize}
	\item Требования к функционалу системы (для каждой таблицы):
	\begin{itemize}
		\item Отображение всех документов collection
		\item Добавление документа
		\item Удаление документа
		\item Поиск документа по какому-либо критерию
		\item Сортировка документов
	\end{itemize}
\end{itemize}

\textbf{Требования к отчету}:
В результате вам необходимо предоставить скриншоты настройки и
запуска кластера, подключения к базе данных в коде, методы, реализующие
запросы к базе данных, а также скриншоты работы приложения.\par
Для реализации данного проекта мы будем использовать
язык программирования Python и веб-фреймворк Django.
Python – это высокоуровневый интерпретируемый язык программирования,
который известен своей простотой и удобством в использовании.
Django – это мощный фреймворк для создания веб-приложений на Python,
который предоставляет множество инструментов
и компонентов для быстрой разработки веб-приложений.\par
Использование Python и Django для написания сайта позволит
нам эффективно реализовать функционал расписания для ВУЗа,
обеспечивая удобство использования и надежность работы приложения.
Кроме того, Django предоставляет интеграцию с различными базами данных,
что позволяет нам легко взаимодействовать с MongoDB для хранения данных
о расписании.\par
\textbf{Целью} работы является не только создание работающего приложения,
но и демонстрация понимания принципов работы
с базами данных и их интеграции в веб-приложения.

\clearpage

\section*{\LARGE Выполнение практической работы}
\addcontentsline{toc}{section}{Выполнение практической работы}

\section{Docker-compose}

Для обеспечения работы приложения в контейнеризованной среде
и управления необходимыми сервисами
был разработан файл \texttt{docker-compose.yml}.
Этот файл представляет собой описание сервисов и их конфигураций,
которые будут развернуты с помощью Docker Compose.\par
В соответствии с требованиями, изложенными во введении,
в файле \texttt{docker-compose.yml} определены следующие сервисы:

\begin{enumerate}
    \item \textbf{MongoDB сервисы
		(\texttt{mongo}, \texttt{mongo2}, \texttt{mongo3}):}
    \begin{itemize}
        \item Каждый сервис использует официальный образ MongoDB.
        \item Определены три экземпляра MongoDB,
			формирующих кластер из трех узлов.
        \item Каждый экземпляр настроен с именем пользователя \texttt{root}
			и паролем \texttt{mongoadmin}.
        \item Каждый экземпляр MongoDB прослушивает свой уникальный порт,
			чтобы избежать конфликтов.
        \item Зависимость между экземплярами MongoDB указана
			для правильного порядка инициализации.
    \end{itemize}
    
    \item \textbf{Mongo Express сервис (\texttt{mongo-express}):}
    \begin{itemize}
        \item Используется образ Mongo Express
			для веб-интерфейса управления MongoDB.
        \item Настроен доступ к MongoDB через сервис \texttt{mongo}
			с использованием учетных данных \texttt{root/mongoadmin}.
        \item Предоставлен доступ к веб-интерфейсу по порту 8081.
    \end{itemize}
    
    \item \textbf{Web сервис (\texttt{web}):}
    \begin{itemize}
        \item Сервис для запуска веб-приложения.
        \item Определен сборочный контекст
			для построения образа веб-приложения.
        \item После запуска контейнера выполняются команды
			по миграции базы данных и запуску веб-сервера.
    \end{itemize}
\end{enumerate}

Дополнительно, определена сеть \texttt{mongo-compose-network},
в которой все сервисы будут находиться для обеспечения связи между ними.\par
Файл \texttt{docker-compose.yml} был разработан в соответствии
с требованиями к созданию кластера MongoDB,
а также для обеспечения взаимодействия между сервисами приложения.

\begin{lstlisting}[language=bash]
version: '3'
services:
  mongo:
    image: mongo
    container_name: mongo
    restart: always
    environment:
      MONGO_INITDB_ROOT_USERNAME: root
      MONGO_INITDB_ROOT_PASSWORD: mongoadmin
      MONGO_INITDB_DATABASE: mongo
    ports:
      - 27017:27017
    networks:
      - mongo-compose-network
  mongo2:
    image: mongo
    container_name: mongo2
    restart: always
    environment:
      MONGO_INITDB_ROOT_USERNAME: root
      MONGO_INITDB_ROOT_PASSWORD: mongoadmin
      MONGO_INITDB_DATABASE: mongo
    ports:
      - 27018:27017
    depends_on:
      - mongo
    networks:
      - mongo-compose-network
  mongo3:
    image: mongo
    container_name: mongo3
    restart: always
    environment:
      MONGO_INITDB_ROOT_USERNAME: root
      MONGO_INITDB_ROOT_PASSWORD: mongoadmin
      MONGO_INITDB_DATABASE: mongo
    ports:
      - 27019:27017
    depends_on:
      - mongo2
    networks:
      - mongo-compose-network
  mongo-express:
    image: mongo-express
    container_name: mongo-express
    ports:
      - 8081:8081
    environment:
      ME_CONFIG_MONGODB_SERVER: mongo
      ME_CONFIG_BASICAUTH_USERNAME: admin
      ME_CONFIG_BASICAUTH_PASSWORD: q
      ME_CONFIG_MONGODB_PORT: 27017
      ME_CONFIG_MONGODB_ADMINUSERNAME: root
      ME_CONFIG_MONGODB_ADMINPASSWORD: mongoadmin
    links:
      - mongo
    networks:
      - mongo-compose-network
  web:
    build: .
    command: sh -c "
      python manage.py makemigrations
      && python manage.py migrate
      && python manage.py runserver 0.0.0.0:8000
      "
    volumes:
      - .:/app
    ports:
      - 8000:8000
    depends_on:
      - mongo
    networks:
      - mongo-compose-network
networks:
    mongo-compose-network:
      driver: bridge
\end{lstlisting}

Команда для запуска контейнеров на основе файла \texttt{docker-compose.yml}:

\begin{lstlisting}[language=bash]
docker-compose up -d --build
\end{lstlisting}

\section{Описание таблиц}

Опишем таблицы в виде моделей Django.

\subsection{Модель TinyGroup}

Модель \texttt{TinyGroup} представляет собой таблицу в базе данных,
предназначенную для хранения информации о малых группах.
Каждая запись в таблице содержит следующие поля:

\begin{itemize}
    \item \texttt{groupName}: Поле типа \texttt{CharField}
		с максимальной длиной 15 символов.
		Это поле предназначено для хранения имени малой группы.
    \item \texttt{groupSuffix}: Поле типа \texttt{CharField}
	с максимальной длиной 100 символов.
	Оно предназначено для хранения суффикса или описания малой группы.
\end{itemize}

Эта модель может быть использована в Django-приложении для создания,
сохранения и запроса информации о малых группах.
Например, она может быть использована в системе управления расписанием
в учебном заведении для хранения данных о группах студентов.

\begin{lstlisting}[language=bash]
class TinyGroup(models.Model):
    groupName = models.CharField(max_length=15)
    groupSuffix = models.CharField(max_length=100)
\end{lstlisting}


\subsection{Модель RichGroup}

Модель \texttt{RichGroup} представляет собой таблицу в базе данных,
которая содержит расширенную информацию о малых группах.
Каждая запись в таблице содержит следующие поля:

\begin{itemize}
    \item \texttt{group}: Поле типа \texttt{OneToOneField},
		которое связывает модель \texttt{RichGroup}
		с моделью \texttt{TinyGroup} через отношение "один к одному".
		При удалении связанной записи из модели \texttt{TinyGroup},
		связанная запись из модели \texttt{RichGroup} также будет удалена
		(с помощью параметра \texttt{on\_delete=models.CASCADE}).
    \item \texttt{remoteFile}: Поле типа \texttt{CharField}
		с максимальной длиной 100 символов.
		Оно предназначено для хранения пути
		к удаленному файлу или его названия.
    \item \texttt{unitName}: Поле типа \texttt{CharField}
		с максимальной длиной 100 символов.
		Оно предназначено для хранения названия подразделения,
		к которому относится малая группа.
    \item \texttt{unitCourse}: Поле типа \texttt{CharField}
		с максимальной длиной 100 символов.
		Оно предназначено для хранения информации о курсе,
		к которому относится малая группа.
    \item \texttt{updatedDate}: Поле типа \texttt{DateTimeField}.
		Оно предназначено для хранения даты и времени последнего
		обновления информации о малой группе.
\end{itemize}

Эта модель может быть использована в Django-приложении
для хранения более подробной информации о малых группах,
такой как информация о курсах и подразделениях,
а также для обновления информации о малой группе с отслеживанием даты
и времени обновления.

\begin{lstlisting}[language=bash]
class RichGroup(models.Model):
    group = models.OneToOneField(TinyGroup, on_delete=models.CASCADE)
    remoteFile = models.CharField(max_length=100)
    unitName = models.CharField(max_length=100)
    unitCourse = models.CharField(max_length=100)
    updatedDate = models.DateTimeField()
\end{lstlisting}


\subsection{Модель DaySchedule}

Модель \texttt{DaySchedule} представляет собой таблицу в базе данных,
которая содержит расписание занятий
для определенной группы на определенный день.
Каждая запись в таблице содержит следующие поля:

\begin{itemize}
    \item \texttt{group}: Поле типа \texttt{ForeignKey},
		которое связывает модель \texttt{DaySchedule}
		с моделью \texttt{RichGroup} через отношение "многие к одному".
		При удалении связанной записи из модели \texttt{RichGroup},
		все связанные записи из модели \texttt{DaySchedule}
		также будут удалены
		(с помощью параметра \texttt{on\_delete=models.CASCADE}).
    \item \texttt{day}: Поле типа \texttt{CharField}
		с максимальной длиной 100 символов.
		Оно предназначено для хранения информации о дне недели,
		к которому относится расписание.
\end{itemize}

Эта модель может быть использована в Django-приложении
для хранения расписания занятий для конкретных групп
на определенные дни недели.

\begin{lstlisting}[language=bash]
class DaySchedule(models.Model):
    group = models.ForeignKey(RichGroup, on_delete=models.CASCADE)
    day = models.CharField(max_length=100)
\end{lstlisting}

\subsection{Модель Lesson}

Модель \texttt{Lesson} представляет собой таблицу в базе данных,
которая содержит информацию о конкретном занятии в расписании.
Каждая запись в таблице содержит следующие поля:

\begin{itemize}
    \item \texttt{day\_schedule}: Поле типа \texttt{ForeignKey},
		которое связывает модель \texttt{Lesson}
		с моделью \texttt{DaySchedule} через отношение "многие к одному".
		При удалении связанной записи из модели \texttt{DaySchedule},
		все связанные записи из модели \texttt{Lesson} также будут удалены
		(с помощью параметра \texttt{on\_delete=models.CASCADE}).
    \item \texttt{is\_even\_week}: Поле типа \texttt{BooleanField},
		которое указывает, проходит ли занятие на четной неделе.
		По умолчанию установлено значение \texttt{True}.
    \item \texttt{name}: Поле типа \texttt{CharField}
		с максимальной длиной 100 символов.
		Оно предназначено для хранения названия занятия.
    \item \texttt{type}: Поле типа \texttt{CharField}
		с максимальной длиной 100 символов.
		Оно предназначено для хранения типа занятия
		(например, лекция, практика, семинар).
    \item \texttt{tutor}: Поле типа \texttt{CharField}
		с максимальной длиной 100 символов.
		Оно предназначено для хранения информации о преподавателе,
		ведущем занятие. Поле может быть пустым (\texttt{null=True}),
		если информация о преподавателе неизвестна или не применима.
    \item \texttt{place}: Поле типа \texttt{CharField}
		с максимальной длиной 100 символов.
		Оно предназначено для хранения информации о месте проведения занятия.
		Поле также может быть пустым (\texttt{null=True}),
		если информация о месте неизвестна или не применима.
    \item \texttt{link}: Поле типа \texttt{URLField}
		с максимальной длиной 200 символов.
		Оно предназначено для хранения ссылки на материалы или онлайн-ресурсы,
		связанные с занятием. Поле может быть пустым (\texttt{null=True}),
		если ссылка не предоставлена.
\end{itemize}

Эта модель может быть использована для хранения информации о занятиях
в рамках определенного расписания на определенный день.

\begin{lstlisting}[language=bash]
class Lesson(models.Model):
    day_schedule = models.ForeignKey(DaySchedule, on_delete=models.CASCADE)
    is_even_week = models.BooleanField(default=True)
    name = models.CharField(max_length=100)
    type = models.CharField(max_length=100)
    tutor = models.CharField(max_length=100, blank=True, null=True)
    place = models.CharField(max_length=100, blank=True, null=True)
    link = models.URLField(max_length=200, blank=True, null=True)
\end{lstlisting}

Модель \texttt{Stats} представляет собой таблицу в базе данных,
которая содержит статистическую информацию о системе.
Каждая запись в таблице содержит следующие поля:

\begin{itemize}
    \item \texttt{groupsCount}: Поле типа \texttt{IntegerField}.
		Оно предназначено для хранения количества групп,
		зарегистрированных в системе.
    \item \texttt{scrapperUpdatedDate}: Поле типа \texttt{DateTimeField}.
		Оно предназначено для хранения даты
		и времени последнего обновления информации,
		полученной скрапером или каким-либо другим внешним источником данных.
\end{itemize}

\begin{lstlisting}[language=bash]
class Stats(models.Model):
    groupsCount = models.IntegerField()
    scrapperUpdatedDate = models.DateTimeField()
\end{lstlisting}

\subsection{Модель Homework}

Модель \texttt{Homework} представляет собой таблицу в базе данных,
которая содержит информацию о домашних заданиях.
Каждая запись в таблице содержит следующие поля:

\begin{itemize}
    \item \texttt{title}: Поле типа \texttt{CharField}
		с максимальной длиной 100 символов. Оно предназначено
		для хранения заголовка или названия домашнего задания.
    \item \texttt{description}: Поле типа \texttt{TextField}.
		Оно предназначено для хранения описания домашнего задания.
		Поле может содержать большой объем текстовой информации.
    \item \texttt{due\_date}: Поле типа \texttt{DateField}.
		Оно предназначено для хранения даты сдачи домашнего задания.
    \item \texttt{created\_at}: Поле типа \texttt{DateTimeField}
		с параметром \texttt{auto\_now\_add=True}.
		Это поле автоматически устанавливается при создании записи
		и содержит дату и время создания записи.
\end{itemize}

Эта модель может быть использована для хранения информации
о домашних заданиях, их описании,
сроках сдачи и времени создания записей о них.

\begin{lstlisting}[language=bash]
class Homework(models.Model):
    title = models.CharField(max_length=100)
    description = models.TextField()
    due_date = models.DateField()
    created_at = models.DateTimeField(auto_now_add=True)
\end{lstlisting}

\section{Настройка подключения к базе данных}

Для обеспечения взаимодействия приложения Django
с базой данных MongoDB необходимо произвести соответствующую настройку
подключения. Для этого используется файл настроек `settings.py`
в проекте Django.\par
В данной главе рассмотрим пример конфигурации подключения
к базе данных MongoDB с использованием библиотеки Djongo.

\subsection{Настройка параметров базы данных}

В файле `settings.py` определяется словарь `DATABASES`,
в котором указываются параметры подключения к базе данных.
Приведенный ниже пример демонстрирует конфигурацию подключения
к базе данных MongoDB:

\begin{lstlisting}
	language=Python,
	caption={Настройка подключения к базе данных в Django}
	]
DATABASES = {
    'default': {
        'ENGINE': 'djongo',
        'ENFORCE_SCHEMA': True,
        'NAME': 'mongo',
        "CLIENT": {
            "host": 'mongo',
            "port": 27017,
            "username": 'root',
            "password": 'mongoadmin',
        },
    }
}
\end{lstlisting}

В данной конфигурации:

\begin{itemize}
    \item \texttt{ENGINE}: Указывает на использование бэкенда Djongo
		для взаимодействия с MongoDB.
    \item \texttt{ENFORCE\_SCHEMA}: Указывает, следует ли применять
		схему базы данных при миграциях (по умолчанию \texttt{True}).
    \item \texttt{NAME}: Имя базы данных,
		с которой приложение будет взаимодействовать.
    \item \texttt{CLIENT}: Параметры клиента для подключения к MongoDB.
		В данном случае указаны хост, порт, имя пользователя и пароль.
\end{itemize}

После настройки параметров подключения приложение Django
будет готово использовать базу данных MongoDB для хранения данных.

\subsection{Применение миграций}

После настройки подключения к базе данных следует выполнить миграции, чтобы создать необходимые таблицы и индексы в MongoDB. Для этого используется команда:

\begin{verbatim}
python manage.py migrate
\end{verbatim}

Эта команда автоматически создаст необходимые структуры в базе данных в соответствии с моделями Django.

\section{Создание команды для получения данных из внешнего API}

Для загрузки данных из внешнего API в модели Django можно создать
специальную команду, которая будет выполнять эту задачу.
В данной главе будет описан процесс создания такой команды
на примере получения информации о группах из внешнего API.

\subsection{Описание команды}

Сначала определяется класс команды,
который наследуется от базового класса \texttt{BaseCommand}:

\begin{lstlisting}
	language=Python,
	caption={Определение команды для получения данных из API}
	]
from django.core.management.base import BaseCommand
from myapp.models import Stats, TinyGroup, RichGroup, DaySchedule, Lesson
import requests
import json
class Command(BaseCommand):
    url = "https://mirea.xyz/api/v1.3/"
    groups = "groups/all"
    certain = "groups/certain"
    status = "stats"
    help = 'Load data from API into Django models'
    def handle(self, *args, **options):
\end{lstlisting}

В методе \texttt{handle} определяется логика загрузки данных
из API и сохранения их в моделях Django.

\subsection{Загрузка данных из API}

В методе \texttt{handle} производится запрос к API
для получения информации о группах и статистики.
Полученные данные затем сохраняются в моделях Django:

\begin{lstlisting}[language=Python, caption={Загрузка данных из API}]
def handle(self, *args, **options):
    stats = requests.get(f"{self.url}{self.status}").json()
    stats = Stats.objects.create(
        groupsCount=stats['groupsCount'],
        scrapperUpdatedDate=stats['scrapperUpdatedDate']
    )
    stats.save()
    with open("res/data/data.json") as f:
        groups = json.load(f)
    for group in groups:
        group_name = group['groupName']
        group_suffix = group['groupSuffix']
        t_group, _ = TinyGroup.objects.get_or_create(
			groupName=group_name, groupSuffix=group_suffix
			)
        data = requests.get(
            f"{self.url}{self.certain}",
            params={'name': group_name, 'suffix': group_suffix}
        ).json()
        for group_data in data:
            r_group, _ = RichGroup.objects.update_or_create(
                group=t_group,
                defaults={
                    "remoteFile": group_data["remoteFile"],
                    "unitName": group_data["unitName"],
                    "unitCourse": group_data["unitCourse"],
                    "updatedDate": group_data["updatedDate"]
                }
            )
            for day_schedule_data in group_data["schedule"]:
                day_schedule, _ = DaySchedule.objects.get_or_create(
                    group=r_group,
                    day=day_schedule_data["day"]
                )
                for f_even, lessons in enumerate(
						[day_schedule_data['odd'], day_schedule_data['even']]):
                    for dict_lesson in lessons:
                        for lesson in dict_lesson:
                            lsn = Lesson.objects.create(
                                day_schedule=day_schedule,
                                is_even_week=f_even,
                                name=lesson["name"],
                                type=lesson["type"],
                                tutor=lesson.get("tutor"),
                                place=lesson.get("place"),
                                link=lesson.get("link")
                            )
                            lsn.save()
\end{lstlisting}

Этот код иллюстрирует процесс загрузки данных из внешнего API
и сохранения их в моделях Django.

\subsection{Использование команды}

После создания команды ее можно использовать
из командной строки Django для загрузки данных из API:

\begin{verbatim}
python manage.py load_data_from_api
\end{verbatim}

Это выполнит метод \texttt{handle} команды, который загрузит данные из API и сохранит их в моделях Django.

\section{Работа приложения в действии}

В данной главе представлены скриншоты работы веб-приложения,
созданного на основе разработанных моделей и функционала.

\subsection{Главная страница приложения}

\begin{image}
	\includegrph{Screenshot from 2024-05-11 13-27-36}
	\caption{Объекты}
    \label{fig:homepage}
\end{image}

На рисунке~\ref{fig:homepage} показана главная страница приложения, на которой пользователь может видеть общий обзор расписания вуза.

\subsection{Страница просмотра групп}

\begin{image}
	\includegrph{Screenshot from 2024-05-11 13-28-05}
    \caption{Страница просмотра групп}
    \label{fig:groups_page}
\end{image}

На рисунке~\ref{fig:groups_page} показана страница,
на которой пользователь может просматривать список доступных групп.

\subsection{Страница просмотра занятий}

\begin{image}
	\includegrph{Screenshot from 2024-05-11 13-28-05}
    \caption{Страница просмотра занятий}
    \label{fig:lesson_page}
\end{image}

На рисунке~\ref{fig:lesson_page} показана страница,
на которой пользователь может просматривать расписание группы.

\subsection{Страница просмотра статуса парсинга расписания}

\begin{image}
	\includegrph{Screenshot from 2024-05-11 13-27-40}
    \caption{Страница просмотра статуса парсинга расписания}
    \label{fig:status_page}
\end{image}

На рисунке~\ref{fig:status_page} показана страница,
на которой пользователь может просматривать статуса парсинга расписания.

\subsection{Страница домашнего задания}

\begin{image}
	\includegrph{Screenshot from 2024-05-11 13-28-25}
    \caption{Страница домашнего задания}
    \label{fig:homework_page}
\end{image}

На рисунке~\ref{fig:homework_page} показана страница,
на которой пользователь может просматривать, создавать и удалять
домашние задания.

\clearpage

\section*{\LARGE Вывод}
\addcontentsline{toc}{section}{Вывод}

В результате выполнения данной практической работы было создано
веб-приложение для управления расписанием в высших учебных заведениях.
Для разработки приложения был выбран язык программирования Python
и веб-фреймворк Django, что позволило эффективно реализовать
требуемый функционал.\par
Основные функции приложения включают отображение расписания,
добавление новых записей, удаление и поиск записей по различным критериям,
а также сортировку данных. Пользовательский интерфейс разработан
с учетом простоты использования и интуитивной навигации,
что делает работу с приложением удобной и эффективной.\par
Для хранения данных была использована база данных MongoDB,
что обеспечило надежное хранение и быстрый доступ к информации.
Кроме того, был создан кластер MongoDB, состоящий из трех узлов,
что повышает отказоустойчивость и масштабируемость системы.\par
В процессе разработки были реализованы все требования к базе данных
и функционалу системы, а также успешно выполнены все задачи,
поставленные перед командой. Полученный опыт разработки веб-приложения
позволил закрепить и расширить знания по использованию Python и Django
для создания современных веб-приложений.\par
В целом, созданное приложение представляет собой полнофункциональное
решение для управления расписанием в ВУЗе и может быть дальше развито
и оптимизировано для удовлетворения потребностей конечных пользователей.\par
Исходный код проекта доступен на GitHub по следующей ссылке:
\url{https://github.com/AndB0ndar/db_web_schedule}.

