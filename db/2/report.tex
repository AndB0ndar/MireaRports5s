\section*{\LARGE Введение}
\addcontentsline{toc}{section}{Введение}

Распределенные базы данных можно широко классифицировать на
однородные и гетерогенные среды распределенных баз данных, каждая из
которых имеет дополнительные подразделения.

Свойства однородных распределенных баз данных:
\begin{itemize}
	\item На сайтах используется очень похожее программное обеспечение.
	\item Сайты используют идентичные СУБД или СУБД одного и того же
		производителя.
	\item Каждый сайт знает обо всех других сайтах и взаимодействует с
		другими сайтами для обработки пользовательских запросов.
	\item Доступ к базе данных осуществляется через единый интерфейс, как
		если бы это была одна база данных.
\end{itemize}

Типы однородной распределенной базы данных:
\begin{itemize}
	\item Автономный — каждая база данных независима и функционирует
		самостоятельно. Они интегрированы управляющим приложением и
		используют передачу сообщений для обмена обновлениями данных.
	\item Неавтономный — данные распределяются по однородным узлам, а
		центральная или главная СУБД координирует обновления данных по
		сайтам.
\end{itemize}

Свойства гетерогенных распределенных баз данных:
\begin{itemize}
	\item Система может состоять из множества СУБД, таких как реляционная,
		сетевая, иерархическая или объектно-ориентированная.
	\item Обработка запросов является сложной из-за разнородных схем.
	\item Обработка транзакций является сложной из-за различий в
		программном обеспечении.
	\item Сайт может не знать о других сайтах, поэтому сотрудничество при
		обработке пользовательских запросов ограничено.
\end{itemize}

Типы гетерогенных распределенных баз данных:
\begin{itemize}
	\item Федеративные — гетерогенные системы баз данных независимы по
		своей природе и объединены вместе, так что они функционируют как
		единая система баз данных.
	\item Без федерации — в системах баз данных используется центральный
		координационный модуль, через который осуществляется доступ к
		базам данных.
\end{itemize}

\textbf{Цель работы} --- научиться создавать кластеры в Apache Cassandra.

\clearpage

\section*{\LARGE Выполнение практической работы}
\addcontentsline{toc}{section}{Выполнение практической работы}

\section{Создание Docker контейнеров}

Запускаем контейнер с первым узлом кассандры, где
\texttt{CASSANDRA\_CLUSTER\_NAME} можно задать своё, например,
инициалы студента:

\begin{lstlisting}[language=bash]
docker run --name cas1 -p 9042:9042 -e
CASSANDRA\_CLUSTER\_NAME=ArbonCluster -e
CASSANDRA\_ENDPOINT\_SNITCH=GossipingPropertyFileSnitch -e
CASSANDRA\_DC=datacenter1 -d cassandra
\end{lstlisting}

Присоединяем к нему следующий узел, где
\texttt{CASSANDRA\_CLUSTER\_NAME} задаём такой же, как в первом узле:

\begin{lstlisting}[language=bash]
docker run --name cas2
	-e CASSANDRA\_SEEDS=
		"\$(docker inspect --format='{{ .NetworkSettings.IPAddress }}' cas1)"
	-e CASSANDRA\_CLUSTER\_NAME=ArbonCluster
	-e CASSANDRA\_ENDPOINT\_SNITCH=GossipingPropertyFileSnitch
	-e CASSANDRA\_DC=datacenter1
	-d cassandra:latest
\end{lstlisting}

(docker
\texttt{inspect \-\-format='\{\{ .NetworkSettings.IPAddress \}\}' cas1}
)~--- это вложенный скрипт при помощи которого мы можем получить
IP-адрес первого узла, тем самым связывая два узла в данном пункте.\par
Присоединяем к первому узлу следующий узел, где
\texttt{CASSANDRA\_CLUSTER\_NAME} задаём такой же, как в первом узле,
но меняем значение \texttt{CASSANDRA\_DC=datacenter1} на \texttt{datacenter2},
тем самым создаём новый узел в «как бы удалённом» датацентре:

\begin{lstlisting}[language=bash]
docker run --name cas3
	-e CASSANDRA\_SEEDS=
		"\$(docker inspect --format='{{ .NetworkSettings.IPAddress }}' cas1)"
	-e CASSANDRA\_CLUSTER\_NAME=ArbonCluster
	-e CASSANDRA\_ENDPOINT\_SNITCH=GossipingPropertyFileSnitch
	-e CASSANDRA\_DC=datacenter2
	-d cassandra:latest
\end{lstlisting}

Теперь у нас есть целый кластер кассандры, в котором все данные
синхронизированы, таким образом все модификации с бд, которые
будут проводиться будут применены ко всему кластеру и неважно с
какого узла эти самые изменения произошли (рис.~\ref{fig:docker}).

\begin{image}
	\includegrph{Screenshot from 2024-02-18 20-11-10}
	\caption{Создание докер контейнеров с Apache Cassandra}
	\label{fig:docker}
\end{image}

Для того чтобы удостовериться, что весь кластер работает правильно
необходимо выполнить команду \texttt{docker exec -ti cas1 nodetool status}
(рис.~\ref{fig:status}).

\begin{image}
	\includegrph{Screenshot from 2024-02-18 20-12-53}
	\caption{Вывод статуса узлов}
	\label{fig:status}
\end{image}

\section{Создание своей БД}

Теперь необходимо создать свою схему бд (в кассандре это
пространство ключей) для этого необходимо перейти в утилиту \texttt{cqlsh} 
при помощи команды \texttt{docker exec -ti cas1 cqlsh} и выполнить скрипт,
где название кейспейса можно придумать самому (желательно номер
студенческого или другой идентификатор).

\begin{lstlisting}[language=bash]
CREATE KEYSPACE arbon1874 WITH replication = {
	'class' : 'NetworkTopologyStrategy',
	'datacenter1' : 1,
	'datacenter2' : 1
	};
\end{lstlisting}

\begin{image}
	\includegrph{Screenshot from 2024-02-18 21-27-55}
	\caption{Создание пространства ключей}
	\label{fig:keyspace}
\end{image}

Структура создаваемой базы данный показана на рисунке~\ref{fig:eer}.

\begin{image}
	\includegrph{Screenshot from 2024-02-20 14-24-05}
	\caption{Создание пространства ключей}
	\label{fig:eer}
\end{image}

Далее создадим таблицы (рис. \ref{fig:first}~-~\ref{fig:last}).

\begin{image}
	\includegrph{Screenshot from 2024-02-18 21-26-14}
	\caption{Таблица factory}
	\label{fig:first}
\end{image}

\begin{image}
	\includegrph{Screenshot from 2024-02-18 21-25-58}
	\caption{Таблица material}
\end{image}

\begin{image}
	\includegrph{Screenshot from 2024-02-18 21-26-22}
	\caption{Таблица equipment}
\end{image}

\begin{image}
	\includegrph{Screenshot from 2024-02-18 21-26-44}
	\caption{Таблица type\_equipment}
	\label{fig:last}
\end{image}

\clearpage

Теперь проверим, что на всех узлах есть созданные таблицы
(рис.~\ref{fig:check}).

\begin{image}
	\includegrph{check.png}
	\caption{Содержимое пространства ключей на разных узлах}
	\label{fig:check}
\end{image}

\clearpage

\section*{\LARGE Вывод}
\addcontentsline{toc}{section}{Вывод}
В результате практической работы была изучена работа
с распределенной бд, используя Apache Cassandra.\par
Были созданы докер контейнеры прдеставляющие разные узлы.
И базе данных созданы таблицы.

