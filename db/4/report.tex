\section*{\LARGE Введение}
\addcontentsline{toc}{section}{Введение}

\paragraph{ОТСУТСТВИЕ JOIN}

В Cassandra нет опции join. Чаще всего, это ограничение обходится с
помощью денормализации данных по дополнительным таблицам.

\paragraph{ОТСУТСТВИЕ ВНЕШНИХ КЛЮЧЕЙ}

В Cassandra нет способа связывать сущности из разных таблиц между
собой, как например, это сделано в реляционных базах с помощью внешних
ключей. Как следствие отсутствуют такие операции как каскадное удаление.

\paragraph{ДЕНОРМАЛИЗАЦИЯ}

Сама по себе денормализация обладает одним достоинством, в отличие от
обычного подход. Ее удобно использовать для хранения исторических данных,
которые ни при каких обстоятельствах не должны изменяться. В качестве
примера можно привести хранение покупок клиента со ссылками на товар. Т.к.
цена товара меняется с течением времени, чтобы получить стоимость каждого
товара на момент приобретения, нужно отдельно хранить цену товара. При
использовании подхода с денормализацией, можно в таблице с покупками
хранить полностью объект товара. При этом текущий товар можно изменять
любым образом и даже удалять и это никак не повлияет на историю покупок
клиента.

\paragraph{ЗАПРОС ПЕРВИЧЕН}

При проектировании, нужно учитывать, что запрос должен получить за
один раз все данные из одной таблицы (т.к. нет join). Поэтому сначала
проектируются все возможные запросы, а затем под них создаются таблицы.
Таким образом в Cassandra создание структуры БД начинается с определения
запросов.

\paragraph{ОПТИМАЛЬНОЕ ХРАНЕНИЕ}

В реляционных базах редко можно встретить рекомендации к структуре
БД для оптимального хранения и чтения данных. Чаще всего никто не заботится
об этом. В распределенных системах дело обстоит иначе, т.к. данные
располагаются на нескольких узлах, более высокую производительно будет
демонстрировать тот запрос, который отдает данные с одной ноды. Таким
образом, желательно располагать данные так, чтобы данные возвращались из
одной ноды.

\section*{\LARGE Выполнение практической работы}
\addcontentsline{toc}{section}{Выполнение практической работы}

\section{Создание кластера на 3 узла с помощью Docker Compose}

В первой части практики необходимо составить docker-compose.yml.
Требования, предъявляемые к кластеру:
\begin{itemize}
	\item 3 сервиса: cassandra-seed-bondar, cassandra-node-bondar\_1
		, cassandra-node-bondar\_2
	\item Для каждого сервиса изменить (использовать) следующие
	стандартные переменные в environment:
	\begin{itemize}
		\item CASSANDRA\_CLUSTER\_NAME
		\item CASSANDRA\_ENDPOINT\_SNITCH
		\item CASSANDRA\_DC
	\end{itemize}
	\item Узлы cassandra-seed-\ldots и cassandra-node-\ldots1 должны
		находиться в одинаковых датацентрах
	\item Для cassandra-node-\ldots1 и cassandra-node-\ldots2 добавить
		настройку depends\_on в значении которой указать cassandra-seed-\ldots
	\item Создать собственный volume для каждого сервиса чтобы было проще
		искать данные
		(пример \texttt{"cassandra\_data\_seed:/var/lib/cassandra"})
	\item Открыть порты для каждого сервиса (пример 19042:9042)
	\item Для главного узла cassandra-seed-*фамилия студента*
		необходимо написать в environment\texttt{AUTO\_BOOTSTRAP=false} 
\end{itemize}

\lstinputlisting[language=xml]{./4/code/docker-compose.yml}

Запустим его (рис. \ref{fig:docker:compouse}).

\begin{image}
	\includegrph{Screenshot from 2024-03-23 11-26-25}
	\caption{Запуск Docker compouse}
	\label{fig:docker:compouse}
\end{image}

\section{Ключи в Cassandra}

Для начала подключимся к любому из контейнеру.

\begin{verbatim}
	docker exec -it cas1 cqlsh
\end{verbatim}

Создадим пространство ключей:

\begin{verbatim}
	CREATE KEYSPACE learn_cassandra WITH REPLICATION = {
	'class' : 'NetworkTopologyStrategy',
	'datacenter1' : 2
	};
\end{verbatim}

Создадим таблицу:

\begin{verbatim}
CREATE TABLE learn_cassandra.users_by_country (
	country text,
	user_email text,
	first_name text,
	last_name text,
	age smallint,
	PRIMARY KEY ((country), user_email)
);
\end{verbatim}

\begin{image}
	\includegrph{Screenshot from 2024-03-23 11-26-25}
	\caption{Содание KEYSPACE и таблицы users\_by\_country}
	\label{fig:keyspace}
\end{image}

Давайте наполним таблицу некоторыми данными

\begin{verbatim}
INSERT INTO learn_cassandra.users_by_country
(country,user_email,first_name,last_name,age)
 VALUES('US', 'john@email.com', 'John','Wick',55);
INSERT INTO learn_cassandra.users_by_country
(country,user_email,first_name,last_name,age)
 VALUES('UK', 'peter@email.com', 'Peter','Clark',65);
INSERT INTO learn_cassandra.users_by_country
(country,user_email,first_name,last_name,age)
 VALUES('UK', 'bob@email.com', 'Bob','Sandler',23);
INSERT INTO learn_cassandra.users_by_country
(country,user_email,first_name,last_name,age)
 VALUES('UK', 'alice@email.com', 'Alice','Brown',26);
\end{verbatim}

\begin{image}
	\includegrph{Screenshot from 2024-03-23 11-30-05}
	\caption{Заполнение таблицы users\_by\_country}
	\label{fig:insert:users-by-country}
\end{image}

Давайте создадим еще одну таблицу и заполним их записями.
Раздел будет определяться только столбцом user\_email.

\begin{verbatim}
CREATE TABLE learn_cassandra.users_by_email (
 user_email text,
 country text,
 first_name text,
 last_name text,
 age smallint,
 PRIMARY KEY (user_email)
);
INSERT INTO learn_cassandra.users_by_email (user_email,
country,first_name,last_name,age)
 VALUES('john@email.com', 'US', 'John','Wick',55);
INSERT INTO learn_cassandra.users_by_email
(user_email,country,first_name,last_name,age)
 VALUES('peter@email.com', 'UK', 'Peter','Clark',65);
INSERT INTO learn_cassandra.users_by_email
(user_email,country,first_name,last_name,age)
 VALUES('bob@email.com', 'UK', 'Bob','Sandler',23);
INSERT INTO learn_cassandra.users_by_email
(user_email,country,first_name,last_name,age)
 VALUES('alice@email.com', 'UK', 'Alice','Brown',26);
\end{verbatim}

\begin{image}
	\includegrph{Screenshot from 2024-03-23 11-31-15}
	\caption{Создание и заполнение таблицы users\_by\_email}
	\label{fig:table:users-by-email}
\end{image}

Кассандра старается избегать вредных запросов.
Если вы хотите выполнить фильтрацию по столбцу,
который не является ключом раздела, вам необходимо это явно указать,
как показано в следующей команде:

\begin{verbatim}
SELECT * FROM learn_cassandra.users_by_email WHERE age=26 ALLOW FILTERING;
\end{verbatim}

\begin{image}
	\includegrph{Screenshot from 2024-03-23 11-32-00}
	\caption{Фильтрация по столбцу, который не является ключом раздела}
	\label{fig:filter:not-key}
\end{image}

\section{Согласованность}

Мы устанавливали коэффициент репликации равным 2.
Таким образом, можно использовать уровень согласованности ALL или TWO.
Для выбора уровня согласованности пропишем команды,
а дальше получим данные из БД.

\begin{verbatim}
CONSISTENCY ALL;
SELECT * FROM learn_cassandra.users_by_country WHERE country='US';
\end{verbatim}

\begin{image}
	\includegrph{Screenshot from 2024-03-23 11-32-41}
	\caption{Выбор уровня согласованности и вывод данных}
	\label{fig:consistency:all}
\end{image}

Если высокая согласованность не важна, то можно прописать уровень
согласованности, равный 1.
За счет этого повышается скорость обработки запросов

\begin{verbatim}
CONSISTENCY ONE;
SELECT * FROM learn_cassandra.users_by_country WHERE country='US';
\end{verbatim}

\begin{image}
	\includegrph{Screenshot from 2024-03-23 11-33-13}
	\caption{Выбор уровня согласованности и вывод данных}
	\label{fig:consistency:one}
\end{image}

Основой для хранения данных являются так называемые SSTables.\par
SSTables --- это неизменяемые файлы данных, которые Cassandra использует для
сохранения данных на диске.\par
Вы можете задать различные стратегии для таблицы, которые определяют,
как данные должны быть объединены и сжаты. Эти стратегии влияют на
производительность чтения и записи:

\begin{itemize}
	\item SizeTieredCompactionStrategy используется по умолчанию и особенно
		эффективна, если у вас больше операций записи, чем операций чтения.
	\item LeveledCompactionStrategy оптимизирует чтение вместо записи. Эта
		оптимизация может быть дорогостоящей
		и должна быть тщательно опробована в продакшене.
	\item TimeWindowCompactionStrategy предназначена для данных временных
		рядов.
\end{itemize}

По умолчанию в таблицах используется SizeTieredCompactionStrategy.
Проверим это командой:

\begin{verbatim}
DESCRIBE TABLE learn_cassandra.users_by_country;
\end{verbatim}

\begin{image}
	\includegrph{Screenshot from 2024-03-23 11-33-43}
	\caption{Вывод информации о таблицы}
	\label{fig:info}
\end{image}

В таблице users\_by\_country можно определить age как еще один столбец
кластеризации для сортировки данных.

\begin{verbatim}
CREATE TABLE learn_cassandra.users_by_country_sorted_by_age_asc (
 country text,
 user_email text,
 first_name text,
 last_name text,
 age smallint,
 PRIMARY KEY ((country), age, user_email)) WITH CLUSTERING ORDER BY (age ASC);
\end{verbatim}

Вставьте данные в таблицу и удостоверьтесь, что записи придут в порядке
увеличения возраста

\begin{verbatim}
INSERT INTO learn_cassandra.users_by_country_sorted_by_age_asc
(country,user_email,first_name,last_name,age) VALUES('US','john@email.com',
'John','Wick',10);
INSERT INTO learn_cassandra.users_by_country_sorted_by_age_asc
(country,user_email,first_name,last_name,age) VALUES('UK', 'peter@email.com',
'Peter','Clark',30);
INSERT INTO learn_cassandra.users_by_country_sorted_by_age_asc
(country,user_email,first_name,last_name,age) VALUES('UK', 'bob@email.com',
'Bob','Sandler',20);
INSERT INTO learn_cassandra.users_by_country_sorted_by_age_asc
(country,user_email,first_name,last_name,age) VALUES('UK', 'alice@email.com',
'Alice','Brown',40);
SELECT * FROM learn_cassandra.users_by_country_sorted_by_age_asc;
\end{verbatim}

\begin{image}
	\includegrph{Screenshot from 2024-03-23 11-36-33}
	\caption{Создание и заполнение таблицы, а также вывод данных}
	\label{fig:filter:key}
\end{image}

\section*{\LARGE Вывод}
\addcontentsline{toc}{section}{Вывод}
В результате практической работы были изучены
особенности при создании таблиц Cassandra,
ключи и согласованность в Cassandra.\par
Также изучили стратегии для таблиц, которые определяют,
как данные должны быть объединены и сжаты.

