\section*{\LARGE Введение}
\addcontentsline{toc}{section}{Введение}

Столбцы и таблицы поддерживают необязательный срок действия,
называемый TTL (time-to-live); TTL не поддерживается для столбцов счетчиков.
Значение TTL задается в секундах. Срок действия данных истекает, когда они
превышают период TTL, после чего они помечаются меткой tombstone. Данные
с истекшим сроком действия по-прежнему доступны для запросов на чтение в
течение определенного периода. Обычные процессы сжатия данных и починки
автоматически удаляют tombstones.\par
В Cassandra удаленные данные не удаляются сразу с диска. Вместо этого
Cassandra записывает специальное значение, известное как tombstone, чтобы
указать, что данные были удалены.\par
Назначение:\par
Эти значения предотвращают возврат удаленных данных во время чтения
и в конечном итоге позволяют удалить данные посредством сжатия.\par
Tombstones являются частью реализации модели хранения данных (во
время записи существующие данные не читаются).\par
Tombstones --- это записи---  они проходят обычный путь записи,
занимают место на диске и используют механизмы согласованности Cassandra.\par
Если кластер управляется правильно, это гарантирует, что данные останутся
удаленными, даже если узел не работает, когда выполняется удаление.\par
Tombstones генерируются при помощи:

\begin{itemize}
	\item операции DELETE;
	\item установки TTL;
	\item вставки null значений;
	\item вставки данных в части коллекций.
\end{itemize}

\clearpage

\section*{\LARGE Выполнение практической работы}
\addcontentsline{toc}{section}{Выполнение практической работы}

\section{Утилита sstabledump и ее установка}

Для начала необходимо скачать образ Cassandra и запустить контейнер.
Далее нужно зайти в консоль контейнера.

\begin{lstlisting}[language=bash]
docker pull cassandra
docker run -d --name cas cassandra
docker exec -ti cas bash
\end{lstlisting}

\begin{image}
	\includegrph{Screenshot from 2024-03-18 12-24-42}
	\caption{Создание образа Cassandra}
	\label{fig:docker:cas}
\end{image}

Далее:
\begin{itemize}
	\item добавим репозиторий с инструментами Cassandra (echo …);
	\item обновим зависимости;
	\item скачаем необходимые пакеты для работы (curl, gnupg);
	\item подпишем репозиторий для скачивания необходимого нам пакета;
	\item обновим зависимости с целью поттягивания репозитория;
	\item скачаем пакет cassandra-tools.
\end{itemize}

После этого его можно использовать.

\begin{lstlisting}[language=bash]
\$ echo "deb https://debian.cassandra.apache.org 40x main" | tee –a
	/etc/apt/sources.list.d/cassandra.sources.list
\$ apt update \&\& apt upgrade
\$ apt install curl
\$ apt-get install gnupg
\$ curl https://downloads.apache.org/cassandra/KEYS | apt-key add -
\$ apt-get update
\$ apt-get install cassandra-tools
\end{lstlisting}

\begin{image}
	\includegrph{Screenshot from 2024-03-18 12-26-33}
	\caption{Настрока контейнера}
	\label{fig:docker:bash}
\end{image}

\section{Создание keyspace и таблиц}

В данной практической работе будет использоваться keyspace "cycling"
для того, чтобы продемонстрировать различные варианты tombstone. Будут
использоваться следующие таблицы:

\begin{itemize}
	\item rand\_by\_year\_and\_cycling\_name
	\item cyclist\_career\_teams
	\item calendar
\end{itemize}

Для начала необходимо перейти в \texttt{cqlsh} консоль для создания keyspace,
2 таблиц, и вставки записей в rank\_by\_year\_and\_cycling\_name. Вставка
записей в cyclist\_career\_teams и calendar будет произведена
позднее

\begin{lstlisting}[language=bash]
CREATE KEYSPACE cycling WITH replication =
{'class': 'SimpleStrategy', 'replication\_factor': '1'} AND durable\_writes =
true;

CREATE TABLE cycling.rank\_by\_year\_and\_name (
 race\_year int,
 race\_name text,
 rank int,
 cyclist\_name text,
 PRIMARY KEY ((race\_year, race\_name), rank)
) WITH CLUSTERING ORDER BY (rank ASC);

INSERT INTO cycling.rank\_by\_year\_and\_name
(race\_year, race\_name, cyclist\_name, rank)
VALUES
(2015, 'Tour of Japan - Stage 4 - Minami > Shinshu', 'Benjamin PRADES', 1);

INSERT INTO cycling.rank\_by\_year\_and\_name
(race\_year, race\_name, cyclist\_name, rank)
VALUES
(2015, 'Tour of Japan - Stage 4 - Minami > Shinshu', 'Adam PHELAN', 2);

INSERT INTO cycling.rank\_by\_year\_and\_name
(race\_year, race\_name, cyclist\_name, rank)
VALUES
(2015, 'Tour of Japan - Stage 4 - Minami > Shinshu', 'Thomas

LEBAS', 3);
INSERT INTO cycling.rank\_by\_year\_and\_name
(race\_year, race\_name, cyclist\_name, rank)
VALUES
(2015, 'Giro d''Italia - Stage 11 - Forli > Imola', 'Ilnur

ZAKARIN', 1);
INSERT INTO cycling.rank\_by\_year\_and\_name
(race\_year, race\_name, cyclist\_name, rank)
VALUES
(2015, 'Giro d''Italia - Stage 11 - Forli > Imola', 'Carlos

BETANCUR', 2);
INSERT INTO cycling.rank\_by\_year\_and\_name
(race\_year, race\_name, cyclist\_name, rank)
VALUES
(2014, '4th Tour of Beijing', 'Phillippe GILBERT', 1);

INSERT INTO cycling.rank\_by\_year\_and\_name
(race\_year, race\_name, cyclist\_name, rank)
VALUES
(2014, '4th Tour of Beijing', 'Daniel MARTIN', 2);

INSERT INTO cycling.rank\_by\_year\_and\_name
(race\_year, race\_name, cyclist\_name, rank)
VALUES
(2014, '4th Tour of Beijing', 'Johan Esteban CHAVES', 3);

CREATE TABLE cycling.cyclist\_career\_teams (
 id UUID PRIMARY KEY,
 lastname text,
 teams set<text>
);
CREATE TABLE cycling.calendar (
 race\_id int,
 race\_name text,
 race\_start\_date timestamp,
 race\_end\_date timestamp,
 PRIMARY KEY (race\_id, race\_start\_date, race\_end\_date)
);
\end{lstlisting}

\begin{image}
	\includegrph{Screenshot from 2024-03-18 12-26-33}
	\caption{Заполнение базы данных}
	\label{fig:cqlsh:create}
\end{image}

\section{Выгрузка в SSTable}

После заполнения таблиц необходимо выполнить команду nodetool
flush для созданного keyspace. Она необходима для выгрузки из кэша в
SSTable-файлы, для дальнейшего использования.

\begin{lstlisting}[language=bash]
\$ nodetool flush cycling;
\end{lstlisting}

После выполнения команды необходимо получить выгрузку в формате
json с помощью утилиты sstabledump из файла. Для начала необходимо перейти
в соответствующую директорию cycling и найти rank\_by\_year\_and\_name-<...>.
Название папки может частично отличаться.
Чтобы узнать имя необходимой директории перейдите
в /var/lib/cassandra/data/cycling и введите команду \texttt{ls}.

\begin{lstlisting}[language=bash]
\$ cd /var/lib/cassandra/data/cycling/
	rank\_by\_year\_and\_namebc05fba12baf11e8b4a8ad2b042f3e18
\end{lstlisting}

После этого выполняем команду sstabledump для файла "<mc-2-big-Data.db">.

\begin{lstlisting}[language=bash]
\$ sstabledump mc-2-big-Data.db
\end{lstlisting}

\begin{image}
	\includegrph{Screenshot from 2024-03-18 13-43-09}
	\caption{Выгрузка в формате json}
	\label{fig:sstabledump:json}
\end{image}

\section{Tombstone для разделов}

Tombstone для разделов (Partition tombstones) создаются, когда весь
раздел (partition) удален. Можно увидеть, что условие WHERE совпадает с
ключом раздела (partition key).

\begin{image}
	\includegrph{Screenshot from 2024-03-18 13-46-28}
	\caption{Удаление раздела}
	\label{fig:cqlsh:partition:delete}
\end{image}

В выводе \texttt{sstabledump} для данного раздела (тут также необходимо
выполнить nodetool flush и получить выгрузку с последнего созданного
файла "<<...>-big-Data.db">). Данные о времени удаления (deletion\_info)
написаны на уровне раздела и не привязаны к конкретным строкам или
ячейкам.

\begin{image}
	\includegrph{Screenshot from 2024-03-18 13-47-29}
	\caption{Выгрузка в формате json}
	\label{fig:sstabledump:partition:delete}
\end{image}

\section{Tombstone для строк}

Tombstone для строк генерируется при удалении для каждой строки
отдельно. Созданная ранее таблица rank\_by\_year\_and\_name имеет
составной первичный ключ, который включает в себя как ключ раздела, так и
ключ кластеризации. В данном случае при удалении условие WHERE
затрагивает оба ключа.

\begin{image}
	\includegrph{Screenshot from 2024-03-18 13-48-37}
	\caption{Удаление строки}
	\label{fig:cqlsh:line:delete}
\end{image}

В выводе \texttt{sstabledump} для этого раздела,
tombstone маркер deletion\_info находится на уровне строки,
так как затрагивает уже не только ключ раздела, но еще и ключ кластеризации.
При этом раздел и ячейки не содержат этого маркера.

\begin{image}
	\includegrph{Screenshot from 2024-03-18 13-55-29}
	\caption{Выгрузка в формате json}
	\label{fig:sstabledump:line:delete}
\end{image}

\section{Tombstone для диапазонов}

Tombstones для диапазона возникают, когда записи в разделе удаляются с
помощью поиска по диапазону. Для начала необходимо удалить таблицу
rank\_by\_year\_and\_name и создать заново, перезаполнив ее (См. пункт
"<Создание keyspace и таблиц">). Обратите внимание, что создастся новая
директория, так как таблица пересоздана.\par
Затем выполнить следующую команду, в которой указано также
неравенство (rank > 1). Выполните команду и посмотрите, что выдаст
\texttt{sstabledump}.

\begin{image}
	\includegrph{Screenshot from 2024-03-18 14-03-52}
	\caption{Удаление строки}
	\label{fig:cqlsh:range:delete}
\end{image}

В данном случае при выводе результата с \texttt{sstabledump} можно увидеть 2
маркера range\_tombstone\_bound, которые отмечают область удаленных записей.

\begin{image}
	\includegrph{Screenshot from 2024-03-18 14-04-04}
	\caption{Выгрузка в формате json}
	\label{fig:sstabledump:range:delete}
\end{image}

\section{ComplexColumn tombstones}

ComplexColumn Tombstones возникают при вставке или обновлении
столбца, хранящего коллекции (set, list, map). Ранее мы создали таблицу
cyclist\_care\_teams. Выполните следующую команду в cqlsh, чтобы
вставить запись в таблицу.

\begin{image}
	\includegrph{Screenshot from 2024-03-18 13-55-29}
	\caption{Удаление строки}
	\label{fig:cqlsh:complexcolumn}
\end{image}

В выходных данных \texttt{sstabledump} для раздела таблицы
cyclist\_career\_teams маркер deletion\_info указан на уровне ячейки
только для столбцов типа коллекции, хоть в разделе не происходит удаления.

\begin{image}
	\includegrph{Screenshot from 2024-03-18 13-57-48}
	\caption{Выгрузка в формате json}
	\label{fig:sstabledump:complexcolumn}
\end{image}

\section{Tombstone для ячейки}

Tombstone для ячеек создаются при удалении значения из ячейки или при
вставке в ячейку null, как показано в следующем примере.\par
В данном случае deletion\_info будет привязано к конкретной ячейке.

\begin{image}
	\includegrph{cell}
	\caption{Удаление ячейки и выгрузка в формате json}
	\label{fig:cqlsh-sstabledump:complexcolumn}
\end{image}

\section{Установка времени жизни записи (TTL)}

TTL можно использовать для вставки данных в таблицу на определенный
срок, указываемый в секундах. Чтобы задать TTL для записи, используются
ключевые слова USING TTL. Чтобы определить время жизни записи
необходима функция TTL.\par
Вставьте данные в таблицу cycling.calendar и используйте USING
TTL, чтобы установить период истечения жизни записи на 86400 секунд.\par
Выполните команду SELECT, чтобы определить, сколько запись будет
хранится.\par
Значение времени жизни также может быть обновлено с помощью
ключевых слов USING TTL в команде UPDATE.\par
Выполните SELECT еще раз, чтобы удостовериться, что TTL изменился.
Также можно установить TTL для одной ячейки.

\begin{image}
	\includegrph{ttl}
	\caption{Установка времени жизни записи ttl}
	\label{fig:ttl}
\end{image}

\section{TTL tombstones}

TTL tombstones создаются, когда TTL записи истек. Cassandra создает не
такой tombstone для TTL, какой создается при удалении с помощью оператора
DELETE. Также даже если раздел содержит только одну строку (без ключа
кластеризации), отметка TTL все равно делается на уровне строки.
В первом случае при использовании sstabledump CQL помечает всю
строку (partition key: 200) маркером "expired" : true в разделе liveness\_info.

\begin{image}
	\includegrph{tombstones}
	\caption{Выгрузка в формате json}
	\label{fig:tombstones:cycling}
\end{image}

Во втором случае помечается отдельная ячейка.

\begin{image}
	\includegrph{tombstones1}
	\caption{Выгрузка в формате json}
	\label{fig:tombstone:rank-by-year-and-name}
\end{image}

Вам необходимо применить функцию USING TTL для строки и для
ячейки, а после сделать выгрузку с помощью sstabledump после того, как срок
жизни строки/ячейки закончится.

\clearpage

\section*{\LARGE Вывод}
\addcontentsline{toc}{section}{Вывод}
В данной работе мы ознакомились со специальноыми значениями,
известными как tombstone, которое Cassandra создает чтобы указать,
что данные были удалены.\par
Эти значения предотвращают возврат удаленных данных во время чтения
и в конечном итоге позволяют удалить данные посредством сжатия.\par
Tombstones являются частью реализации модели хранения данных (во
время записи существующие данные не читаются).\par
Также познакомились со столбцами и таблицами поддерживающими
необязательный срок действия, называемый TTL (time-to-live).

