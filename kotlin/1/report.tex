\graphicspath{{./1/img/}} % path to graphics

\section*{\LARGE Цель практической работы}
\addcontentsline{toc}{section}{Цель практической работы}

\textbf{Цель работы} --- познакомится с реализацией ООП в языке
программирования Kotlin.

\textbf{Задачи}
\begin{itemize}
	\item Кофемашина умеет готовить четыре вида кофе: эспрессо, американо,
		капучино и латте.
	\item Требуется создать класс CoffeeMachine с публичным методом start() и
		полями water, milk и beans целочисленного типа.
	\item При вызове метод start() должен выводить сообщение "Кофемашина
		готова к работе", а после в бесконечном цикле выводить сообщение
		"Введите команду" и обрабатывать ввод.
		Рассмотрим каждую команду по-отдельности:
	\item \textit{"выключить"} --- по этой команде мы должны вывести сообщение
		"До свидания!" и выйти из функции.
	\item \textit{"наполнить"} --- увеличиваем значение переменных
		до максимальных (water до 2000, milk до 1000 и beans до 500)
		и выводим сообщение "Ингридиенты пополнены".
	\item \textit{"статус"} --- выводим сообщение "Ингридиентов осталось:"
		, а после - список оставшихся ингредиентов в следующем формате:
		"\$water мл воды\\n\$milk мл молока\\n\$beans гр кофе".
	\item \textit{"кофе"} --- выводим "Введите название напитка или "назад" для
		возврата в главное меню" и обрабатываем введённую команду.
\end{itemize}

Командой может быть как название кофе, так и фраза "назад", после
которой мы должны вернуться к предыдущему состоянию. Если введено
неизвестное название кофе, вывести фразу "Рецепт не найден!". Если введено
верное название кофе, то мы должны уменьшить запасы ингредиентов
согласно рецепту и вывести сообщение "\$coffeeName готов".

Если у нас недостаточно ингредиентов, вывести сообщение
"Недостаточно воды!" или "Недостаточно молока!" или "Недостаточно
кофе!" и вернуться к предыдущему состоянию.

Рецепты кофе (необходимо оформить в виде enum или data-класса):
\begin{itemize}
	\item Эспрессо --- 60 мл воды и 10 гр кофейных зёрен.
	\item Американо --- 120 мл воды и 10 гр кофейных зёрен.
	\item Капучино --- 120 мл воды, 20 гр кофейных зёрен и 60 мл молока.
	\item Латте --- 240 мл воды, 20 гр кофейных зёрен и 120 мл молока.
\end{itemize}

\clearpage

\section*{\LARGE Выполнение практической работы}
\addcontentsline{toc}{section}{Выполнение практической работы}

Для реализации функционала кофемашины был создан класс \texttt{CoffeeMachine},
который содержит поля \texttt{water}, \texttt{milk} и \texttt{beans}
типа \texttt{int} для хранения количества воды,
молока и кофейных зерен соответственно.
Класс также содержит публичный метод \texttt{start()},
который запускает основной цикл работы кофемашины.\par
При вызове метода \texttt{start()} выводится сообщение о готовности кофемашины
к работе, после чего начинается бесконечный цикл
ожидания пользовательского ввода.
Каждая команда обрабатывается в соответствии с требованиями задачи.\par
Код реализованной программы проиллюстрирован в листнинге ниже.

\lstinputlisting[language=Java]{./1/code/Main.kt}

Результат работы программы проиллюстрирован на рисунке\,\ref{fig:run}.

\begin{image}
	\includegrph{run}
	\caption{Работа программы}
	\label{fig:run}
\end{image}

\clearpage

\section*{\LARGE Вывод}
\addcontentsline{toc}{section}{Вывод}
В результате практической работы была разработана программная эмуляция
работы кофемашины на языке Kotlin.
В ходе работы была изучена работа с классами в рамках ООП.

Были рассмотрены три типа классов:
\begin{itemize}
	\item \textbf{Классы, описывающие какую-то логику}. В таких классах
		обычно содержатся поля, которые хранят какое-то состояние объекта, и
		методы, позволяющие выполнять какие-то действия с объектом.
	\item \textbf{Классы с данными, на английском Data Classes}. Такие
		классы обычно хранят в себе только набор каких-то полей, отражающих
		состояние объекта. В них зачастую нет методов, исполняющих какую-то
		логику. Они могут иметь в себе вспомогательные методы, помогающие
		отразить состояние объекта.
	\item \textbf{Утилитарные классы}. В Java такие классы часто имеют
		только статические методы и переменные (константы). Хорошие
		примеры таких классов: Arrays, Collections или Math, которые хранят в
		себе набор полезных функций для работы.
\end{itemize}

