\section*{\LARGE Цель практической работы}
\addcontentsline{toc}{section}{Цель практической работы}

\textbf{Задание 1}

Если в приложении для курсовой работы не предполагается подключение API,
необходимо подготовить «приложение-заглушку» для демонстрации работы
API.

\textbf{Задание 2}

Подключить к странице поиска любую открытую API. Ссылку на список
открытых API можно найти в СДО. Реализовать корректный поиск согласно
теме проекта. Отображение результатов поиска должно быть реализовано с
помощью RecycleView.

\textbf{Задание 3}

Реализовать плейсхолдеры:

\begin{enumerate}
	\item Если нет результатов поиска.
	\item Если поисковый запрос завершился неудачно, при этом на данном
		плейсхолдере должна быть размещена кнопка «Обновить».
\end{enumerate}

\textbf{Задание 4}

Если на плейсхолдере присутствует кнопка «Обновить», то необходимо
реализовать повторное отправление последнего запроса при нажатии на нее.

\clearpage

\section*{\LARGE Выполнение практической работы}
\addcontentsline{toc}{section}{Выполнение практической работы}

\section{Retrofit}

Retrofit — это библиотека для упрощённого клиент-серверного
взаимодействия через REST API. Её особенность --- простота в использовании
HTTP-запросов на основе JSON или других протоколов.\par
Если мобильное приложение создаётся как совершенно новый продукт с
собственной инфраструктурой, то за создание серверной части отвечает
backend-команда. В таком случае API уникально для конкретного приложения
и вся информация о нём хранится у backend-разработчиков. Часто возникает
необходимость добавить в приложение дополнительные трудозатратные
возможности, которые также требуют участия сервера. В таких случаях
наиболее выгодное и правильное решение --- использовать API готовых
продуктов с нужными функциями.\par

Класс сервис для реализации запроса к API:
\begin{lstlisting}[language=Java]
interface SubjectService {
    @GET("groups/certain")
    fun getGroupData(@Query("name") groupName: String?): Call<List<GroupResponse?>>?
}
\end{lstlisting}

Класс клиент для создания объкта Retrofit:
\begin{lstlisting}[language=Java]
object RetrofitClient {
    private const val BASE_URL = "https://mirea.xyz/api/v1.3/"

    private val retrofit: Retrofit by lazy {
        Retrofit.Builder()
            .baseUrl(BASE_URL)
            .addConverterFactory(GsonConverterFactory.create())
            .build()
    }

    fun getRetrofitInstance(): Retrofit {
        return retrofit
    }
}
\end{lstlisting}

Класс код отправки запроса и получение данных из API:
\begin{lstlisting}[language=Java]
    private val apiService: SubjectService = RetrofitClient.getRetrofitInstance().create(SubjectService::class.java)
    private val groupLiveData: MutableLiveData<List<GroupResponse?>> = MutableLiveData()
    private val searchText: MutableLiveData<String> = MutableLiveData()
    private val errorLiveData: MutableLiveData<String> = MutableLiveData()

    fun getError(): LiveData<String> {
        return errorLiveData
    }
    fun getGroupData(groupName: String?): LiveData<List<GroupResponse?>> {
        apiService.getGroupData(groupName)?.enqueue(object : Callback<List<GroupResponse?>> {
            override fun onResponse(
                call: Call<List<GroupResponse?>>,
                response: Response<List<GroupResponse?>>
            ) {
                if (response.isSuccessful) {
                    groupLiveData.postValue(response.body())
                } else {
                    Log.d("HomeViewModel", "Response ERROR: ${response.code()} - ${response.message()}")
                    errorLiveData.postValue(response.message())
                }
            }
            override fun onFailure(call: Call<List<GroupResponse?>>, t: Throwable) {
                Log.d("HomeViewModel", "onFailure: " + t.message)
                errorLiveData.postValue(t.message)
            }
        })
        return groupLiveData
    }
\end{lstlisting}

\section{Плейсхолдеры}

Для реализации плейсхолдеров в приложении были созданы соответствующие макеты
XML. Плейсхолдер для отсутствия результатов поиска содержит текстовое поле
с сообщением о том, что результаты не найдены.
Плейсхолдер для неудачного поискового запроса включает в
себя текстовое поле с сообщением об ошибке и кнопку "Обновить".
В активности были добавлены обработчики событий для отображения
соответствующих плейсхолдеров в зависимости от результатов поиска
и статуса запроса.

Код плейсхолдеров:
\begin{lstlisting}[language=XML]
<TextView
	android:id="@+id/textViewError"
	android:layout_width="wrap_content"
	android:layout_height="wrap_content"
	android:text="Result is empty"
	android:textSize="24sp"
	android:textStyle="bold"
	android:visibility="invisible"
	app:layout_constraintBottom_toTopOf="@+id/buttonAddQueue"
	app:layout_constraintEnd_toEndOf="parent"
	app:layout_constraintStart_toStartOf="parent"
	app:layout_constraintTop_toBottomOf="@+id/buttonClear"/>

<RelativeLayout
	android:id="@+id/layoutError"
	android:layout_width="wrap_content"
	android:layout_height="wrap_content"
	android:layout_centerHorizontal="true"
	android:visibility="invisible"
	app:layout_constraintBottom_toTopOf="@+id/buttonAddQueue"
	app:layout_constraintEnd_toEndOf="parent"
	app:layout_constraintStart_toStartOf="parent"
	app:layout_constraintTop_toBottomOf="@+id/buttonClear">

	<TextView
		android:id="@+id/textViewError0"
		android:layout_width="wrap_content"
		android:layout_height="wrap_content"
		android:text="Error" />

	<Button
		android:id="@+id/buttonRetry"
		android:layout_width="wrap_content"
		android:layout_height="wrap_content"
		android:backgroundTint="@color/blue"
		android:layout_below="@id/textViewError0"
		android:text="Retry" />
</RelativeLayout>
\end{lstlisting}

Код работы с API в активити:
\begin{lstlisting}[language=Java]
override fun onCreate(savedInstanceState: Bundle?) {
	super.onCreate(savedInstanceState)
	binding = ActivityHomeBinding.inflate(layoutInflater)
	setContentView(binding.root)
	groupViewModel = ViewModelProvider(this)[HomeViewModel::class.java]


	binding.editTextSearch.addTextChangedListener(object : TextWatcher {
		override fun beforeTextChanged(s: CharSequence?, start: Int, count: Int, after: Int) {
		}
		override fun onTextChanged(s: CharSequence?, start: Int, before: Int, count: Int) {
			if (s.isNullOrEmpty()) {
				binding.buttonClear.visibility = View.GONE
				queueAdapter.filter("")
			} else {
				binding.buttonClear.visibility = View.VISIBLE
				queueAdapter.filter(s.toString().trim())
			}
		}
		override fun afterTextChanged(s: Editable?) {
		}
	})

	binding.buttonClear.setOnClickListener {
		binding.editTextSearch.text.clear()
	}
	binding.buttonRetry.setOnClickListener {
		fetchQueueData()
	}

	binding.recyclerViewQueueList.layoutManager = LinearLayoutManager(this)
	queueAdapter = QueueListAdapter(queueList)
	binding.recyclerViewQueueList.adapter = queueAdapter
	queueAdapter.setOnClickListener(object : QueueListAdapter.OnClickListener {
		override fun onClick(position: Int) {
			navigateToQueue()
		}
	})

	fetchQueueData()

	binding.buttonAddQueue.setOnClickListener {
		navigateToAddQueue()
	}
	binding.imageViewProfile.setOnClickListener {
		navigateToProfile()
	}
}

private fun fetchThisWeekLessons(groupResponseList: List<GroupResponse?>): List<ScheduleItem> {
	val thisWeekLessons = mutableListOf<ScheduleItem>()
	val isEvenWeek = Calendar.getInstance()
		.get(Calendar.WEEK_OF_YEAR) \% 2 == 0

	Log.d("HomeActivity", "isEvenWeek " + isEvenWeek)

	for (groupResponse in groupResponseList) {
		groupResponse?.let {
			for (daySchedule in it.schedule) {
				val lessons = if (isEvenWeek) daySchedule.even 
					else daySchedule.odd
				for (lessonList in lessons) {
					for (lesson in lessonList) {
						if (lesson.weeks.isNullOrEmpty()
								|| lesson.weeks.contains("1")) {
							thisWeekLessons.add(lesson)
						}
					}
				}
			}
		}
	}

	return thisWeekLessons
}

private fun fetchQueueData() {
	Log.d("HomeActivity", "Response")
	groupViewModel.getGroupData("ИКБ-06-21").observe(this, Observer { groupResponse ->
		groupResponse?.let {
			queueList.clear()
			Log.d("HomeActivity", "START")
			val thisWeekLessons = fetchThisWeekLessons(groupResponse)
			for (lesson in thisWeekLessons) {
				if (!queueList.contains(lesson.name)) {
					Log.d("HomeActivity", lesson.name)
					queueList.add(lesson.name)
				}
			}
			Log.d("HomeActivity", queueList.size.toString())

			queueList.add("example item")

			queueAdapter.notifyDataSetChanged()
			queueAdapter.filter("")
			if (queueList.isEmpty())
				binding.textViewError.visibility = View.VISIBLE
			else
				binding.textViewError.visibility = View.GONE
		}
	})
	groupViewModel.getError().observe(this, Observer { error ->
		if (error.isNullOrEmpty()) {
			Log.d("HomeActivity", "groupResponse ERROR")
			binding.recyclerViewQueueList.visibility = View.GONE
			binding.layoutError.visibility = View.VISIBLE
		}
	})
}
\end{lstlisting}

\clearpage

\section*{\LARGE Вывод}
\addcontentsline{toc}{section}{Вывод}

В результате выполнения практической работы было реализовано приложение
с использованием открытого API для поиска и отображения данных. 

В ходе работы были выполнены следующие задания:

\begin{enumerate}
    \item Разработана "приложение-заглушка" для демонстрации работы
		приложения в случае отсутствия подключения к реальному API.
    \item Подключено открытое API к странице поиска,
		обеспечив корректный поиск данных, соответствующих теме проекта,
		и их отображение с использованием RecycleView.
    \item Реализованы плейсхолдеры для двух сценариев:
		отсутствия результатов поиска и неудачного завершения поискового
		запроса. Кнопка "Обновить" добавлена на плейсхолдер
		для повторной отправки последнего запроса.
    \item Добавлена функциональность повторной отправки запроса
		при нажатии на кнопку "Обновить" на плейсхолдере,
		обеспечивая удобство и интуитивную понятность для пользователей.
\end{enumerate}

Таким образом, выполнение практической работы позволило ознакомиться
с процессом работы с API, реализацией поиска и отображением данных,
а также с важной концепцией плейсхолдеров для обработки различных сценариев
взаимодействия с пользователем.

