\section*{\LARGE Цель практической работы}
\addcontentsline{toc}{section}{Цель практической работы}

\textbf{Задание 1}: Реализация переключателя темной темы

\textbf{Задание 2}: Необходимо реализовать хранение истории поиска с помощью
Shared Preferences.

Обязательные условия:
\begin{itemize}
	\item При нажатии на поисковую строку пользователь должен видеть историю
		поиска (при условии, что у пользователя непустая история и он не
		очищал историю поиска).
	\item В истории должно храниться не больше 10 элементов.
	\item Нажатие на элемент в списке результатов поиска добавляет его в
		историю.
	\item Элементы хранятся в истории в порядке добавления (новые
		отображаются в верхней части списка).
	\item Нажатие на кнопку «Очистить историю» полностью очищает записи
		истории поиска.
\end{itemize}

\textbf{Задание 3}: Реализовать автоматическое отправление поискового запроса,
если пользователь не выполняет никаких действий в течении 2 секунд

\textbf{Задание 4}: Реализовать отображение ProgressBar при выполнении
поискового запроса.

\clearpage

\section*{\LARGE Выполнение практической работы}
\addcontentsline{toc}{section}{Выполнение практической работы}

\section{Переключатель темной темы}

Реализуйте переключатель с помощью элемента SwitchMaterial.
В XMLфайле это выглядит так:

\begin{lstlisting}[language=XML]
<com.google.android.material.switchmaterial.SwitchMaterial
	android:id="@+id/themeSwitcher"
	android:layout_width="match_parent"
	android:layout_height="wrap_content"
	android:layout_marginTop="@dimen/margin_large"
	android:layout_marginHorizontal="60dp"
	android:text="Темная тема"
	app:layout_constraintEnd_toEndOf="parent"
	app:layout_constraintStart_toStartOf="parent"
	app:layout_constraintTop_toBottomOf="@+id/constraintLayout2" />
\end{lstlisting}

Для работы с переключателем необходим метод setOnCheckedChangeListener,
применяемый к переключателю в коде:

\begin{lstlisting}[language=Java]
binding.themeSwitcher.setOnCheckedChangeListener { switcher, checked ->
	(applicationContext as App).switchTheme(checked)
}
\end{lstlisting}

Создайте класс App и унаследуйте от класса Application, а затем
реализуйте в нём метод onCreate:

\begin{lstlisting}[language=Java]
class App : Application() {
    private var darkTheme = false

    override fun onCreate() {
        super.onCreate()
        val sharedPrefs = getSharedPreferences(
			EXAMPLE_PREFERENCES, MODE_PRIVATE
			)
        darkTheme = sharedPrefs.getBoolean(
            "dark_mode",
            false
        )
        if (darkTheme) {
            setTheme(R.style.Theme_Black)
        } else {
            setTheme(R.style.Theme)
        }
    }

    fun switchTheme(darkThemeEnabled: Boolean) {
        darkTheme = darkThemeEnabled
        AppCompatDelegate.setDefaultNightMode(
            if (darkThemeEnabled) {
                AppCompatDelegate.MODE_NIGHT_YES
            } else {
                AppCompatDelegate.MODE_NIGHT_NO
            }
        )
    }

    companion object {
        const val EXAMPLE_PREFERENCES = "example_preferences"
    }
}
\end{lstlisting}

\section{Хранение истории поиска с помощью Shared Preferences}

Создадим разметку для поля ввода и RecycleView для вывода истории.

\begin{lstlisting}[language=XML]
<androidx.constraintlayout.widget.ConstraintLayout
	android:id="@+id/searchLayout"
	android:layout_width="match_parent"
	android:layout_height="wrap_content"
	android:layout_marginTop="@dimen/margin_small"
	android:visibility="visible"
	android:background="@drawable/search_bg"
	app:layout_constraintEnd_toEndOf="parent"
	app:layout_constraintStart_toStartOf="parent"
	app:layout_constraintTop_toBottomOf="@+id/textViewQueueList">

	<EditText
		android:id="@+id/editTextSearch"
		android:layout_width="0dp"
		android:layout_height="@dimen/edit_text_height"
		android:hint="@string/search_hint"
		android:inputType="text"
		android:paddingStart="@dimen/padding_medium"
		android:paddingEnd="@dimen/padding_medium"
		android:visibility="gone"
		android:textColorHint="@color/black"
		app:layout_constraintEnd_toEndOf="parent"
		app:layout_constraintStart_toStartOf="parent"
		app:layout_constraintTop_toTopOf="parent" />

	<androidx.constraintlayout.widget.ConstraintLayout
		android:id="@+id/history_layout"
		android:layout_width="match_parent"
		android:layout_height="0dp"
		android:background="@color/grey"
		android:visibility="gone"
		app:layout_constraintEnd_toEndOf="@+id/editTextSearch"
		app:layout_constraintStart_toStartOf="@+id/editTextSearch"
		app:layout_constraintTop_toBottomOf="@+id/editTextSearch">

		<androidx.recyclerview.widget.RecyclerView
			android:id="@+id/recyclerViewSearchHistory"
			android:layout_width="match_parent"
			android:layout_height="50dp"
			app:layout_constraintEnd_toEndOf="parent"
			app:layout_constraintStart_toStartOf="parent"
			app:layout_constraintTop_toTopOf="parent" />

		<Button
			android:id="@+id/buttonClearHistory"
			android:layout_width="match_parent"
			android:layout_height="35dp"
			android:backgroundTint="@color/blue"
			android:text="Clear"
			app:layout_constraintEnd_toEndOf=
				"@+id/recyclerViewSearchHistory"
			app:layout_constraintTop_toBottomOf=
				"@+id/recyclerViewSearchHistory" />
	</androidx.constraintlayout.widget.ConstraintLayout>
</androidx.constraintlayout.widget.ConstraintLayout>
\end{lstlisting}

Далее создадим методы для добавления, получения и очестки истории поиска:

\begin{lstlisting}[language=Java]
fun addSearchQuery(query: String?) {
	var searchHistory: MutableList<String?> = getSearchHistory()
	searchHistory.remove(query)
	searchHistory.add(0, query)
	if (searchHistory.size > MAX_HISTORY_SIZE) {
		searchHistory = searchHistory.subList(0, MAX_HISTORY_SIZE)
	}
	val sharedPrefs = getSharedPreferences(
		HISTORY_PREFERENCES, Context.MODE_PRIVATE
		)
	val editor = sharedPrefs?.edit()
	editor?.putStringSet(SEARCH_HISTORY_KEY, HashSet(searchHistory))
	editor?.apply()
}
fun getSearchHistory(): MutableList<String?> {
	val preferences = getSharedPreferences(
		HISTORY_PREFERENCES, Context.MODE_PRIVATE)
	val historySet = preferences.getStringSet(
		SEARCH_HISTORY_KEY, HashSet<String>())
	return ArrayList(historySet)
}
fun clearSearchHistory() {
	val preferences = getSharedPreferences(
		HISTORY_PREFERENCES, Context.MODE_PRIVATE)
	val editor = preferences.edit()
	editor.remove(SEARCH_HISTORY_KEY)
	editor.apply()
}
\end{lstlisting}

И добавим обработку в onClick.

\begin{lstlisting}[language=Java]
binding.editTextSearch.addTextChangedListener(object : TextWatcher {
	override fun beforeTextChanged(
		s: CharSequence?, start: Int, count: Int, after: Int) {
	}
	override fun onTextChanged(
		s: CharSequence?, start: Int, before: Int, count: Int) {
		val query = s.toString().trim()
		if (query.isNullOrEmpty()) {
			binding.buttonClear.visibility = View.GONE
			queueAdapter.clear()
		} else {
			binding.buttonClear.visibility = View.VISIBLE
			queueAdapter.filter(query)
		}
		queueAdapter.filter(s.toString())
	}
	override fun afterTextChanged(s: Editable?) {
	}
})
binding.editTextSearch.setOnFocusChangeListener { view, hasFocus ->
	if (hasFocus) {
		if (searchHistory.isNotEmpty()) {
			binding.historyLayout.visibility = View.VISIBLE
		}
	} else {
		binding.historyLayout.visibility = View.GONE
	}
}
binding.buttonClear.setOnClickListener {
	binding.editTextSearch.text.clear()
}
\end{lstlisting}

\section{Автоматическое отправление поискового запроса}

У каждого потока может быть только один Looper и, как следствие, только
одна очередь MessageQueue. Так как Handler --- посредник между разными
потоками и скрывает от программиста логику работы с очередью событий,
каждый Handler должен знать, к какому именно потоку (Looper) он привязан.
От этого зависит, в какую именно очередь событий Handler будет вставлять
новые сообщения и какую именно очередь он будет слушать.\par
Debounce --- это особая логика (функция или сочетание функций), которая
откладывает выполнение какого-либо действия на определённое время в
зависимости от требуемых условий. Использование времени системы --- лишь
один из вариантов реализации debounce.\par

\begin{lstlisting}[language=Java]
private var searchHandler = Handler(Looper.getMainLooper())
private var searchRunnable = Runnable { searchRequest() }

private fun searchRequest() {
	val query = binding.editTextRequest.text.toString().trim()
	fetchQueueData(query)
}
private fun searchDebounce() {
	searchHandler.removeCallbacks(searchRunnable)
	searchHandler.postDelayed(searchRunnable, SEARCH_DEBOUNCE_DELAY)
}

companion object {
	private const val SEARCH_DEBOUNCE_DELAY = 3000L
}
\end{lstlisting}

У нас есть Runnable, который инициализирует отправку поискового запроса.
В момент вызова функции searchDebounce() мы удаляем последнюю
запланированную отправку запроса и тут же, используя метод postDelayed(),
планируем запуск этого же Runnable через две секунды.\par
Теперь добавьте вызов функции в TextWatcher.

\begin{lstlisting}[language=Java]
binding.editTextRequest.addTextChangedListener(object : TextWatcher {
	override fun beforeTextChanged(
		s: CharSequence?, start: Int, count: Int, after: Int
		) {
	}
	override fun onTextChanged(
		s: CharSequence?, start: Int, before: Int, count: Int
		) {
		searchDebounce()
	}
	override fun afterTextChanged(s: Editable?) {
	}
})
\end{lstlisting}

\section{ProgressBar}
ProgressBar --- специальный View-элемент индикации прогресса. Его можно
встретить почти в каждом приложении.
Вот так можно добавить ProgressBar в файл вёрстки:

\begin{lstlisting}[language=XML]
<ProgressBar
        android:id="@+id/progressBar"
        android:layout_width="44dp"
        android:layout_height="44dp"
        android:layout_gravity="center"
        android:indeterminateTint="@color/black"
        android:visibility="gone"
        app:layout_constraintBottom_toTopOf="@+id/buttonAddQueue"
        app:layout_constraintEnd_toEndOf="parent"
        app:layout_constraintStart_toStartOf="parent"
        app:layout_constraintTop_toBottomOf="@+id/searchLayout"/>
\end{lstlisting}

Теперь перед отправкой запроса нужно скрыть с экрана все прочие элементы
и сделать ProgressBar видимым. А после завершения запроса (успешного или
нет) спрятать его и показать либо список с результатами, либо
соответствующее сообщение.

\begin{lstlisting}[language=Java]
progressBar.visibility = View.VISIBLE
\end{lstlisting}

Элемент ProgressBar --- верный помощник в любом интерфейсе. Вместо
бесконечной анимации с его помощью можно отображать реальный прогресс
и заполнять окружность по мере выполнения какой-то задачи. Сам элемент
можно отобразить в виде горизонтальной полосы прогресса.

\clearpage

\section*{\LARGE Вывод}
\addcontentsline{toc}{section}{Вывод}

В результате выполнения практической работы были
изучены и реализованы таке возмжности PDK Android как:

\begin{itemize}
	\item переключатель темной темы;
	\item хранение истории поиска с помощью Shared Preferences;
	\item автоматическое отправление поискового запроса,
		если пользователь не выполняет никаких действий в течении 2 секунд;
	\item отображение ProgressBar при выполнении поискового запроса.
\end{itemize}

