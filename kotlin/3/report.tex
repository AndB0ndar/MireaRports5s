\section*{\LARGE Цель практической работы}
\addcontentsline{toc}{section}{Цель практической работы}

\textbf{Задание 1}

Создать или найти в открытых источниках макет мобильного приложения в
Figma на выбранную Вами тему.

\textbf{Задание 2}

Сверстать экраны, согласно макету. Реализовать какое-либо взаимодействие с
пользовательским интерфейсом не нужно.
Реализовать переходы между Activity.

Обязательные условия:

\begin{itemize}
	\item Полученные экраны идентичны макету в Figma.
	\item Свёрстанный экран адаптируется по размеры различных устройств.
	\item Все строки и цвета вынесены в ресурсы.
	\item Размеры, которые используются несколько раз,
		также вынесены в ресурсы.
\end{itemize}

\clearpage

\section*{\LARGE Выполнение практической работы}
\addcontentsline{toc}{section}{Выполнение практической работы}

\section{Сверстаный экраны, согласно макету}

Пользовательский интерфейс, или UI --- от английского User Interface.
Разработкой внешнего вида приложения в большинстве компаний
занимаются отдельные люди --- дизайнеры интерфейсов. Результат их работы
--- это изображения всех экранов приложения, а таких экранов у одного
приложения может быть много.\par
Дело разработчиков --- кодом отобразить на экранах этот интерфейс в
том виде, как его нарисовал дизайнер. Создание интерактивного
пользовательского интерфейса называется вёрсткой.\par
У сервиса Figma есть настольная версия и веб-приложение. Эти версии
практически не отличаются по интерфейсу.\par
На рисунке \ref{fig:figma} представлен пример отображения макета в Figma.

\begin{image}
	\includegrph{figma}
	\caption{Дизайн в Figma}
	\label{fig:figma}
\end{image}

\section{Сверстаный экраны, согласно макету}

В данной главе мы рассмотрим процесс верстки дизайна мобильного приложения
очереди, основываясь на предоставленных макетах из Figma.
Дизайн макетов включает в себя основные элементы пользовательского интерфейса,
структуру экранов и их взаимосвязь.
Наша задача --- преобразовать эти макеты в функциональный интерфейс,
который пользователи смогут использовать на своих мобильных устройствах.

Сверстанные экраны показаны на рисунках \ref{fig:login}\,-\,\ref{fig:queue}.

\begin{image}
	\includegrph{Screenshot from 2024-03-17 21-13-45}
	\caption{Экран авторизации}
	\label{fig:login}
\end{image}

\begin{image}
	\includegrph{Screenshot from 2024-03-17 21-13-51}
	\caption{Экран регистрации}
	\label{fig:signup}
\end{image}

\begin{image}
	\includegrph{Screenshot from 2024-03-17 21-14-41}
	\caption{Экран списка очередей}
	\label{fig:queue:list}
\end{image}

\begin{image}
	\includegrph{Screenshot from 2024-03-17 21-14-47}
	\caption{Экран профеля}
	\label{fig:profile}
\end{image}

\begin{image}
	\includegrph{Screenshot from 2024-03-17 21-14-51}
	\caption{Экран добавления очереди}
	\label{fig:queue:add}
\end{image}

\begin{image}
	\includegrph{Screenshot from 2024-03-17 21-14-57}
	\caption{Экран очереди}
	\label{fig:queue}
\end{image}

\clearpage

\section*{\LARGE Вывод}
\addcontentsline{toc}{section}{Вывод}


В процессе верстки дизайна мобильного приложения очереди
на основе макетов из Figma мы прошли через несколько важных этапов.\par
Создание архитектуры приложения было ключевым шагом,
где мы определили структуру приложения и решали вопросы,
связанные с организацией пользовательского интерфейса.\par
Верстка дизайна мобильного приложения включала в себя создание
пользовательских интерфейсов на основе макетов,
адаптацию элементов под различные размеры экранов и устройств,
а также обеспечение соответствия дизайна и функциональности макетов.\par
В результате мы успешно создали интерфейс мобильного приложения очереди,
который соответствует предоставленным макетам из Figma.
Этот процесс позволил нам преобразовать концепцию дизайна в реальный,
функциональный интерфейс, готовый к дальнейшей разработке и тестированию.

