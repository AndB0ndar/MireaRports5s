\section*{\LARGE Цель практической работы}
\addcontentsline{toc}{section}{Цель практической работы}

\textbf{Задание}:
если в приложении для курсовой работы не предполагается
создание собственного сервера, необходимо повторить действия,
описанные в практической работе.

\clearpage

\section*{\LARGE Выполнение практической работы}
\addcontentsline{toc}{section}{Выполнение практической работы}

\section{Переключатель темной темы}

Ktor (произносится как Кэй-тор) - это созданный с нуля фреймворк,
основу которого составляет Kotlin и корутины. С помощью Ktor можно
создавать клиентские и серверные приложения, которые будут работать на
разных платформах. Фреймворк отлично подходит для приложений,
требующих связь по HTTP и/или сокетам. Это могут быть, например, HTTPбэкенды и RESTful-системы, независимо от того, построены ли они по
микросервисному принципу или нет.\par
Ktor родился под влиянием других фреймворков, таких как Wasabi и
Kara, с целью максимально использовать некоторые возможности языка Kotlin,
такие как DSL (внутренний предметно-ориентированный язык) и корутины.
Если необходимо создать системы, которые будут связаны друг с другом, Ktor
отлично подходит для этого и является производительным, асинхронным,
многоплатформенным решением.\par
В настоящее время клиентская часть Ktor работает на всех платформах,
на которые ориентирован Kotlin, то есть JVM, JavaScript и Native. Однако,
серверная часть Ktor пока ограничена только JVM.\par

\section{Создание проекта Ktor}

Для создания нового проекта Ktor необходимо использовать
вебгенератор проектов Ktor.\par
Укажите название Вашего проекта в соответствующем поле.
Нажмите на "<Настроить параметры проекта">
("Adjust project setting").\par
В открывшемся меню необходимо настроить:

\begin{itemize}
	\item систему сборки (Build system) --- Gradle Kotlin;
	\item название пакета проекта (Website);
	\item движок (Engine), используемый для запуска сервера --- CIO;
	\item как будет происходить настройка сервера
			(Configuration in) --- Code.
\end{itemize}

Нажмите "<Добавить плагины"> ("Add plugins"),
в полученном списке выберите Content Negotiation,
kotlinx.serialization, Routing, postgresql.\par
Нажмите на кнопку "Generate project",
после чего, должна начаться загрузка проекта.\par
Запустите проект в любой IDE.\par
Запустите Postman, и отправьте GET-запрос по адресу,
указанному в файле Application.kt (host и port).

\section{Настройка сериализации}

Добавьте зависимость для kotlinx.

\begin{verbatim}
implementation("org.jetbrains.kotlinx:kotlinx-coroutines-core-jvm:1.6.0-RC")
\end{verbatim}

Добавьте в класс Routing data class Test,
который содержит только одно поле text строкового формата,
а функцию configureRouting() необходимо изменить таким образом,
чтобы в качестве ответа на GET-запрос был получен объект
класса Test.\par
Проверьте работоспособность сервера с помощью Postman, в отчете
укажите, какой ответ был получен. 

\section{Регистрация и логин}

\subsection{Модели}

Создайте пакет models, а в нем класс Authorization.

\begin{image}
	\includegrph{Screenshot from 2024-05-17 09-55-22.png}
	\caption{Класс Authorization}
    \label{fig:loginremote}
\end{image}

В данных классах мы указываем, какие данные мы будем отправлять
при запросах и получать в ответ. 
Таким образом, при регистрации мы отправляем логин, пароль и почту,
а в ответ получим токен, то есть уникальный номер пользователя, а при входе
(логине) отправляем логин и пароль, в ответ, также, получаем токен.

\subsection{Сервис для работы с базой данных}

Напишем сервис для для работы с базой данный регистрации.

\begin{lstlisting}[language=Java]
class AuthorizationService(private val connection: Connection) {
    companion object {
        const val TABLE_NAME = "UserAuth"
        const val COLUMN_ID = "ID"
        private const val COLUMN_LOGIN = "LOGIN"
        private const val COLUMN_FIRSTNAME = "FIRSTNAME"
        private const val COLUMN_LASTNAME = "LASTNAME"
        private const val COLUMN_GROUP_ID = "GROUP_ID"
        private const val COLUMN_PASSWORD_HASH = "PASSWORD_HASH"
        private const val COLUMN_TOKEN = "TOKEN"
        private const val CREATE_TABLE_USERS =
            "CREATE TABLE IF NOT EXISTS \$TABLE_NAME (" +
                    "\$COLUMN_ID SERIAL PRIMARY KEY" +
                    ", \$COLUMN_LOGIN VARCHAR(255)" +
                    ", \$COLUMN_FIRSTNAME VARCHAR(255)" +
                    ", \$COLUMN_LASTNAME VARCHAR(255)" +
                    ", \$COLUMN_GROUP_ID INT" +
                    ", \$COLUMN_PASSWORD_HASH VARCHAR(255)" +
                    ", \$COLUMN_TOKEN VARCHAR(255)" +
                    ", FOREIGN KEY (\$COLUMN_GROUP_ID) " +
					"REFERENCES \${GroupService.TABLE_NAME}(\$COLUMN_ID)" +
                    ");"
        private const val INSERT_USER =
            "INSERT INTO \$TABLE_NAME (" +
                    "\$COLUMN_LOGIN, \$COLUMN_FIRSTNAME, " +
					"\$COLUMN_LASTNAME, \$COLUMN_GROUP_ID, " +
					"\$COLUMN_PASSWORD_HASH, \$COLUMN_TOKEN" +
                    ") VALUES (?, ?, ?, ?, ?, ?)"
        private const val SELECT_USER_BY_LOGIN =
            "SELECT * FROM \$TABLE_NAME WHERE \$COLUMN_LOGIN = ?"
        private const val SELECT_USER_BY_LOGIN_AND_PASSWORD =
            "SELECT * FROM \$TABLE_NAME " +
			"WHERE \$COLUMN_LOGIN = ? AND \$COLUMN_PASSWORD_HASH = ?"
        private const val SELECT_USER_BY_TOKEN =
            "SELECT * FROM \$TABLE_NAME WHERE \$COLUMN_TOKEN = ?"
    }
    init {
        val statement = connection.createStatement()
        statement.executeUpdate(CREATE_TABLE_USERS)
    }
    fun register(
		login: String
		, firstName: String, lastName: String
		, groupId: Int, password: String
		): String? {
        if (checkUserExistsByLogin(login)) {
            return null
        }
        val hashedPassword = hashPassword(password)
        val token = UUID.randomUUID().toString()
        val preparedStatement = connection.prepareStatement(INSERT_USER)
        preparedStatement.setString(1, login)
        preparedStatement.setString(2, firstName)
        preparedStatement.setString(3, lastName)
        preparedStatement.setInt(4, groupId)
        preparedStatement.setString(5, hashedPassword)
        preparedStatement.setString(6, token)
        preparedStatement.executeUpdate()
        return token
    }
    fun login(login: String, password: String): String? {
        val hashedPassword = hashPassword(password)
        val preparedStatement = connection.prepareStatement(
			SELECT_USER_BY_LOGIN_AND_PASSWORD
			)
        preparedStatement.setString(1, login)
        preparedStatement.setString(2, hashedPassword)
        val resultSet = preparedStatement.executeQuery()
        return if (resultSet.next()) {
            resultSet.getString(COLUMN_TOKEN)
        } else {
            null
        }
    }
    fun getUserByToken(token: String): User? {
        val preparedStatement = connection.prepareStatement(
			SELECT_USER_BY_TOKEN
			)
        preparedStatement.setString(1, token)
        val resultSet = preparedStatement.executeQuery()
        return if (resultSet.next()) {
            User(
                id = resultSet.getInt(COLUMN_ID),
                login = resultSet.getString(COLUMN_LOGIN),
                firstName = resultSet.getString(COLUMN_FIRSTNAME),
                lastName = resultSet.getString(COLUMN_LASTNAME),
                groupId = resultSet.getInt(COLUMN_GROUP_ID),
                token = resultSet.getString(COLUMN_TOKEN)
            )
        } else {
            null
        }
    }
    private fun checkUserExistsByLogin(login: String): Boolean {
        val preparedStatement = connection.prepareStatement(
			SELECT_USER_BY_LOGIN
			)
        preparedStatement.setString(1, login)
        val resultSet = preparedStatement.executeQuery()
        return resultSet.next()
    }
    private fun hashPassword(password: String): String {
        val bytes = password.toByteArray()
        val digest = MessageDigest.getInstance("SHA-256")
        val hashedBytes = digest.digest(bytes)
        return hashedBytes.joinToString("") { "%02x".format(it) }
    }
}
\end{lstlisting}

\subsection{Обработчики запросов}

В пакете handler создайте класс RegisterHandler.

\begin{image}
	\includegrph{Screenshot from 2024-05-17 09-49-47.png}
	\caption{Класс RegisterHandler}
    \label{fig:registerhandler}
\end{image}

Первое условие проверяет введенную почту в написанном ранее сервисе.
Второе условие проверяет был ли зарегистрирован пользователь ранее.
Далее формируется токен, и пара логин -- токен записывается в базу данный.\par
Добавьте в класс LoginHandler проверку введенного пароля.

\begin{image}
	\includegrph{Screenshot from 2024-05-17 09-49-18.png}
	\caption{Класс LoginHandler}
    \label{fig:loginhandler}
\end{image}

Функция firstOrNull возвращает первый элемент, соответствующий
заданному, или null, если элемент не найден.\par
Таким образом, если пользователь не зарегистрирован, то переменная
first будет равна null, и в ответ будет получено сообщение "User not found",
иначе, если введенный пароль совпадает с зарегистрированным, то в ответ
будет получен токен, в обратном случае будет получено сообщение "Invalid
password".

\subsection{Конфигурационный файл}

Далее напишем конфигурационный файл для работы приложения
и подключения к базе данный.

\begin{image}
	\includegrph{Screenshot from 2024-05-17 09-48-06.png}
	\caption{Конфигурационный файл}
    \label{fig:configfile}
\end{image}

\subsection{Подключение к базе данный}

Напишем клас для создание подключения к базе данный.
Для дальнейшей работы сервисов.

\begin{image}
	\includegrph{Screenshot from 2024-05-17 09-48-36.png}
	\caption{Подключение к базе данный}
    \label{fig:connection}
\end{image}

\subsection{Main функция сервера и модули}

Для запуска сервера необходима загрузить настройки из файла конфигурации
с помощью движка CIO.
Далее необходимо инициализировать подключение к базе данный
и вызвать обработчик запросов.

\begin{image}
	\includegrph{Screenshot from 2024-05-17 09-47-53.png}
	\caption{Main функция сервера и модули}
    \label{fig:main}
\end{image}

\subsection{Обработчик запросов}

Создадим обработчкик запросов.
Это будет отправной функцией для всех остальный обработчиков.

\begin{image}
	\includegrph{Screenshot from 2024-05-17 09-49-07.png}
	\caption{Обработчик запросов}
    \label{fig:routing}
\end{image}


\clearpage

\section*{\LARGE Вывод}
\addcontentsline{toc}{section}{Вывод}

В результате выполнения практической работы мы ознакомились с фреймворком Ktor
и написали свой сервер с API для нашего клиенского приложения.

