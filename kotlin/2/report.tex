\section*{\LARGE Цель практической работы}
\addcontentsline{toc}{section}{Цель практической работы}

\textbf{Задачи}
\begin{itemize}
	\item Сформулировать тему клиент-серверного мобильного приложения.
		При этом в приложении должна быть возможность реализовать
		страницу с поиском.
	\item Обосновать актуальность выбранной темы.
	\item Провести сравнительный анализ аналогов разрабатываемого
		мобильного приложения.
	\item Описать функциональные возможности мобильного приложения в
		формате пользовательских историй.
	\item Описать межэкранное взаимодействие в выбранном мобильном
		клиент-серверном приложении.
\end{itemize}

\clearpage

\section*{\LARGE Выполнение практической работы}
\addcontentsline{toc}{section}{Выполнение практической работы}

\section{Актуальность выбранной темы}

В современном мире, особенно в условиях быстрого развития информационных
технологий, эффективное управление временем и ресурсами играет ключевую роль.
Одним из актуальных аспектов этого управления является организация очередности
в различных сферах деятельности, включая образование.
Студенты, преподаватели и сотрудники учебных заведений часто сталкиваются
с необходимостью эффективной организации очередности, например,
при сдаче работ, консультациях или других административных процедурах.\par
Существующие методы управления очередностью,
такие как ручное назначение времени или фиксированные расписания,
часто не обеспечивают оптимального результата. Они могут быть неудобными,
недостаточно гибкими и могут приводить к неэффективному использованию времени
как для преподавателей, так и для студентов.
Разработка мобильного приложения "Очередь"
представляет собой инновационный подход к решению этой проблемы,
предоставляя удобный инструмент для организации
и управления очередностью через мобильные устройства.\par
Такое приложение будет способствовать оптимизации процесса
организации очередности, уменьшению времени ожидания
и повышению эффективности работы как для преподавателей,
так и для студентов. Приложение "Очередь" будет актуальным
и востребованным инструментом, отвечающим современным требованиям управления
временем и ресурсами в образовательных учреждениях.\par

\section{Анализ предметной области}

Сравнительный анализ аналогов разрабатываемого мобильного приложения "Очередь"
позволяет выявить и оценить существующие решения в данной области
и определить их преимущества и недостатки.
Ниже представлен анализ нескольких аналогичных приложений:

\begin{itemize}
	\item \textbf{Qminder:} Приложение Qminder предназначено
		для управления очередями в различных сферах,
		включая образование (рис.~\ref{fig:an:qminder}).
		Оно предлагает возможность виртуального ожидания
		в очереди через мобильное устройство,
		уведомления о статусе очереди и т.д.
		Однако некоторые пользователи отмечают сложность в настройке
		и использовании интерфейса.
	\item \textbf{WaitWhile:} WaitWhile также позволяет организовывать
		очереди и уведомлять клиентов о статусе (рис.~\ref{fig:an:waitwhile}).
		Пользователи отмечают простоту использования
		и настраиваемые опции уведомлений.
		Однако интеграция с учебными заведениями может быть ограничена.
\end{itemize}

\begin{image}
	\includegrph{qminder}
	\caption{Приложение Qminder}
	\label{fig:an:qminder}
\end{image}

\begin{image}
	\includegrph{waitwhile}
	\caption{Приложение WaitWhile}
	\label{fig:an:waitwhile}
\end{image}

В сравнении с вышеупомянутыми приложениями,
разрабатываемое приложение "Очередь" будет ориентировано
на специфические потребности учебных заведений,
обеспечивая простоту использования, гибкость настройки
и оптимизацию процесса управления очередностью как для преподавателей,
так и для студентов.

\section{Функциональные истории}

Пользовательские истории являются важным инструментом
при разработке программного обеспечения, в том числе и мобильных приложений.
Они помогают команде разработчиков и заказчику точно определить требования
к продукту и обеспечить его соответствие ожиданиям пользователей.

\begin{itemize}
	\item \textbf{Как преподаватель}, я хочу создавать новые очереди
		для приема студентов, чтобы организовать процесс консультаций.
	\item \textbf{Как студент}, я хочу просматривать доступные очереди
		и записываться в них, чтобы получить помощь от преподавателей.
	\item \textbf{Как владельца очереди}, я хочу иметь возможность управлять
		пользователями и их правами доступа, чтобы обеспечить безопасность
		и эффективность системы.
	\item \textbf{Как преподаватель}, я хочу получать уведомления
		о новых запросах на консультацию, чтобы своевременно отвечать на них.
	\item \textbf{Как студент}, я хочу видеть свое место в очереди
		и примерное время ожидания, чтобы планировать свое время.
	\item \textbf{Как владельц очереди}, я хочу иметь возможность настраивать
		параметры каждой очереди, такие как максимальное количество студентов,
		время работы и т.д.,
		чтобы учитывать особенности каждого учебного заведения.
	\item \textbf{Как преподаватель}, я хочу иметь возможность отмечать
		студентов, которые были обслужены,
		чтобы поддерживать актуальную информацию о состоянии очереди.
	\item \textbf{Как студент}, я хочу отменять свою запись в очередь,
		если у меня изменятся планы,
		чтобы освободить место для других студентов.
	\item \textbf{Как владельц очереди}, я хочу иметь доступ к отчетам
		и статистике о работе приложения,
		чтобы анализировать его эффективность и делать необходимые улучшения.
	\item \textbf{Как студент}, я хочу иметь возможность оставить обратную
		связь о качестве обслуживания,
		чтобы помочь улучшить процесс консультаций и организацию очередей.
\end{itemize}

Каждая из этих пользовательских историй определяет конкретные требования
к функциональности мобильного приложения "Очередь"
и помогает сформулировать его основные возможности.

\section{Межэкранное взаимодействие}

Давайте рассмотрим межэкранные взаимодействия
в мобильном клиент-серверном приложении "Очередь",
предназначенном для организации очередности в учебных заведениях.

\begin{itemize}
   \item Экран авторизации:
		\begin{itemize}
			\item Пользователь вводит свои учетные данные (логин и пароль).
			\item Приложение отправляет запрос на сервер
				для проверки подлинности.
			\item В случае успешной аутентификации пользователь
				перенаправляется на главный экран приложения.
		\end{itemize}
   \item Главный экран:
		\begin{itemize}
			\item Пользователь может видеть доступные очереди
				и выбирать нужную.
			\item При выборе очереди приложение отправляет запрос
				на сервер для получения дополнительной информации о ней.
			\item Пользователь может удалить или создать новую очередь.
			\item Пользователь в строке поиска может
				искать необходимую ему очередь.
		\end{itemize}
   \item Экран деталей очереди:
		\begin{itemize}
			\item Пользователь может увидеть информацию о выбранной очереди,
				такую как свое место, места других пользователей и описание.
			\item Пользователь может добавить себя в очередь,
				нажав соответствующую кнопку.
			\item Приложение отправляет запрос на сервер
				для записи пользователя в очередь и получения подтверждения.
		\end{itemize}
   \item Экран владельца очереди:
		\begin{itemize}
			\item При наличии прав владельца очереди пользователь
				может управлять параметрами очередей,
				просматривать статистику и изменять очередь.
			\item Изменения, внесенные владельцем очереди,
				сохраняются на сервере и отображаются другим пользователям.
		\end{itemize}
\end{itemize}

В каждом из этих сценариев межэкранные взаимодействия включают отправку
запросов на сервер для получения или обновления данных,
а также отображение результата операции на мобильном устройстве пользователя.
Это обеспечивает плавное и эффективное взаимодействие
между клиентским и серверным компонентами приложения.

\clearpage

\section*{\LARGE Вывод}
\addcontentsline{toc}{section}{Вывод}

Разработка мобильного клиент-серверного приложения "Очередь"
имеет высокий потенциал для улучшения организации процессов
в учебных заведениях.
Предложенное приложение обладает рядом функциональных возможностей,
которые могут значительно облегчить процесс управления очередностью
как для преподавателей, так и для студентов.\par
Межэкранные взаимодействия в приложении позволяют пользователям легко
и интуитивно взаимодействовать с системой, предлагая удобный интерфейс
для просмотра доступных очередей, записи в них,
а также управления параметрами и администрирования системы.\par
В результате анализа существующих аналогов и разработанных пользовательских
историй, можно заключить, что приложение "Очередь"
предлагает конкурентоспособное решение,
которое учитывает потребности и ожидания пользователей.
Оно обладает потенциалом для улучшения эффективности управления очередностью
и оптимизации процессов в учебных заведениях.\par
В целом, разработка мобильного приложения "Очередь"
представляет собой перспективное направление,
способное значительно улучшить пользовательский опыт
и эффективность работы в образовательной среде.
Дальнейшее тестирование и развитие приложения могут помочь реализовать
его полный потенциал и достичь поставленных целей.\par

