\section*{\LARGE Цель практической работы}
\addcontentsline{toc}{section}{Цель практической работы}

\textbf{Задание 1}

\begin{enumerate}
	\item Вынести повторяющиеся атрибуты в стили.
	\item Добавить новые шрифты в экраны.
	\item Адаптировать интерфейс под темную тему.
\end{enumerate}

\textbf{Задание 2}

Реализовать экран \textbf{Поиск} без выполнения поискового запроса.
Экран должен содержать заголовок, кнопку "<Назад"> и поле ввода поискового
запроса с кнопкой "<Очистить">.
Обязательные условия:

\begin{enumerate}
	\item Нажатие кнопки "<Назад"> возвращает пользователя на предыдущий экран.
	\item В поле ввода поискового запроса отображается подсказка, если поле
		ввода пустое.
	\item При нажатии на поле ввода появляется клавиатура.
	\item Если в поле ввода поискового запроса введён какой-то текст, то
		отображается кнопка "<Очистить">.
	\item Если в поле ввода поискового запроса нет никакого текста, то кнопка
		"<Очистить"> не отображается.
	\item Нажатие на кнопку "<Очистить"> удаляет текст в поле ввода поискового
		запроса и скрывает клавиатуру.
\end{enumerate}

\textbf{Задание 3}

Реализовать сохранение текста поискового запроса в жизненном цикле, то
есть если в поле ввода поискового запроса есть какой-то текст, то при
повороте устройства в ландшафтную ориентацию и обратно текст в поле
ввода поискового запроса не исчезает.

\clearpage

\section*{\LARGE Выполнение практической работы}
\addcontentsline{toc}{section}{Выполнение практической работы}

\section{Стили}

При разработке приложений для платформы Android важно не только писать
эффективный код на Java или Kotlin, но и уделять внимание переиспользованию
кода и ресурсов. Одним из инструментов, способствующих этому, являются стили.
Стиль представляет собой специальный вид ресурсов, который содержит
определенный набор значений атрибутов для View-элемента.
Описывается он в XML-формате и позволяет задать общие характеристики
для нескольких элементов интерфейса.\par
Для создания стилей в проекте Android Studio необходимо создать
новый ресурсный файл в папке values. Назовите его, например, styles.xml.
Внутри этого файла определите объекты стилей, используя следующий формат:

\begin{lstlisting}[language=XML]
<?xml version="1.0" encoding="utf-8"?>
<resources>
    <style name="Base.MyTheme" parent="Theme.Material3.DayNight.NoActionBar">
        <item name="android:windowBackground">@color/white</item>
        <item name="android:textColor">@color/black</item>
        <item name="android:editTextStyle">@style/EditTextStyle</item>
        <item name="android:buttonStyle">@style/ButtonStyle</item>
        <item name="android:fontFamily">@font/font</item>
    </style>
    <style name="EditTextStyle" parent="Widget.AppCompat.EditText">
        <item name="android:background">@drawable/edittext_bg</item>
        <item name="android:textColor">@color/black</item>
        <item name="android:textColorHint">@color/black</item>
        <item name="android:backgroundTint">@color/light_blue</item>
        <item name="android:layout_height">@dimen/edit_text_height</item>
    </style>
    <style name="ButtonStyle" parent="Widget.AppCompat.Button">
        <item name="android:textColor">@color/white</item>
        <item name="android:background">@color/blue</item>
    </style>
    <style name="MyTheme" parent="Base.MyTheme" />
</resources>
\end{lstlisting}

И чтобы использовать стиль с указанными значениями в вёрстке,
достаточно сделать следующее:

\begin{lstlisting}[language=XML]
<application
	android:theme="@style/MyTheme"
	.../>
	...
</application>
\end{lstlisting}

\section{Шрифт}

В приложениях Android, помимо функциональных аспектов,
важно также уделять внимание визуальному оформлению, включая выбор шрифтов.
По умолчанию Android Studio использует удобный для чтения шрифт,
но для создания приложений с уникальным стилем часто требуется использование
специальных шрифтов, особенно при разработке для крупных компаний
с собственным брендом и требованиями к визуальному оформлению.\par
Для добавления пользовательских шрифтов в приложение необходимо создать
специальную ресурсную директорию. После этого, файлы шрифтов следует
скопировать в созданную папку. Обратите внимание,
что в именах ресурсных файлов не допускаются пробелы и дефисы.
Используйте в качестве разделителя подчёркивание.

\begin{lstlisting}[language=XML]
<?xml version="1.0" encoding="utf-8"?>
<font-family xmlns:android="http://schemas.android.com/apk/res/android">
    <font
        android:font="@font/font"
        android:fontStyle="normal"
        android:fontWeight="400"
        />
</font-family>
\end{lstlisting}

Чтобы применить добавленные шрифты для View-элементов,
необходимо указать имя шрифта в качестве значения атрибута fontFamily
в разметке XML соответствующего элемента.

\begin{lstlisting}[language=XML]
<style name="Base.MyTheme" parent="Theme.Material3.DayNight.NoActionBar">
	...
	<item name="android:fontFamily">@font/font</item>
</style>
\end{lstlisting}

Настройка шрифтов для View-элементов позволяет создавать приложения
с индивидуальным стилем и уникальным визуальным оформлением.
Путем добавления пользовательских шрифтов в проект и использования
их в разметке XML можно значительно улучшить визуальное впечатление
от приложения.

\section{Темная тема}

Чтобы поддержать тёмную тему, необходимо настроить соответствующие ресурсы.
Создайте или найдите папку values-night в папке res
и скопируйте туда файл themes.xml из values.\par
В файле themes.xml определяются стили и цвета для ночного режима,
которые будут использоваться в приложении.
Например, можно задать цвета элементов интерфейса и статус-бара.

\begin{lstlisting}[language=XML]
<?xml version="1.0" encoding="utf-8"?>
<resources>
    <style name="Base.MyTheme" parent="Theme.Material3.DayNight.NoActionBar">
        <item name="android:windowBackground">@color/black</item>
        <item name="android:textColor">@color/white</item>
        <item name="android:editTextStyle">@style/EditTextStyle</item>
        <item name="android:buttonStyle">@style/ButtonStyle</item>
        <item name="android:fontFamily">@font/font</item>
    </style>
    <style name="EditTextStyle" parent="Widget.AppCompat.EditText">
        <item name="android:background">@drawable/edittext_bg</item>
        <item name="android:textColor">@color/white</item>
        <item name="android:textColorHint">@color/white</item>
        <item name="android:backgroundTint">@color/light_blue</item>
        <item name="android:layout_height">@dimen/edit_text_height</item>
    </style>
    <style name="ButtonStyle" parent="Widget.AppCompat.Button">
        <item name="android:textColor">@color/white</item>
        <item name="android:background">@color/blue</item>
    </style>
    <style name="MyTheme" parent="Base.MyTheme" />
</resources>
\end{lstlisting}

Чтобы увидеть, как приложение выглядит в ночной теме,
воспользуйтесь окном предпросмотра в Android Studio.
Выберите режим Night в верхней части окна предпросмотра.

\section{Реализация "<Поиска">}

Данный экран является ключевым для обеспечения удобства использования
приложения и обеспечения интуитивно понятного интерфейса для пользователей.

Листнинг реализации обработки виджетов:

\begin{lstlisting}[language=Java]
binding.editTextSearch.addTextChangedListener(object : TextWatcher {
	override fun beforeTextChanged(
		s: CharSequence?, start: Int, count: Int, after: Int
		) {}
	override fun onTextChanged(
		s: CharSequence?, start: Int, before: Int, count: Int
		) {
		if (s.isNullOrEmpty()) {
			binding.buttonClear.visibility = View.INVISIBLE
		} else {
			binding.buttonClear.visibility = View.VISIBLE
		}
	}
	override fun afterTextChanged(s: Editable?) {
	}
})
binding.buttonClear.setOnClickListener {
	binding.editTextSearch.text.clear()
}
\end{lstlisting}

\section{Хранение данных}

Чтобы не потерять данные при пересоздании, у activity существует
Instance State --- изначальное состояние. Для сохранения и получения
состояния в жизненном цикле activity есть два метода: onSaveInstanceState() и
onRestoreInstanceState().

\begin{lstlisting}[language=Java]
companion object {
	const val EDIT_TEXT_STATE_KEY = "EDIT_TEXT_STATE_KEY"
	const val SEARCH_DEF = ""
}
override fun onSaveInstanceState(outState: Bundle) {
	super.onSaveInstanceState(outState)
	outState.putString(
		EDIT_TEXT_STATE_KEY
		, binding.editTextSearch.text.toString()
		)
}
override fun onRestoreInstanceState(savedInstanceState: Bundle) {
	super.onRestoreInstanceState(savedInstanceState)
	val editTextState = savedInstanceState.getString(
		EDIT_TEXT_STATE_KEY, SEARCH_DEF
		)
	binding.editTextSearch.setText(editTextState)
}
\end{lstlisting}

\clearpage

\section*{\LARGE Вывод}
\addcontentsline{toc}{section}{Вывод}

Внедрение светлой и тёмной темы в Android позволяет автоматически переключать
приложения между ними. Для этого нужно настроить соответствующие ресурсы.
Это позволяет улучшить пользовательский опыт
и адаптировать интерфейс к различным условиям освещения.\par
Сохранение текста поисковых запросов между сеансами использования
приложения повышает удобство для пользователей,
а создание экрана поиска без выполнения запроса позволяет пользователям
изучать интерфейс поиска до его фактического использования.
Эти аспекты важны для создания качественного
и удобного в использовании приложения.

