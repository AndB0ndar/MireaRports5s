\section{Выполнение практической работы}

\section{Настройка основных параметров коммутатора}

\subsection{Настройте базовые параметры каждого коммутатора}

\begin{verbatim}
enable
config t
no ip domain lookup
hostname S1\_BONDAR
service password-encryption
banner motd # You must be authorizeded! #
enable secret class
line console 0
password cisco
login
line vty 0 15
password cisco
login
logging synchronous
interface range f0/5, f0/7-24, g0/1-2  // S1
interface range f0/5-17, f0/19-24, g0/1-2  // S2
interface range f0/5-17, f0/19-24, g0/1-2  // S3
shutdown
vlan 99
name Management
vlan 13
name Staff
interface f0/6  // S1
interface f0/18  // S2
interface f0/18  // S3
switchport mode access
switchport access vlan 13
no shutdown
interface vlan 99
ip address 192.168.99.11 255.255.255.0
no shutdown
end
copy running-config startup-config
\end{verbatim}

\section{Настройка протокола PAgP}
\subsection{Настройте PAgP на S1\_ФАМИЛИЯ и S3}

\begin{verbatim}
config t
interface range f0/3-4
channel-group 1 mode desirable  // S1
channel-group 1 mode auto  // S3
no shutdown
\end{verbatim}

\subsection{Проверьте конфигурации на портах}

\begin{verbatim}
show run interface f0/3
show interfaces f0/3 switchport
\end{verbatim}

\subsection{Убедитесь, что порты объединены}

\begin{verbatim}
show etherchannel summary
\end{verbatim}

\textbf{Что означают флаги «SU» и «P» в сводных данных по Ethernet?}\\
Флаг P указывает, что порты объединены в порт-канал.\\
Флаг S указывает, что порт-канал является EtherChannel уровня 2.\\
Флаг U указывает, что EtherChannel используется.\\

\subsection{Настройте транковые порты}

\textbf{S1 S3}

\begin{verbatim}
config t
interface port-channel 1
switchport mode trunk
switchport trunk native vlan 99
\end{verbatim}

\subsection{Убедитесь в том, что порты настроены в качестве транковых}

Выполните команды \verb|show running-startup| идентификатор-интерфейса
на S1\_ФАМИЛИЯ и S3.
Какие команды включены в список для интерфейсов F0/3 и F0/4
на обоих коммутаторах?

\begin{verbatim}
switchport trunk native vlan 99
switchport mode trunk
\end{verbatim}

Сравните результаты с текущей конфигурацией для интерфейса Po1.
Запишите наблюдения.

Команды, относящиеся к конфигурации таранков, одинаковы.
Когда команды таранков были применены к EtherChannel,
команды также повлияли на отдельные ссылки в пакете.

\begin{verbatim}
!
interface Port-channel1
 switchport trunk native vlan 99
 switchport mode trunk
!
interface FastEthernet0/1
!
interface FastEthernet0/2
!
interface FastEthernet0/3
 switchport trunk native vlan 99
 switchport mode trunk
 channel-group 1 mode desirable
!
\end{verbatim}

Выполните команды \verb|show interfaces trunk| и \verb|show spanning-tree|
на S1\_ФАМИЛИЯ и S3.
Какой транковый порт включен в список?
Какая используется сеть native VLAN?
Какой вывод можно сделать на основе выходных данных?

Указанный таранковый порт - Po1.
Собственная VLAN - 99.
После объединения ссылок в некоторых командах отображается только
агрегированный интерфейс.

Какие значения стоимости и приоритета порта
для агрегированного канала отображены
в выходных данных команды show spanning-tree?

Стоимость порта для Po1 равна 12, а приоритет порта равен 128.

\begin{verbatim}
S1_BONDAR#show interfaces trunk
Port        Mode         Encapsulation  Status        Native vlan
Po1         on           802.1q         trunking      99
Port        Vlans allowed on trunk
Po1         1-1005
Port        Vlans allowed and active in management domain
Po1         1,13,99
Port        Vlans in spanning tree forwarding state and not pruned
Po1         1,13,99
S1_BONDAR#
\end{verbatim}

\begin{verbatim}
S1_BONDAR#show spanning-tree 
VLAN0001
  Spanning tree enabled protocol ieee
  Root ID    Priority    32769
             Address     0002.176D.07E9
             Cost        12
             Port        27(Port-channel1)
             Hello Time  2 sec  Max Age 20 sec  Forward Delay 15 sec
  Bridge ID  Priority    32769  (priority 32768 sys-id-ext 1)
             Address     0060.7031.8697
             Hello Time  2 sec  Max Age 20 sec  Forward Delay 15 sec
             Aging Time  20
Interface        Role Sts Cost      Prio.Nbr Type
---------------- ---- --- --------- -------- --------------------------------
Fa0/1            Desg FWD 19        128.1    P2p
Fa0/2            Desg FWD 19        128.2    P2p
Po1              Root FWD 12        128.27   Shr
VLAN0013
  Spanning tree enabled protocol ieee
  Root ID    Priority    32781
             Address     0002.176D.07E9
             Cost        12
             Port        27(Port-channel1)
             Hello Time  2 sec  Max Age 20 sec  Forward Delay 15 sec
  Bridge ID  Priority    32781  (priority 32768 sys-id-ext 13)
             Address     0060.7031.8697
             Hello Time  2 sec  Max Age 20 sec  Forward Delay 15 sec
             Aging Time  20
Interface        Role Sts Cost      Prio.Nbr Type
---------------- ---- --- --------- -------- --------------------------------
Fa0/6            Desg FWD 19        128.6    P2p
Po1              Root FWD 12        128.27   Shr
VLAN0099
  Spanning tree enabled protocol ieee
  Root ID    Priority    32867
             Address     0002.176D.07E9
             Cost        12
             Port        27(Port-channel1)
             Hello Time  2 sec  Max Age 20 sec  Forward Delay 15 sec
  Bridge ID  Priority    32867  (priority 32768 sys-id-ext 99)
             Address     0060.7031.8697
             Hello Time  2 sec  Max Age 20 sec  Forward Delay 15 sec
             Aging Time  20
Interface        Role Sts Cost      Prio.Nbr Type
---------------- ---- --- --------- -------- --------------------------------
Po1              Root FWD 12        128.27   Shr
S1_BONDAR#
\end{verbatim}

\begin{verbatim}
S3#show spanning-tree 
VLAN0001
  Spanning tree enabled protocol ieee
  Root ID    Priority    32769
             Address     0002.176D.07E9
             This bridge is the root
             Hello Time  2 sec  Max Age 20 sec  Forward Delay 15 sec
  Bridge ID  Priority    32769  (priority 32768 sys-id-ext 1)
             Address     0002.176D.07E9
             Hello Time  2 sec  Max Age 20 sec  Forward Delay 15 sec
             Aging Time  20
Interface        Role Sts Cost      Prio.Nbr Type
---------------- ---- --- --------- -------- --------------------------------
Fa0/1            Desg FWD 19        128.1    P2p
Fa0/2            Desg FWD 19        128.2    P2p
Po1              Desg FWD 12        128.27   Shr
VLAN0013
  Spanning tree enabled protocol ieee
  Root ID    Priority    32781
             Address     0002.176D.07E9
             This bridge is the root
             Hello Time  2 sec  Max Age 20 sec  Forward Delay 15 sec

  Bridge ID  Priority    32781  (priority 32768 sys-id-ext 13)
             Address     0002.176D.07E9
             Hello Time  2 sec  Max Age 20 sec  Forward Delay 15 sec
             Aging Time  20
Interface        Role Sts Cost      Prio.Nbr Type
---------------- ---- --- --------- -------- --------------------------------
Fa0/18           Desg FWD 19        128.18   P2p
Po1              Desg FWD 12        128.27   Shr
VLAN0099
  Spanning tree enabled protocol ieee
  Root ID    Priority    32867
             Address     0002.176D.07E9
             This bridge is the root
             Hello Time  2 sec  Max Age 20 sec  Forward Delay 15 sec
  Bridge ID  Priority    32867  (priority 32768 sys-id-ext 99)
             Address     0002.176D.07E9
             Hello Time  2 sec  Max Age 20 sec  Forward Delay 15 sec
             Aging Time  20
Interface        Role Sts Cost      Prio.Nbr Type
---------------- ---- --- --------- -------- --------------------------------
Po1              Desg FWD 12        128.27   Shr
S3#
\end{verbatim}

\section{Настройка протокола LACP}

Протокол LACP является открытым протоколом агрегирования каналов,
разработанным на базе стандарта IEEE.
В части 3 необходимо выполнить настройку канала между S1\_ФАМИЛИЯ и S2
и канала между S2 и S3 с помощью протокола LACP.
Кроме того, отдельные каналы необходимо настроить в качестве транковых
и указать native vlan, прежде чем они будут объединены в каналы EtherChannel.

\subsection{Настройте LACP между S1\_ФАМИЛИЯ и S2}

Настройте канал между S2 и S3 как Po3,
используя LACP как протокол агрегирования каналов.
Канал на S1\_ФАМЛИЛИЯ должен быть в режиме active,
а канал на S2 – в режиме passive.

\begin{verbatim}
interface range f0/1-2
switchport mode trunk
switchport trunk native vlan 99
channel-group 2 mode active  // S1
channel-group 2 mode passive  // S2
no shutdown
\end{verbatim}

\subsection{Убедитесь, что порты объединены}

Какой протокол использует Po2 для агрегирования каналов?
Какие порты агрегируются для образования Po2?
Запишите команду, используемую для проверки.

Po2 использует LACP. F0/1 и F0/2 агрегируются с образованием Po2.

\begin{verbatim}
S1_BONDAR#show etherchannel summary
Flags:  D - down        P - in port-channel
        I - stand-alone s - suspended
        H - Hot-standby (LACP only)
        R - Layer3      S - Layer2
        U - in use      f - failed to allocate aggregator
        u - unsuitable for bundling
        w - waiting to be aggregated
        d - default port
Number of channel-groups in use: 2
Number of aggregators:           2
Group  Port-channel  Protocol    Ports
------+-------------+-----------+----------------------------------------------
1      Po1(SU)           PAgP   Fa0/3(P) Fa0/4(D) 
2      Po2(SU)           LACP   Fa0/1(P) Fa0/2(P) 
S1_BONDAR#
\end{verbatim}

\begin{verbatim}
S2#show etherchannel summary
Flags:  D - down        P - in port-channel
        I - stand-alone s - suspended
        H - Hot-standby (LACP only)
        R - Layer3      S - Layer2
        U - in use      f - failed to allocate aggregator
        u - unsuitable for bundling
        w - waiting to be aggregated
        d - default port
Number of channel-groups in use: 1
Number of aggregators:           1
Group  Port-channel  Protocol    Ports
------+-------------+-----------+----------------------------------------------
2      Po2(SU)           LACP   Fa0/1(P) Fa0/2(P) 
S2#
\end{verbatim}

\subsection{Настройте LACP между S2 и S3}

Аналогично настройте канал между S2 и S3 как Po3,
используя LACP как протокол агрегирования каналов.

\begin{verbatim}
interface range f0/3-4  // S2
interface range f0/1-2  // S3
switchport mode trunk
switchport trunk native vlan 99
channel-group 3 mode active  // S2
channel-group 3 mode passive  // S3
no shutdown
\end{verbatim}

b. Убедитесь в том, что канал EtherChannel образован.

\begin{verbatim}
S2#show etherchannel summary
Flags:  D - down        P - in port-channel
        I - stand-alone s - suspended
        H - Hot-standby (LACP only)
        R - Layer3      S - Layer2
        U - in use      f - failed to allocate aggregator
        u - unsuitable for bundling
        w - waiting to be aggregated
        d - default port
Number of channel-groups in use: 2
Number of aggregators:           2

Group  Port-channel  Protocol    Ports
------+-------------+-----------+----------------------------------------------

2      Po2(SU)           LACP   Fa0/1(P) Fa0/2(P) 
3      Po3(SU)           LACP   Fa0/3(P) Fa0/4(P) 
S2#
\end{verbatim}

\begin{verbatim}
S3#show etherchannel summary
Flags:  D - down        P - in port-channel
        I - stand-alone s - suspended
        H - Hot-standby (LACP only)
        R - Layer3      S - Layer2
        U - in use      f - failed to allocate aggregator
        u - unsuitable for bundling
        w - waiting to be aggregated
        d - default port
Number of channel-groups in use: 2
Number of aggregators:           2
Group  Port-channel  Protocol    Ports
------+-------------+-----------+----------------------------------------------
1      Po1(SU)           PAgP   Fa0/3(P) Fa0/4(D) 
3      Po3(SU)           LACP   Fa0/1(P) Fa0/2(P) 
S3#
\end{verbatim}

\subsection{Проверьте наличие сквозного соединения}

Убедитесь в том, что все устройства могут передавать друг другу эхо-запросы
в пределах одной сети VLAN.
Если нет, устраните неполадки,
чтобы установить связь между конечными устройствами.

192.168.99.11
192.168.99.12
192.168.99.13

192.168.13.1
192.168.13.2
192.168.13.3

Да, между компьютерами. Да, между коммутаторами.
Между компьютером и коммутатором не пингуется.

\section{Ответы на контрольные вопросы}
\subsection{Дайте определение понятию “агрегирование каналов”. Опишите 
преимущества технологии EtherChannel.}
Агрегирование каналов — это объединение нескольких физических 
портов в одну логическую магистраль на канальном уровне модели OSI с 
целью образования высокоскоростного канала передачи данных и 
повышения отказоустойчивости. Все избыточные связи в одном 
агрегированном канале остаются в рабочем состоянии, а имеющийся трафик 
распределяется между ними для достижения балансировки нагрузки.

Технология EtherChannel имеет много достоинств:
Большинство задач конфигурации выполняется на интерфейсе 
EtherChannel, а не на отдельных портах.
EtherChannel использует существующие порты коммутатора.
Между каналами, которые являются частью одного и того же 
EtherChannel, происходит распределение нагрузки. Может быть 
реализован один или несколько методов балансировки нагрузки.
EtherChannel создает объединение, которое рассматривается, как один 
логический канал. Если между двумя коммутаторами существует 
несколько объединений EtherChannel, протокол STP может 
блокировать одно из объединений во избежание петель коммутации.
EtherChannel предоставляет функции избыточности, поскольку общий 
канал считается одним логическим соединением. Кроме того, потеря 
одного физического соединения в пределах канала не приводит к 
изменению в топологии. Поэтому перерасчет STP не требуется.

\subsection{Опишите назначение технологии EtherChannel. Какие 
ограничения существуют при использовании технологии 
EtherChannel?}
EtherChannel --- это технология агрегации каналов, которая 
группирует несколько физических каналов Ethernet вместе в один логический 
канал. Он используется для обеспечения отказоустойчивости, распределения 
нагрузки, увеличения пропускной способности и избыточности между 
коммутаторами, маршрутизаторами и серверами.

EtherChannel имеет следующие ограничения реализации:
Нельзя одновременно использовать разные типы интерфейсов.
Коммутатор Cisco Catalyst 2960 уровня 2 в настоящее время 
поддерживает до шести каналов EtherChannel.
Конфигурация порта отдельного участника группы EtherChannel 
должна выполняться согласованно на обоих устройствах.
Каждый канал EtherChannel имеет логический интерфейс 
агрегированного канала. Настройка интерфейса агрегированного 
канала применяется на все физические интерфейсы, связанные с этим 
каналом.

\subsection{Дайте характеристику протоколу PAgP. Какие настройки 
должны иметь все порты в группе для удачного создания 
агрегированного канала?}
PAgP — это проприетарный протокол Cisco, который предназначен 
для автоматизации создания каналов EtherChannel. Когда канал EtherChannel 
настраивается с помощью PAgP, пакеты PAgP пересылаются между портами 
с поддержкой EtherChannel в целях согласования создания канала. Когда 
PAgP определяет совпадающие соединения Ethernet, он группирует их в 
канал EtherChannel. Далее EtherChannel добавляется в дерево кратчайших 
путей как один порт.

В EtherChannel все порты обязательно должны иметь одинаковую 
скорость, одинаковые настройки дуплекса и одинаковые настройки VLAN.

\subsection{Перечислите и охарактеризуйте режимы работы протокола 
PAgP. При настройке каких режимов PAgP на обоих концах 
будет невозможно создать агрегированный канал (перечислите 2 
сценария)?}
Режимы PAgP:
\begin{itemize}
	\item On (Вкл) --- этот режим принудительно назначает интерфейс в 
		канал без использования PAgP. Интерфейсы, настроенные в режиме 
		On (Вкл), не обмениваются пакетами PAgP.
	\item PAgP desirable (рекомендуемый) --- этот режим PAgP помещает 
		интерфейс в активное состояние согласования, в котором 
		интерфейс инициирует согласование с другими интерфейсами 
		путем отправки пакетов PAgP.
	\item PAgP auto (автоматический) --- этот режим PAgP помещает 
		интерфейс в пассивное состояние согласования, в котором 
		интерфейс отвечает на полученные пакеты PAgP, но не инициирует 
		согласование PAgP.
\end{itemize}

2 сценария, когда на обоих концах будет невозможно создать 
агрегированный канал:
PAgP auto на обоих концах.
On на одном конце и PAgP desirable или PAgP auto на другом 
конце.

\subsection{Дайте характеристику протоколу LACP. Перечислите и 
охарактеризуйте режимы работы протокола LACP}
LACP определяется стандартом IEEE (802.3ad), который обеспечивает 
возможность объединения нескольких физических портов для создания 
единого логического канала. LACP обеспечивает возможность согласования 
коммутатором автоматического объединения путем отправки пакетов LACP 
на другой коммутатор. Протокол LACP можно использовать для упрощения 
работы с каналами EtherChannel в неоднородных средах.

Режимы LACP:
\begin{itemize}
	\item On (Вкл) — этот режим принудительно помещает интерфейс в 
		канал без использования LACP. Интерфейсы, настроенные в 
		режиме On, не обмениваются пакетами LACP.
	\item LACP active (активный) --- в этом режиме LACP порт 
		помещается в активное состояние согласования. В этом состоянии 
		порт инициирует согласование с другими портами путем отправки 
		пакетов LACP.
	\item LACP passive (пассивный) --- в этом режиме LACP порт 
		помещается в пассивное состояние согласования. В этом состоянии 
		порт отвечает на полученные пакеты LACP, но не инициирует 
		согласование пакетов LACP.
\end{itemize}

\subsection{При настройке каких режимов LACP на обоих концах будет 
невозможно создать агрегированный канал (перечислите 2 
сценария)? Опишите алгоритм создания агрегированного 
канала на коммутаторе}
Два сценария, когда на обоих концах будет невозможно создать 
агрегированный канал:
\begin{itemize}
	\item On на одном конце и LACP active или LACP passive на другом конце.
	\item LACP active на одном конце и LACP passive на другом конце.
\end{itemize}

Алгоритм создания агрегированного канала на коммутаторе:
\begin{itemize}
	\item Подключитесь к коммутатору.
	\item Настройте каждый физический порт, который вы хотите 
		объединить в порт-транк, удостоверившись, что порты находятся в 
		одной и той же VLAN и обладают совместимыми параметрами.
	\item Создайте логический интерфейс порт-транка (или 
		агрегированный интерфейс) на коммутаторе и добавьте в него 
		указанные физические порты.
	\item Настройте параметры порт-транка.
	\item Проверьте правильность настройки, убедившись, что порт-
		транк создан и физические порты правильно агрегированы.
\end{itemize}

\subsection{Опишите взаимодействие протокола STP с технологией 
EtherChannel. Какие два метода балансировки нагрузки могут 
быть реализованы с технологией EtherChannel?}
Когда протокол STP взаимодействует с технологией EtherChannel, он 
учитывает логический канал, созданный с помощью EtherChannel, как одно 
соединение. Это позволяет балансировать нагрузку на портах EtherChannel и 
обеспечивает надежность работы сети, так как STP будет рассматривать все 
физические порты EtherChannel как одно соединение и будет отключать 
избыточные пути для предотвращения возникновения петель.

Два метода балансировки нагрузки:
\begin{itemize}
	\item Балансировка нагрузки по MAC-адресам или IP-адресам.
	\item Балансировка нагрузки по портам.
\end{itemize}

\subsection{Какие параметры обязательно должны быть одинаковыми на 
всех интерфейсах EtherChannel для его корректного 
функционирования? Перечислите распространенные проблемы, 
с которыми можно столкнуться при работе с EtherChannel}
В EtherChannel все порты обязательно должны иметь одинаковую 
скорость, одинаковые настройки дуплекса и одинаковые настройки VLAN.
Распространенные проблемы EtherChannel:
Назначенные порты в EtherChannel не являются частью одной 
VLAN или не настроены как транки. Порты с различными native 
VLAN не могут образовать EtherChannel.
Транк был настроен на некоторых портах, которые составляют 
EtherChannel, но не на всех из них. Не рекомендуется настраивать 
режим транкинга на отдельных портах, составляющих EtherChannel. 
При настройке магистрального канала в EtherChannel проверьте 
режим транкинга в EtherChannel.
Если допустимый диапазон VLAN не совпадает, порты не 
формируют EtherChannel, даже если PAgP установлен в режим auto 
или desirable.
Параметры динамического согласования для PAgP и LACP не 
совместимы на обоих концах EtherChannel.

