\section{Выполнение практической работы}

\section{Настройка основных параметров коммутатора}
\subsection{Шаг 1: Создайте сеть согласно топологии}
Подключите устройства, как показано в топологии, и подсоедините необходимые кабели.
\subsection{Шаг 2: Выполните инициализацию и перезагрузку коммутаторов}

\subsection{Шаг 3: Настройте базовые параметры каждого коммутатора}


enable
config t

no ip domain lookup

hostname S1_BONDAR

service password-encryption

banner motd # You must be authorizeded! #

enable secret class

line console 0
password cisco
login

line vty 0 15
password cisco
login

logging synchronous

\textbf{S1} interface range f0/5, f0/7-24, g0/1-2
\textbf{S2} interface range f0/5-17, f0/19-24, g0/1-2
\textbf{S3} interface range f0/5-17, f0/19-24, g0/1-2
shutdown

vlan 99
name Management
vlan 13
name Staff

\textbf{S1} interface f0/6
\textbf{S2} interface f0/18
\textbf{S3} interface f0/18
switchport mode access
switchport access vlan 13
no shutdown

interface vlan 99
ip address 192.168.99.11 255.255.255.0
no shutdown


end
copy running-config startup-config

\section{Настройка протокола PAgP}
\subsection{Шаг 1: Настройте PAgP на S1_ФАМИЛИЯ и S3}
\textbf{S1}

config t
interface range f0/3-4
channel-group 1 mode desirable
no shutdown

\textbf{S3}

config t
interface range f0/3-4
channel-group 1 mode auto
no shutdown

\subsection{Шаг 2: Проверьте конфигурации на портах}

show run interface f0/3
show interfaces f0/3 switchport

\subsection{Шаг 3: Убедитесь, что порты объединены}

show etherchannel summary

\textbf{Что означают флаги «SU» и «P» в сводных данных по Ethernet?}\\
Флаг P указывает, что порты объединены в порт-канал.\\
Флаг S указывает, что порт-канал является EtherChannel уровня 2.\\
Флаг U указывает, что EtherChannel используется.\\

\subsection{Шаг 4: Настройте транковые порты}

\textbf{S1 S3}

config t
interface port-channel 1
switchport mode trunk
switchport trunk native vlan 99

\subsection{Шаг 5: Убедитесь в том, что порты настроены в качестве транковых}

a. Выполните команды show run interface идентификатор-интерфейса на S1_ФАМИЛИЯ и S3.
Какие команды включены в список для интерфейсов F0/3 и F0/4 на обоих коммутаторах?
Сравните результаты
с текущей конфигурацией для интерфейса Po1. Запишите наблюдения.
b. Выполните команды show interfaces trunk и show spanning-tree на S1_ФАМИЛИЯ и S3. Какой
транковый порт включен в список? Какая используется сеть native VLAN? Какой вывод можно
сделать на основе выходных данных?
Какие значения стоимости и приоритета порта для агрегированного канала отображены
в выходных данных команды show spanning-tree?
Часть 3: Настройка протокола LACP
Протокол LACP является открытым протоколом агрегирования каналов, разработанным на базе
стандарта IEEE. В части 3 необходимо выполнить настройку канала между S1_ФАМИЛИЯ и S2 и
канала между S2 и S3 с помощью протокола LACP. Кроме того, отдельные каналы необходимо
настроить в качестве транковых и указать native vlan, прежде чем они будут объединены в каналы
EtherChannel.
Шаг 1: Настройте LACP между S1_ФАМИЛИЯ и S2.
a. Настройте канал между S2 и S3 как Po3, используя LACP как протокол агрегирования каналов. Канал
на S1_ФАМЛИЛИЯ должен быть в режиме active, а канал на S2 – в режиме passive.
Шаг 2: Убедитесь, что порты объединены.
Какой протокол использует Po2 для агрегирования каналов? Какие порты агрегируются для
образования Po2? Запишите команду, используемую для проверки.
_______________________________________________________________________________________
_______________________________________________________________________________________
_______________________________________________________________________________________
Шаг 3: Настройте LACP между S2 и S3.
a. Аналогично настройте канал между S2 и S3 как Po3, используя LACP как протокол агрегирования
каналов.
b. Убедитесь в том, что канал EtherChannel образован.
Шаг 4: Проверьте наличие сквозного соединения.
Убедитесь в том, что все устройства могут передавать друг другу эхо-запросы в пределах одной сети
VLAN. Если нет, устраните неполадки, чтобы установить связь между конечными устройствами.
Вопросы для повторения
Что может препятствовать образованию каналов EtherChannel?
