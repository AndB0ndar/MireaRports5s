\section{Выполнение практической работы}


\section{Создание сети и настройка основных параметров устройства}

В первой части лабораторной работы вам предстоит создать топологию сети и настроить базовые
параметры для узлов ПК и коммутаторов.

\subsection{Создайте сеть согласно топологии}
Подключите устройства, как показано в топологии, и подсоедините необходимые кабели.

\subsection{Настройте базовые параметры для маршрутизатора}
Настроем базовыве параметры маршрутизатора.

\begin{lstlisting}
enabel
config t
hostname R1_BONDAR
no ip domain lookup
enable secret class
line console 0
password cisco
login
line vty 0 4
password cisco
login
service password-encryption
banner motd # You must be authorizeded! #
exit
copy running-config startup-config
clock set 15:00:00 21 Feb 2024
\end{lstlisting}

\begin{image}
	\includegrph{router0}
	\caption{Конфигурация маршрутизатора}
	\label{fig:switch:router:0}
\end{image}

\begin{image}
	\includegrph{router1}
	\caption{Конфигурация маршрутизатора}
	\label{fig:switch:router:1}
\end{image}

\subsection{Настройте базовые параметры каждого коммутатора}
Настроем базовыве параметры коммутатора.

\begin{lstlisting}
enable
config t
hostname S1
no ip domain lookup
enable secret class
line console 0
password cisco
login
line vty 0 15
password cisco
login
service password-encryption
banner motd # You must be authorizeded! #
exit
clock set 15:00:00 21 Feb 2024
copy running-config startup-config
\end{lstlisting}

\begin{image}
	\includegrph{switch0}
	\caption{Конфигурация коммутатора}
	\label{fig:switch:set:0}
\end{image}

\begin{image}
	\includegrph{switch1}
	\caption{Конфигурация коммутатора}
	\label{fig:switch:set:1}
\end{image}

\subsection{Настройте узлы ПК.}
Адреса ПК можно посмотреть в таблице адресации.


\section{Создание сетей VLAN и назначение портов коммутатора}
Во второй части вы создадите VLAN, как указано в таблице выше
, на обоих коммутаторах.
Затем вы назначите VLAN соответствующему интерфейсу 
и проверите настройки конфигурации.
Выполните следующие задачи на каждом коммутаторе

\subsection{Создайте сети VLAN на коммутаторах.}
Создайте и назовите необходимые VLAN на каждом коммутаторе из таблицы выше.

\begin{verbatim}
	vlan 13
	name Management
	vlan 23
	name Sales
	vlan 33
	name Operations
	vlan 999
	name Parking_Lot
	vlan 1000
	name Native
\end{verbatim}

Настройте интерфейс управления и шлюз по умолчанию на каждом коммутаторе,
используя информацию об IP-адресе в таблице адресации.

\begin{verbatim}
	interface vlan 13
	ip address 192.168.13.11 255.255.255.0  // S1
	ip address 192.168.13.12 255.255.255.0  // S2
	no shutdown
	exit
	ip default-gateway 192.168.13.1
\end{verbatim}

Назначьте все неиспользуемые порты коммутатора VLAN Parking\_Lot,
настройте их для статического режима доступа
и административно деактивируйте их.\par
\textbf{Примечание}.
Команда interface range полезна для выполнения этой задачи
с минимальным количеством команд.

\begin{verbatim}
	interface range f0/2 - 4 , f0/7 - 24 , g0/1 - 2  // S1
	interface range f0/2 - 17 , f0/19 - 24 , g0/1 - 2  // S2
	switchport mode access
	switchport access vlan 999
	shutdown
\end{verbatim}

\subsection{Назначьте сети VLAN соответствующим интерфейсам коммутатора}
Назначьте используемые порты соответствующей VLAN
(указанной в таблице VLAN выше)
и настройте их для режима статического доступа.

\begin{verbatim}
	*S1* interface f0/6
	*S2* interface f0/18
	switchport mode access
	*S1* switchport access vlan 23
	*S2* switchport access vlan 33
\end{verbatim}

Убедитесь, что VLAN назначены на правильные интерфейсы.

\begin{verbatim}
	show vlan brief
\end{verbatim}

\begin{image}
	\includegrph{switch_vlan}
	\caption{Сеть VLAN}
	\label{fig:switch:vlan}
\end{image}


\section{Конфигурация магистрального канала стандарта 802.1Q
	между коммутаторами}
В части 3 вы вручную настроите интерфейс F0/1 как транковый канал.

\subsection{Вручную настройте магистральный интерфейс F0/1
	на коммутаторах S1 и S2}

Настройте интерфейс F0/1 как транковый для обоих коммутаторов.

\begin{verbatim}
	interface f0/1
	switchport mode trunk
\end{verbatim}

Установите native VLAN 1000 на обоих коммутаторах.

\begin{verbatim}
	switchport trunk native vlan 1000
\end{verbatim}

Укажите, что VLAN X+10, X+20, X+30 и 1000 могут проходить по транковому каналу

\begin{verbatim}
	switchport trunk allowed vlan 13,23,33,1000
\end{verbatim}

Проверьте транковые каналы, native VLAN
и разрешенные VLAN через транковые каналы.

\begin{verbatim}
	show interfaces trunk
\end{verbatim}

\begin{image}
	\includegrph{switch_trunk0}
	\caption{Конфигурация транковых каналов}
	\label{fig:switch:trunk:0}
\end{image}

\subsection{Вручную настройте магистральный интерфейс F0/5 на коммутаторе S1.}
Настройте интерфейс S1 F0/5 с теми же параметрами транкового канала,
что и F0/1. Это транковый канал до маршрутизатора.

\begin{verbatim}
	interface f0/5  // S1
	switchport mode trunk
	switchport trunk native vlan 1000
	switchport trunk allowed vlan 13,23,33,1000
\end{verbatim}

Сохраните текущую конфигурацию в файл загрузочной конфигурации.

\begin{verbatim}
	copy running-config startup-config
\end{verbatim}

Проверьте транковый канал.

\begin{verbatim}
	show interfaces trunk
\end{verbatim}

\begin{image}
	\includegrph{switch_trunk}
	\caption{Конфигурация транковых каналов}
	\label{fig:switch:trunk}
\end{image}


\textbf{Что произойдет, если G0/0/1 на R1\_ФАМИЛИЯ будет отключен?}

Не будет отобразаться

\section{Настройка маршрутизации между сетями VLAN}
\subsection{Настройте маршрутизатор}

При необходимости активируйте интерфейс G0/0/1 на маршрутизаторе.

\begin{verbatim}
	interface g0/0/1
	no shutdown
\end{verbatim}

Настройте подинтерфейсы для каждой VLAN, как указано в таблице IP-адресации.
Все подинтерфейсы используют инкапсуляцию 802.1Q@.
Убедитесь, что подинтерфейсу для native VLAN не назначен IP-адрес.
Включите описание для каждого подинтерфейса.

\begin{verbatim}
	interface g0/0/1.13
	description Management VLAN
	encapsulation dot1q 13
	ip address 192.168.13.1 255.255.255.0
	interface g0/0/1.23
	encapsulation dot1q 23
	description Sales VLAN
	ip address 192.168.23.1 255.255.255.0
	interface g0/0/1.33
	encapsulation dot1q 33
	description Operations VLAN
	ip address 192.168.33.1 255.255.255.0
	interface g0/0/1.1000
	encapsulation dot1q 1000 native
	description Native VLAN
\end{verbatim}

Убедитесь, что подинтерфейсы работают.

\begin{verbatim}
	show ip interface brief
\end{verbatim}

\begin{image}
	\includegrph{routing}
	\caption{Маршрутизация между сетями VLAN}
	\label{fig:switch:routing}
\end{image}

\section{Проверьте, работает ли маршрутизация между VLAN}
\subsection{Выполните следующие тесты с PC-A. Все должно быть успешно.}

Отправьте эхо-запрос с PC-A на шлюз по умолчанию.

\begin{verbatim}
	ping 192.168.23.1
\end{verbatim}

Отправьте эхо-запрос с PC-A на PC-B@.

\begin{verbatim}
	ping 192.168.33.3
\end{verbatim}

Отправьте эхо-запрос с компьютера PC-A на коммутатор S2.

\begin{verbatim}
	ping 192.168.13.12
\end{verbatim}

\begin{image}
	\includegrph{pc0ping}
	\caption{Проверка работы маршрутизации между VLAN}
	\label{fig:pc0ping}
\end{image}

\subsection{Пройдите следующий тест с PC-B}
В окне командной строки на PC-B выполните команду tracert на адрес PC-A.
Какие промежуточные IP-адреса отображаются в результатах?

\begin{verbatim}
tracert 192.168.23.3
\end{verbatim}

\begin{verbatim}
Tracing route to 192.168.23.3 over a maximum of 30 hops: 

  1   0 ms      0 ms      4 ms      192.168.33.1
  2   0 ms      0 ms      0 ms      192.168.23.3

Trace complete.
\end{verbatim}

\begin{image}
	\includegrph{pc1ping}
	\caption{Проверка работы маршрутизации между VLAN}
	\label{fig:pc1ping}
\end{image}


\section{Ответы на вопросы}

\subsection{Что такое маршрутизация между VLAN? Какие бывают методы
	маршрутизации между VLAN?}

\textbf{VLAN маршрутизация} означает направление сетевого трафика
между виртуальными сетями (VLAN) при помощи специализированного маршрутизатора
или многоуровневого коммутатора.
Этот процесс позволяет направлять пакеты данных из одной VLAN в другую,
используя маршрутизацию между VLAN.

\textbf{Методы VLAN маршрутизации} включают:

\begin{itemize}
	\item Устаревший метод;
	\item Метод "Router-on-a-stick";
	\item Маршрутизацию на основе коммутатора 3-го уровня;
\end{itemize}

\subsection{Опишите устаревший метод маршрутизации между сетями VLAN. В 
чем заключается преимущество маршрутизации между VLAN с 
помощью коммутатора уровня 3?}

\subsection{Описание устаревшего метода маршрутизации между VLAN. В чем преимущество маршрутизации между VLAN с использованием коммутатора уровня 3?}

\textbf{Устаревший метод маршрутизации между VLAN} предполагает использование
маршрутизаторов с несколькими физическими интерфейсами.
Каждый из этих интерфейсов должен был быть соединен с отдельной сетью
и настроен с определенной подсетью.
В этом устаревшем подходе маршрутизаця между VLAN осуществляется путем
подключения различных физических интерфейсов маршрутизатора
к разным физическим портам коммутатора.
Порты коммутатора, подключенные к маршрутизатору, переводятся в режим доступа,
и каждый физический интерфейс назначается отдельной VLAN.
Таким образом, каждый интерфейс маршрутизатора способен принимать трафик
из VLAN, связанной с интерфейсом коммутатора, к которому он подключен,
и направлять трафик в другие VLAN, подключенные к другим интерфейсам.\par
\textbf{Преимущество маршрутизации между VLAN с использованием коммутатора
уровня 3} заключается в том, что устройство 3-го уровня обеспечивает
более эффективное управление передачей трафика между сегментами сети,
включая сегменты, созданные с использованием VLAN. Обычно полоса пропускания
коммутаторов 3-го уровня позволяет передавать миллионы пакетов
в секунду (pps), в то время как стандартные маршрутизаторы поддерживают
скорость коммутации от 100 тысяч до 1 миллиона пакетов в секунду.

\subsection{Дайте характеристику методу маршрутизации Router-on-a-Stick. В 
	чем заключается недостаток устаревшего метода маршрутизации 
	между сетями VLAN?}

\textbf{Метод "Router-on-a-Stick"} представляет собой конфигурацию
маршрутизатора, при которой один физический интерфейс маршрутизирует
трафик между несколькими VLAN. В этой топологии маршрутизатор
с внешними подынтерфейсами подключается через транковые каналы
к коммутатору 2-го уровня.\par
Интерфейс маршрутизатора настраивается для работы в режиме транка
и подключается к транковому порту коммутатора.
Маршрутизатор выполняет маршрутизацию между VLAN,
принимая трафик с метками VLAN на транковом интерфейсе,
поступающий от смежного коммутатора, и маршрутизируя его между VLAN
с помощью подынтерфейсов.
Затем маршрутизированный трафик отправляется обратно через
тот же физический интерфейс с соответствующей меткой VLAN для назначения.

\textbf{Недостаток устаревшего метода маршрутизации}: При использовании
устаревшего метода маршрутизации между VLAN создаются параллельные каналы
для создания транков между коммутаторами с целью агрегации
и резервирования каналов. Однако резервные каналы усложняют топологию
и могут привести к проблемам с подключением,
если не управлять ими должным образом.

\subsection{Опишите алгоритм настройки маршрутизации между сетями VLAN 
	методом Router-on-a-Stick. В чем заключается недостаток метода 
	маршрутизации Router-on-a-Stick?}

При использовании метода "<Router-on-a-Stick"> на каждом логическом
подынтерфейсе необходимо настроить соответствующие IP-адреса
и параметры VLAN. Также требуется настройка транка
и инкапсуляции на маршрутизаторе и соответствующем порту коммутатора.\par
Однако следует отметить, что маршрутизация между VLAN
с использованием метода "<Router-on-a-Stick"> не масштабируется при работе
с более чем 50 сетями VLAN.

\subsection{Опишите алгоритм настройки маршрутизации между VLAN с 
	помощью коммутатора уровня 3. Дайте определение понятию 
	"<подынтерфейс">.}

\textbf{Алгоритм:}

\begin{enumerate}
	\item Включите маршрутизацию на коммутаторе\\
		Войдите в интерфейс командной строки коммутатора.
		Введите команду \texttt{ip routing} для включения маршрутизации.
	\item Создайте VLAN и назначьте порты\\
		Создайте VLAN с помощью команды \texttt{vlan [номер VLAN]}.
		Назначьте порты VLAN с помощью команды
		\texttt{interface vlan [номер VLAN]}.
	\item Настройте IP-адреса и шлюзы для VLAN\\
		Назначьте IP-адрес и маску подсети каждому VLAN с помощью 
		команды \texttt{interface vlan [номер VLAN]}.
		Настройте шлюз по умолчанию для каждого VLAN с помощью 
		команды \texttt{ip default-gateway [IP-адрес шлюза]}.
	\item Создайте статические маршруты\\
		Создайте статические маршруты для пересылки трафика между VLAN.
		Используйте команду:
		\texttt{ip route [сеть назначения] [маска подсети] 
		[IP-адрес следующего перехода] [интерфейс]}.
	\item Проверьте конфигурацию\\
		Проверьте конфигурацию с помощью команды \texttt{show ip route}.
		Убедитесь, что маршруты созданы правильно и что трафик 
		пересылается между VLAN.
\end{enumerate}

\textbf{Подынтерфейсы} --- это программные виртуальные интерфейсы, 
связанные с одним физическим интерфейсом.

\subsection{Опишите алгоритм настройки маршрутизации на коммутаторе 
	уровня 3. В чем заключается недостаток использования 
	многоуровневых коммутаторов для маршрутизации между 
	VLAN?}

Алгоритм настройки маршрутизации на коммутаторе уровня 3:

\begin{enumerate}
	\item Настройте маршрутизируемый порт.
	\item Включите маршрутизацию.
	\item Настройте маршрутизацию.
	\item Проверка маршрутизации.
	\item Проверьте подключение.
\end{enumerate}

Недостаток в том, что для каждой VLAN нужен отдельный интерфейс 
маршрутизатора.

\subsection{Какие неполадки могут возникнуть при настройке 
	маршрутизации между VLAN и как их исправить? В каком 
	режиме должен находиться порт коммутатора при подключении 
	его к маршрутизатору для маршрутизации между VLAN 
	методом Router-on-a-Stick?}

Неполадки при настройке маршрутизации между VLAN и способы их 
исправления:
\begin{itemize}
	\item Отсутствующие сети VLAN\\
		Решение:
		\begin{itemize}
			\item Создайте (или повторно создайте) VLAN,
				если она не существует.
			\item Убедитесь, что порт хоста назначен правильной VLAN.
		\end{itemize}
	\item Проблемы магистрального порта коммутатора\\
		Решение:
		\begin{itemize}
			\item Убедитесь, что магистральные соединения настроены правильно.
			\item Убедитесь, что порт является магистральным портом и включен.
		\end{itemize}
	\item Неполадки в работе порта коммутатора\\
		Решение:
		\begin{itemize}
			\item Назначьте порт соответствующей сети VLAN.
			\item Убедитесь, что порт является портом доступа и включен.
			\item Неправильно настроен узел в неправильной подсети.
		\end{itemize}
	\item Неполадки в настройках маршрутизатора\\
		Решение:
		\begin{itemize}
			\item IPv4-адрес подынтерфейса маршрутизатора настроен 
				неправильно.
			\item Подынтерфейс маршрутизатора назначается
				с идентификатором VLAN.
		\end{itemize}
\end{itemize}

Порт коммутатора работает в режиме транка.

\subsection{Какими возможностями обладает коммутатор уровня 3 по 
	сравнению с коммутатором уровня 2? Между какими 
	устройствами необходимо настроить магистральный канал при 
	использовании метода Router-on-a-Stick?}

Корпоративные локальные сети используют коммутаторы уровня 3 для 
обеспечения маршрутизации между VLAN. Коммутаторы уровня 3 
используют аппаратную коммутацию для достижения более высоких 
скоростей обработки пакетов, чем маршрутизаторы. Коммутаторы уровня 3 
также обычно используются в корпоративных сетях стойка уровня 
распределения.\par
Возможности коммутатора уровня 3 включают в себя возможность 
выполнения следующих действий:

\begin{itemize}
	\item Маршрутизация от одной VLAN к другой с использованием 
		нескольких коммутируемых виртуальных интерфейсов (SVI).
	\item Преобразовать порт коммутатора уровня 2 в интерфейс уровня 3 
		(т.е. маршрутизируемый порт). Маршрутизируемый порт — 
		простой интерфейс 3-го уровня, аналогичный физическому 
		интерфейсу на маршрутизаторе Cisco IOS.
\end{itemize}

При использовании метода Router-on-a-Stick между маршрутизаторами 
и коммутаторами необходимо настроить магистральный канал.

