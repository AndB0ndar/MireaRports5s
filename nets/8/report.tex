\section{Выполнение практической работы}

\section{Создание сети и настройка основных параметров устройства}

\subsection{Настройте базовые параметры каждого коммутатора}

\begin{verbatim}
no ip domain lookup
hostname S1_BONDAR
line console 0
password cisco
login
logging synchronous
line vty 0 4
password cisco
login
enable secret class
interface range f0/1-24, g0/1-2
shutdown
end
copy running-config startup-config
\end{verbatim}

\section{Настройка сетей VLAN, native VLAN и транковых каналов}

\subsection{Создайте сети VLAN}

Используйте соответствующие команды,
чтобы создать сети VLAN 10 и 99 на всех коммутаторах.
Присвойте сети VLAN 10 имя User\_ФАМИЛИЯ, а сети VLAN 99 --- имя Management.

\begin{verbatim}
vlan 10
name User_BONDAR
vlan 99
name Management
\end{verbatim}

\subsection{Переведите пользовательские порты в режим доступа
	и назначьте сети VLAN}

Для интерфейса F0/6 S1\_ФАМИЛИЯ и интерфейса F0/18 S3 включите порты,
настройте их в качестве портов доступа и назначьте их сети VLAN 10.

\begin{verbatim}
interface f0/6  // S1
interface f0/18  // S3
no shutdown
switchport mode access
switchport access vlan 10
\end{verbatim}

\subsection{Настройте транковые порты и назначьте их сети native VLAN 99}

Для портов F0/1 и F0/3 на всех коммутаторах включите порты,
настройте их в качестве транковых и назначьте их сети native VLAN 99.

\begin{verbatim}
interface range f0/1,f0/3
no shutdown
switchport mode trunk
switchport trunk native vlan 99
\end{verbatim}

\subsection{Настройте административный интерфейс на всех коммутаторах}

Используя таблицу адресации, настройте на всех коммутаторах административный
интерфейс с соответствующим IP-адресом.

\begin{verbatim}
interface vlan 99
ip address 192.168.4.11 255.255.255.0  // S1
ip address 192.168.4.12 255.255.255.0  // S2
ip address 192.168.4.13 255.255.255.0  // S3
\end{verbatim}

\subsection{Проверка конфигураций и возможности подключения}

Используйте команду \texttt{show vlan brief} на всех коммутаторах,
чтобы убедиться в том, что все сети VLAN внесены в таблицу VLAN
и назначены правильные порты.\par
Используйте команду \texttt{show interfaces trunk} на всех коммутаторах
для проверки магистральных интерфейсов.\par
Используйте команду \texttt{show running-config} на всех коммутаторах,
чтобы проверить все остальные конфигурации.

\begin{image}
	\includegrph{Screenshot from 2024-03-18 17-02-37}
	\caption{Результат комадны show vlan brief}
\end{image}

\begin{image}
	\includegrph{Screenshot from 2024-03-18 17-04-04}
	\caption{Результат комадны show interfaces trunk}
\end{image}

\begin{image}
	\includegrph{Screenshot from 2024-03-18 17-05-28}
	\caption{Результат комадны show running-config}
\end{image}

\textbf{Какие настройки используются для режима протокола spanning-tree
	на коммутаторах Cisco?}

Режим связующего дерева по умолчанию --- PVST+.

\textbf{Проверьте подключение между компьютерами PC-A и PC-C.
	Удалось ли получить ответ на эхо-запрос?}

Да

Если эхо-запрос выполнить не удалось, следует выполнять отладку до тех пор,
пока проблема не будет решена.

\section{Настройка корневого моста и проверка сходимости PVST+}
\subsection{Определите текущий корневой мост}

\textbf{С помощью какой команды пользователи определяют состояние
протокола spanning-tree коммутатора Cisco Catalyst для всех сетей VLAN?}

\begin{verbatim}
show spanning-tree
\end{verbatim}

\begin{image}
	\includegrph{Screenshot from 2024-03-18 17-13-06}
	\caption{Результат комадны show spanning-tree на S1}
\end{image}

\begin{image}
	\includegrph{Screenshot from 2024-03-18 17-13-34}
	\caption{Результат комадны show spanning-tree на S2}
\end{image}

\begin{image}
	\includegrph{Screenshot from 2024-03-18 17-13-55}
	\caption{Результат комадны show spanning-tree на S3}
\end{image}

\textbf{Каков приоритет моста коммутатора S1\_ФАМИЛИЯ для сети VLAN 1?}

32769

\textbf{Каков приоритет моста коммутатора S2 для сети VLAN 1?}

32769

\textbf{Каков приоритет моста коммутатора S3 для сети VLAN 1?}

32769

\textbf{Какой коммутатор является корневым мостом?}

S3

\textbf{Почему этот коммутатор выбран в качестве корневого моста?}

По умолчанию связующее дерево выбирает корневой мост
на основе наименьшего MAC-адреса.

\subsection{Настройте основной и вспомогательный корневые мосты
	для всех существующих сетей VLAN}

Настройте коммутатор S2 в качестве основного корневого моста для всех существующих сетей VLAN.

\begin{verbatim}
spanning-tree vlan 1,10,99 root primary
\end{verbatim}

Настройте коммутатор S1\_ФАМИЛИЯ в качестве
вспомогательного корневого моста для всех существующих сетей VLAN.
Запишите команду в строке ниже.

\begin{verbatim}
spanning-tree vlan 1,10,99 root secondary
\end{verbatim}

\begin{image}
	\includegrph{Screenshot from 2024-03-18 19-20-31}
	\caption{Результат комадны show spanning-tree}
\end{image}

Используйте команду \texttt{show spanning-tree} для ответа
на следующие вопросы:

\textbf{Какой приоритет моста используется
для коммутатора S1\_ФАМИЛИЯ в сети VLAN 1?}

24577

\textbf{Какой приоритет моста используется для коммутатора S2 в сети VLAN 1?}

24577

\textbf{Какой интерфейс в сети находится в состоянии блокировки?}

Интерфейс F0/3 на коммутаторе S3

\subsection{Измените топологию 2-го уровня и проверьте сходимость}

Выполните команду\texttt{debug spanning-tree events} 
в привилегированном режиме на коммутаторе S3.

\textbf{НЕ РАБОТАЕТ}

Через какие состояния портов проходит каждая сеть VLAN
на интерфейсе F0/3 в процессе схождения сети?

Используя временную метку из первого и последнего сообщений отладки STP,
рассчитайте время (округляя до секунды),
которое потребовалось для схождения сети.
Рекомендация. Формат временной метки сообщений отладки: чч.мм.сс.мс

\section{Настройка Rapid PVST+, PortFast, BPDU Guard и проверка сходимости}

\subsection{Настройте Rapid PVST+}

Настройте S1 для использования Rapid PVST+.

\begin{verbatim}
spanning-tree mode rapid-pvst
\end{verbatim}

Настройте коммутаторы S2 и S3 для Rapid PVST+.

\begin{verbatim}
spanning-tree mode rapid-pvst
\end{verbatim}

Проверьте конфигурации с помощью команды
\texttt{show running-config | include spanning-tree mode}.\\

\textit{spanning-tree mode rapid-pvst}

\subsection{Настройте PortFast и BPDU Guard на портах доступа}

Настройте F0/6 на S1\_ФАМИЛИЯ с помощью функции PortFast.

\begin{verbatim}
interface f0/6
spanning-tree portfast
\end{verbatim}

Настройте F0/6 на S1\_ФАМИЛИЯ с помощью функции BPDU Guard.

\begin{verbatim}
interface f0/6
spanning-tree bpduguard enable
\end{verbatim}

Глобально настройте все нетранковые порты
на коммутаторе S3 с помощью функцииPortFast.

\begin{verbatim}
spanning-tree portfast default
\end{verbatim}

Глобально настройте все нетранковые порты
на коммутаторе S3 с помощью функции BPDU.

\begin{verbatim}
spanning-tree portfast bpduguard default
\end{verbatim}

\subsection{Проверьте сходимость Rapid PVST+}

Выполните команду\texttt{debug spanning-tree events} 
в привилегированном режиме на коммутаторе S3.

\textbf{НЕ РАБОТАЕТ}

Измените топологию, отключив интерфейс F0/1 на коммутаторе S3.
Используя временную метку из первого и последнего сообщений отладки RSTP,
рассчитайте время, которое потребовалось для схождения сети.

\begin{image}
	\includegrph{Screenshot from 2024-03-18 17-39-15}
	\caption{Отправки эхо запросов с копьютера на компьютер}
\end{image}

