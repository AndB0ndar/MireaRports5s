\section{Выполнение практической работы}

\section{Создание сети и настройка основных параметров устройства}

\subsection{Настройте базовые параметры каждого коммутатора}

\begin{verbatim}
no ip domain lookup
hostname S1_BONDAR
line console 0
password cisco
login
logging synchronous
line vty 0 4
password cisco
login
enable secret class
interface range f0/1-24, g0/1-2
shutdown
end
copy running-config startup-config
\end{verbatim}

\section{Настройка сетей VLAN, native VLAN и транковых каналов}

\subsection{Создайте сети VLAN}

Используйте соответствующие команды,
чтобы создать сети VLAN 10 и 99 на всех коммутаторах.
Присвойте сети VLAN 10 имя User\_ФАМИЛИЯ, а сети VLAN 99 --- имя Management.

\begin{verbatim}
vlan 10
name User_BONDAR
vlan 99
name Management
\end{verbatim}

\subsection{Переведите пользовательские порты в режим доступа
	и назначьте сети VLAN}

Для интерфейса F0/6 S1\_ФАМИЛИЯ и интерфейса F0/18 S3 включите порты,
настройте их в качестве портов доступа и назначьте их сети VLAN 10.

\begin{verbatim}
interface f0/6  // S1
interface f0/18  // S3
no shutdown
switchport mode access
switchport access vlan 10
\end{verbatim}

\subsection{Настройте транковые порты и назначьте их сети native VLAN 99}

Для портов F0/1 и F0/3 на всех коммутаторах включите порты,
настройте их в качестве транковых и назначьте их сети native VLAN 99.

\begin{verbatim}
interface range f0/1,f0/3
no shutdown
switchport mode trunk
switchport trunk native vlan 99
\end{verbatim}

\subsection{Настройте административный интерфейс на всех коммутаторах}

Используя таблицу адресации, настройте на всех коммутаторах административный
интерфейс с соответствующим IP-адресом.

\begin{verbatim}
interface vlan 99
ip address 192.168.4.11 255.255.255.0  // S1
ip address 192.168.4.12 255.255.255.0  // S2
ip address 192.168.4.13 255.255.255.0  // S3
\end{verbatim}

\subsection{Проверка конфигураций и возможности подключения}

Используйте команду \texttt{show vlan brief} на всех коммутаторах,
чтобы убедиться в том, что все сети VLAN внесены в таблицу VLAN
и назначены правильные порты.\par
Используйте команду \texttt{show interfaces trunk} на всех коммутаторах
для проверки магистральных интерфейсов.\par
Используйте команду \texttt{show running-config} на всех коммутаторах,
чтобы проверить все остальные конфигурации.

\begin{image}
	\includegrph{Screenshot from 2024-03-18 17-02-37}
	\caption{Результат комадны show vlan brief}
\end{image}

\begin{image}
	\includegrph{Screenshot from 2024-03-18 17-04-04}
	\caption{Результат комадны show interfaces trunk}
\end{image}

\begin{image}
	\includegrph{Screenshot from 2024-03-18 17-05-28}
	\caption{Результат комадны show running-config}
\end{image}

\textbf{Какие настройки используются для режима протокола spanning-tree
	на коммутаторах Cisco?}

Режим связующего дерева по умолчанию --- PVST+.

\textbf{Проверьте подключение между компьютерами PC-A и PC-C.
	Удалось ли получить ответ на эхо-запрос?}

Да

Если эхо-запрос выполнить не удалось, следует выполнять отладку до тех пор,
пока проблема не будет решена.

\section{Настройка корневого моста и проверка сходимости PVST+}
\subsection{Определите текущий корневой мост}

\textbf{С помощью какой команды пользователи определяют состояние
протокола spanning-tree коммутатора Cisco Catalyst для всех сетей VLAN?}

\begin{verbatim}
show spanning-tree
\end{verbatim}

\begin{image}
	\includegrph{Screenshot from 2024-03-18 17-13-06}
	\caption{Результат комадны show spanning-tree на S1}
\end{image}

\begin{image}
	\includegrph{Screenshot from 2024-03-18 17-13-34}
	\caption{Результат комадны show spanning-tree на S2}
\end{image}

\begin{image}
	\includegrph{Screenshot from 2024-03-18 17-13-55}
	\caption{Результат комадны show spanning-tree на S3}
\end{image}

\textbf{Каков приоритет моста коммутатора S1\_ФАМИЛИЯ для сети VLAN 1?}

32769

\textbf{Каков приоритет моста коммутатора S2 для сети VLAN 1?}

32769

\textbf{Каков приоритет моста коммутатора S3 для сети VLAN 1?}

32769

\textbf{Какой коммутатор является корневым мостом?}

S3

\textbf{Почему этот коммутатор выбран в качестве корневого моста?}

По умолчанию связующее дерево выбирает корневой мост
на основе наименьшего MAC-адреса.

\subsection{Настройте основной и вспомогательный корневые мосты
	для всех существующих сетей VLAN}

Настройте коммутатор S2 в качестве основного корневого моста для всех существующих сетей VLAN.

\begin{verbatim}
spanning-tree vlan 1,10,99 root primary
\end{verbatim}

Настройте коммутатор S1\_ФАМИЛИЯ в качестве
вспомогательного корневого моста для всех существующих сетей VLAN.
Запишите команду в строке ниже.

\begin{verbatim}
spanning-tree vlan 1,10,99 root secondary
\end{verbatim}

\begin{image}
	\includegrph{Screenshot from 2024-03-18 19-20-31}
	\caption{Результат комадны show spanning-tree}
\end{image}

Используйте команду \texttt{show spanning-tree} для ответа
на следующие вопросы:

\textbf{Какой приоритет моста используется
для коммутатора S1\_ФАМИЛИЯ в сети VLAN 1?}

24577

\textbf{Какой приоритет моста используется для коммутатора S2 в сети VLAN 1?}

24577

\textbf{Какой интерфейс в сети находится в состоянии блокировки?}

Интерфейс F0/3 на коммутаторе S3

\subsection{Измените топологию 2-го уровня и проверьте сходимость}

Выполните команду\texttt{debug spanning-tree events} 
в привилегированном режиме на коммутаторе S3.

\textbf{НЕ РАБОТАЕТ}

Через какие состояния портов проходит каждая сеть VLAN
на интерфейсе F0/3 в процессе схождения сети?

Используя временную метку из первого и последнего сообщений отладки STP,
рассчитайте время (округляя до секунды),
которое потребовалось для схождения сети.
Рекомендация. Формат временной метки сообщений отладки: чч.мм.сс.мс

\section{Настройка Rapid PVST+, PortFast, BPDU Guard и проверка сходимости}

\subsection{Настройте Rapid PVST+}

Настройте S1 для использования Rapid PVST+.

\begin{verbatim}
spanning-tree mode rapid-pvst
\end{verbatim}

Настройте коммутаторы S2 и S3 для Rapid PVST+.

\begin{verbatim}
spanning-tree mode rapid-pvst
\end{verbatim}

Проверьте конфигурации с помощью команды
\texttt{show running-config | include spanning-tree mode}.\\

\textit{spanning-tree mode rapid-pvst}

\subsection{Настройте PortFast и BPDU Guard на портах доступа}

Настройте F0/6 на S1\_ФАМИЛИЯ с помощью функции PortFast.

\begin{verbatim}
interface f0/6
spanning-tree portfast
\end{verbatim}

Настройте F0/6 на S1\_ФАМИЛИЯ с помощью функции BPDU Guard.

\begin{verbatim}
interface f0/6
spanning-tree bpduguard enable
\end{verbatim}

Глобально настройте все нетранковые порты
на коммутаторе S3 с помощью функцииPortFast.

\begin{verbatim}
spanning-tree portfast default
\end{verbatim}

Глобально настройте все нетранковые порты
на коммутаторе S3 с помощью функции BPDU.

\begin{verbatim}
spanning-tree portfast bpduguard default
\end{verbatim}

\subsection{Проверьте сходимость Rapid PVST+}

Выполните команду\texttt{debug spanning-tree events} 
в привилегированном режиме на коммутаторе S3.

\textbf{НЕ РАБОТАЕТ}

Измените топологию, отключив интерфейс F0/1 на коммутаторе S3.
Используя временную метку из первого и последнего сообщений отладки RSTP,
рассчитайте время, которое потребовалось для схождения сети.

\begin{image}
	\includegrph{Screenshot from 2024-03-18 17-39-15}
	\caption{Отправки эхо запросов с копьютера на компьютер}
\end{image}

\section{Ответы на контрольные вопросы}

\subsection{Опишите преимущества беспроводной связи.
	Кратко охарактеризуйте основные типы беспроводной связи}

WLAN делает возможной мобильность в домашней и деловой среде. 
Беспроводная инфраструктура может адаптироваться к быстро меняющимся 
потребностям и технологиям.

4 типа беспроводной связи:

\begin{itemize}
	\item Беспроводные персональные сети (WPAN) --- используют 
		маломощные передатчики для сетей ближнего действия, обычно от 20 
		до 30 футов (от 6 до 9 метров). Устройства на базе Bluetooth и ZigBee 
		обычно используются в WPAN. WPAN основаны на стандарте 802.15 
		и частоте 2,4 ГГц.
	\item Беспроводные локальные сети (WLAN) --- использует 
		передатчики для покрытия сети среднего размера, обычно до 100 
		метров. Беспроводные локальные сети подходят для использования 
		дома, в офисе и даже в кампусе. Сети WLAN основаны на стандарте 
		802.11 и частоте 2,4 ГГц или 5 ГГц.
	\item Беспроводные MAN (WMAN) --- использует передатчики для 
		предоставления услуг беспроводной связи в большой географической 
		зоне. WMAN подходят для обеспечения беспроводного доступа к 
		столичному городу или конкретному району. WMAN используют 
		определенные лицензированные частоты.
	\item Беспроводные глобальные сети (WWANs) --- использует 
		передатчики для обеспечения покрытия в обширной географической 
		области. WWAN подходят для национальных и глобальных 
		коммуникаций. WWAN также используют определенные 
		лицензированные частоты.
\end{itemize}

\subsection{В каких случаях используются технологии Bluetooth и 
спутниковая широкополосная связь? Для чего была 
разработана технология MIMO?}

Bluetooth --- стандарт IEEE 802.15 WPAN, использующий процесс 
сопряжения устройств для связи на расстоянии до 100 м. Его можно найти в 
устройствах «умный дом», аудиоподключениях, автомобилях и других 
устройствах, требующих подключения на небольшом расстоянии.\par
Спутниковая широкополосная связь --- обеспечивает сетевой доступ 
к удаленным объектам посредством использования направленной 
спутниковой антенны, которая ориентирована на определенный 
геостационарный спутник на орбите Земли. Как правило, эта технология 
отличается более высокой стоимостью и к тому же требует обеспечения 
прямой видимости. Обычно это дороже и требует четкого обзора. Как 
правило, он используется сельскими домовладельцами и предприятиями, где 
нет кабеля и DSL.\par
Некоторые из более новых стандартв, которые передают и принимают 
на более высоких скоростях, требуют, чтобы точки доступа (AP) и 
беспроводные клиенты имели несколько антенн, использующих технологию 
множественного входа и множественного выхода (MIMO). MIMO 
использует несколько антенн в качестве передатчика и приемника для 
улучшения характеристик связи. Может поддерживаться до четырех антенн.

\subsection{Какие роли может выполнять домашний беспроводной 
маршрутизатор? Для чего нужны беспроводные точки доступа?}

Беспроводной маршрутизатор служит как:

\begin{itemize}
	\item Точка доступа --- обеспечивает беспроводной доступ 
		802.11a/b/g/n/ac.
	\item Коммутатор --- обеспечивает четырехпортовый полнодуплексный 
		Ethernet-коммутатор 10/100/1000 для подключения проводных 
		устройств
	\item Маршрутизатор --- обеспечивает шлюз по умолчанию для 
		подключения к другим сетевым инфраструктурам, таким как 
		Интернет.
\end{itemize}

Точки беспроводного доступа необходимы для предоставления 
выделенного беспроводного доступа для пользовательских устройств.

\subsection{Назовите и охарактеризуйте категории точек доступа. 
Перечислите и опишите варианты антенн для беспроводных 
устройств.}

Точки доступа могут быть автономными и управляемыми 
контроллером.\par
Автономные точки доступа --- это автономные устройства, 
настроенные с использованием интерфейса командной строки или 
графического интерфейса. Автономные AP полезны в ситуациях, когда в 
организации требуется только пара AP. \par
AP на основе контроллера --- эти устройства не требуют начальной 
настройки и часто называются облегченными точками доступа (LAP). LAP 
используют протокол облегченной точки доступа (LWAPP) для связи с 
контроллером WLAN (WLC). Точки доступа, управляемые контроллером, 
рекомендуется использовать в случаях, когда в сети требуется много точек 
доступа. Поскольку больше AP добавлено, каждый AP автоматически 
настраивается и управляется WLC.\par
Варианты антенн для беспроводных устройств:

\begin{itemize}
	\item Всенаправленные антенны обеспечивают 360-градусный охват 
		и идеальны в домах, открытых офисных помещениях, конференц-
		залах и вне помещений.
	\item Направленные антенны фокусируют радиосигнал в заданном 
		направлении. Это усиливает сигнал к и от точки доступа в 
		направлении, на которое указывает антенна. Это обеспечивает более 
		сильный уровень сигнала в одном направлении и пониженный 
		уровень сигнала во всех других направлениях.
	\item Несколько выходов (MIMO антенна) использует несколько 
		антенн для увеличения доступной полосы пропускания для 
		беспроводных сетей IEEE 802.11n/ac/ax. Для увеличения пропускной 
		способности можно использовать до восьми передающих и 
		принимающих антенн.
\end{itemize}

\subsection{Дайте характеристику режимам топологий беспроводной сети. В 
чем заключается разница между BSS и ESS?}

Режимы топологии беспроводной сети:
Ad hoc Режим используется, когда два устройства подключаются 
по беспроводной сети в одноранговой (P2P) манере без 
использования точек доступа или беспроводных маршрутизаторов.
Инфраструктурный режим --- это когда беспроводные клиенты 
соединяются через беспроводной маршрутизатор или точку доступа, 
например, в WLAN. Точки доступа подключаются к сетевой 
инфраструктуре с помощью проводной системы распределения, 
например, Ethernet.\par
Режим модема (использование мобильного телефона в качестве 
точки доступа в Интернет) --- смартфон или планшет с сотовым 
доступом к данным включен для создания личной точки доступа. 
Точка беспроводного доступа является временным краткосрочным 
решением, благодаря которому смартфон может обеспечивать 
сервисы беспроводной связи Wi-Fi-маршрутизатора. Другие 
устройства могут подключаться и проверять подлинность со 
смартфоном для использования интернет-соединения.\par
Режим инфраструктуры определяет два строительных блока топологии: 
базовый набор услуг (BSS) и расширенный набор услуг (ESS).
Базовый набор сервисов (BSS) состоит из одной точки доступа, 
которая взаимодействует со всеми связанными беспроводными клиентами.
Расширенный набор сервисов (ESS) – это объединение двух или 
более BSS, связанных между собой проводным DS. Когда одна BSS 
обеспечивает недостаточное покрытие, две или более BSS могут быть 
объединены через общую систему распределения (DS) в ESS.

\subsection{Опишите принцип работы беспроводного клиента при 
использовании метода CSMA/CA. В чем разница между 
пассивным и активным обнаружением точек доступа?}

WLAN используют многостанционный доступ с контролем несущей и 
предотвращение коллизий (CSMA/CA) в качестве метода определения того, 
как и когда отправлять данные в сеть. Беспроводной клиент делает 
следующее:

\begin{enumerate}
	\item Прослушивает канал, чтобы увидеть, не занят ли он, что 
		означает, что он чувствует, что в данный момент на канале нет 
		другого трафика. Канал также называется несущей.
	\item Отправляет сообщение о готовности к отправке (RTS) в точку 
		доступа для запроса выделенного доступа к сети.
	\item Получает сообщение очистки для отправки (CTS) от точки 
		доступа, предоставляющей доступ к отправке.
	\item Если беспроводной клиент не получает сообщение CTS, он 
		ожидает случайное количество времени, прежде чем перезапустить 
		процесс.
	\item После того, как он получает CTS, он передает данные.
	\item Все передачи подтверждены. Если беспроводной клиент не 
		получает подтверждение, он предполагает, что произошло 
		столкновение, и перезапускает процесс.
\end{enumerate}

В пассивном режиме AP открыто объявляет о своей услуге, 
периодически отправляя кадры широковещательного маяка, содержащие 
SSID, поддерживаемые стандарты и настройки безопасности. Основная цель 
маяка состоит в том, чтобы позволить беспроводным клиентам узнавать, 
какие сети и точки доступа доступны в данной области. Это позволяет 
беспроводным клиентам выбирать, какую сеть и точку доступа использовать.
В активном режиме беспроводные клиенты должны знать имя SSID. 
Беспроводной клиент инициирует процесс путем отправки по 
широковещательной рассылке кадра запроса поиска на несколько каналов. 
Запрос поиска содержит имя SSID и сведения о поддерживаемых стандартах. 
AP, настроенные с SSID, отправят ответ на запрос, который включает SSID, 
поддерживаемые стандарты и параметры безопасности. Активный режим 
может понадобиться в том случае, если для беспроводного маршрутизатора 
или точки доступа настроен запрет широковещательной рассылки кадров 
сигнала.

\subsection{Опишите назначение протокола CAPWAP. Назовите основные 
рекомендации по установке точек доступа.}

CAPWAP --- это стандартный протокол IEEE, который позволяет WLC 
управлять несколькими точками доступа и WLAN. CAPWAP также отвечает 
за инкапсуляцию и пересылку клиентского трафика WLAN между AP и 
WLC.\par
При планировании местоположения точек доступа важна 
приблизительная круговая зона покрытия, но есть несколько 
дополнительных рекомендаций:

\begin{itemize}
	\item Если точки доступа должны использовать существующие кабельные 
		системы, или присутствуют расположения, где нельзя разместить 
		точки доступа, следует отметить эти места на карте;
	\item Обратите внимание на все потенциальные источники помех, которые 
		могут включать микроволновые печи, беспроводные видеокамеры, 
		флуоресцентные лампы, детекторы движения или любое другое 
		устройство, использующее диапазон 2,4 ГГц.
	\item Точки доступа следует размещать выше физических препятствий;
	\item По возможности размещать точки доступа вертикально рядом с 
		потолком в центре каждой зоны;
	\item Размещать AP в тех местах, где будут находиться пользователи.
\end{itemize}

\subsection{Опишите основные угрозы при использовании беспроводных 
точек доступа. Какие бывают типы аутентификации в 
беспроводной связи?}

Беспроводные сети особенно подвержены следующим угрозам:
Перехват данных --- беспроводные данные должны быть 
зашифрованы для предотвращения их перехвата.
Беспроводные нарушители --- неавторизованные пользователи, 
пытающиеся получить доступ к сетевым ресурсам, могут быть 
предотвращены с помощью эффективных методов аутентификации.
Атаки типа «отказ в обслуживании» (DoS) - Доступ к услугам 
WLAN может быть скомпрометирован случайно или злонамеренно. 
Существуют различные решения в зависимости от источника атаки 
DoS.\par
Вредоносные точки доступа --- Несанкционированные точки 
доступа, установленные благонамеренными пользователями или в 
злонамеренных целях, можно обнаружить с помощью программного 
обеспечения для управления.\par
Типы аутентификации в беспроводной связи:

\begin{itemize}
	\item Открытая аутентификация. Рабочая станция делает запрос 
		аутентификации, в котором присутствует только MAC-адрес 
		клиента. Точка доступа отвечает либо отказом, либо 
		подтверждением аутентификации.
	\item Аутентификация с общим ключом. Необходимо настроить 
		статический ключ шифрования алгоритма WEP.
	\item Wi-Fi Protected Access (WPA). Существуют два варианта 
		аутентификации: с помощью RADIUS сервера (WPA-Enterprise) и с 
		помощью предустановленного ключа (WPA-PSK).
	\item Wi-Fi Protected Access2 (WPA2, 802.11i). В качестве основного 
		шифра был выбран стойкий блочный шифр AES.
	\item Cisco Centralized Key Managment (CCKM). Вариант 
		аутентификации от фирмы CISCO. Поддерживает роуминг между 
		точками доступа.
\end{itemize}

\subsection{Для чего используется протокол RADIUS? Опишите методы 
аутентификации домашнего пользователя.}

Протокол RADIUS используетсядля аутентификации, авторизации и 
учёта пользователей в сетях. Он предоставляет стандартизированный метод 
для проверки подлинности и предоставления доступа к сетевым ресурсам.\par
Домашние маршрутизаторы обычно имеют два варианта 
аутентификации: WPA и WPA2.

