\section{Построение сети и проверка связи}

\subsection{Произведите базовую настройку маршрутизаторов}

\begin{verbatim}
no ip domain lookup
hostname R2_BONDAR

/* R1 */
interface vlan 1
no shutdown
interface g 0/1
no shutdown
ip address 192.168.1.1 255.255.255.0
interface se 0/0/0
no shutdown
ip address 10.1.1.1 255.255.255.252
clock rate 128000
/* R1 */

/* R2 */
interface se 0/0/0
no shutdown
ip address 10.1.1.2 255.255.255.252
clock rate 128000
interface se 0/0/1
no shutdown
ip address 10.2.2.2 255.255.255.252
clock rate 128000
interface lo1
no shutdown
ip address 209.165.203.225 255.255.255.224
/* R2 */

/* R3 */
interface vlan 1
no shutdown
interface g 0/1
no shutdown
ip address 192.168.1.3 255.255.255.0
interface se 0/0/1
no shutdown
ip address 10.2.2.1 255.255.255.252
clock rate 128000
/* R3 */

line console 0
password cisco
login
logging synchronous
line vty 0 4
password cisco
login
logging synchronous
exit
enable secret class
end
copy running-config startup-config
\end{verbatim}

\subsection{Настройте базовые параметры каждого коммутатора}

\begin{verbatim}
no ip domain lookup
hostname S1
enable secret class

/* S1 */
ip default-gateway 192.168.1.1
interface vlan 1
no shutdown
ip address 192.168.1.11 255.255.255.0
/* S1 */

/* S3 */
ip default-gateway 192.168.1.3
interface vlan 1
no shutdown
ip address 192.168.1.13 255.255.255.0
/* S3 */

line console 0
password cisco
login
logging synchronous
line vty 0 4
password cisco
login
logging synchronous
exit
end
copy running-config startup-config
\end{verbatim}

\subsection{Проверьте подключение между PC-A и PC-C}

Отправьте ping-запрос с компьютера PC-A на компьютер PC-C.

\begin{image}
	\includegrph{Screenshot from 2024-03-25 21-40-09}
	\caption{Выполнение эхо-запросов}
\end{image}

\textbf{Удалось липолучить ответ?}

Да

\subsection{Настройте маршрутизацию}

Настройте RIP версии 2 на всех маршрутизаторах.
Добавьте в процесс RIP все сети, кроме 209.165.203.224/27.

\begin{verbatim}
config t
router rip
version 2
no auto-summary
do sh ip route connected
network 192.168.1.0
network 10.0.0.0
\end{verbatim}

Настройте маршрут по умолчанию на маршрутизаторе R2\_ФАМИЛИЯ
с использованием Lo1 в качестве интерфейса выхода в сеть 209.165.203.224/27.

\begin{verbatim}
ip route 0.0.0.0 0.0.0.0 loopback 1
\end{verbatim}

На маршрутизаторе R2\_ФАМИЛИЯ используйте следующие команды
для перераспределения маршрутапо умолчанию в процесс RIP.

\begin{verbatim}
router rip
default-information originate
\end{verbatim}

\subsection{Проверьте подключение}

Необходимо получить ответ на ping-запросы с компьютера PC-A
от каждого интерфейса на маршрутизаторах R1, R2\_ФАМИЛИЯ и R3,
а также от компьютера PC-C.


\textbf{Удалось ли получить всеответы?}

Необходимо получить ответ на ping-запросы с компьютера PC-C
от каждого интерфейса на маршрутизаторах R1, R2\_ФАМИЛИЯ и R3,
а также от компьютера PC-A.

\textbf{Удалось ли получить все ответы?}

Да

\begin{image}
	\includegrph{Screenshot from 2024-03-25 22-29-53}
	\caption{Выполнение эхо-запросов}
\end{image}

\begin{image}
	\includegrph{Screenshot from 2024-03-25 22-41-03}
	\caption{Выполнение эхо-запросов}
\end{image}

\section{Настройка обеспечения избыточности на первом хопе
с помощью HSRP}











