192.168.0.0/26

11000000.10101000.00000000.00|000000

192.168.0.0/19

11000000.10101000.00000000.00000|000

VLAN 10\\
PCA 192.168.0.11

VLAN 14\\
PCB 192.168.0.19

PCC 192.168.0.29

VLAN 14\\
S0 192.168.0.20

R0 192.168.0.49\\
R2 192.168.0.54 (55?)\\

R1 192.168.0.57\\
R2 192.168.0.62 (63?)\\


\section{Настройка R2}

\begin{verbatim}
// R2
hostname Zachet
no ip domain lookup
line console 0
password console
login
line vty 0 4
password Telnet
login
enable secret privilege
// зашифруте все пароли
service password-encryption
banner motd #Warning#
\end{verbatim}

\section{Настройка S0}

\begin{verbatim}
// S0
interface range f0/3-9, f0/12-24, g0/1-2
shutdown

interface range f0/1-2, f0/10-11
switchport mode access
no shutdown  // ?

interface vlan 14
no shutdown
ip address 192.168.0.20 255.255.255.248

interface range f0/1-2, f0/10-11
switchport mode access
switchport port-security
switchport port-security maximum 5
switchport port-security violation protect
\end{verbatim}

\section{Trunk}

\begin{verbatim}
// S0, R2
interface range f0/1-2, f0/10-11
switchport mode trunk
switchport trunk native vlan 80
switchport trunk allowed vlan 10,14
\end{verbatim}

\section{Routet-on-a-Stick}

https://itexamanswers.net/4-2-7-packet-tracer-configure-router-on-a-stick-inter-vlan-routing-instructions-answer.html

(?)

\section{Статическая маршрутизация}

(?)

\section{HSRPv2}

\begin{verbatim}
// R2
config t
router rip
version 2
no auto-summary
do sh ip route connected
network 192.168.0.0
network 10.0.0.0
ip route 0.0.0.0 0.0.0.0 loopback 1
router rip
default-information originate
\end{verbatim}

\begin{verbatim}
// R0
interface g0/0
standby version 2
standby 1 ip 192.168.0.63 (?)
standby 1 priority 150
standby 1 preempt

// R1
interface g0/0
standby version 2
standby 1 ip 192.168.1.254
\end{verbatim}

