https://itexamanswers.net/7-4-2-lab-implement-dhcpv4-answers.html

\section{Выполнение практической работы}

\section{Настройка основных параметров коммутатора}

\subsection{Настройте базовые параметры каждого коммутатора}

\begin{verbatim}
enable
config t
no ip domain lookup
hostname S1\_BONDAR
service password-encryption
banner motd # You must be authorizeded! #
enable secret class
line console 0
password cisco
login
line vty 0 15
password cisco
login
logging synchronous
interface range f0/5, f0/7-24, g0/1-2  // S1
interface range f0/5-17, f0/19-24, g0/1-2  // S2
interface range f0/5-17, f0/19-24, g0/1-2  // S3
shutdown
vlan 99
name Management
vlan 13
name Staff
interface f0/6  // S1
interface f0/18  // S2
interface f0/18  // S3
switchport mode access
switchport access vlan 13
no shutdown
interface vlan 99
ip address 192.168.99.11 255.255.255.0
no shutdown
end
copy running-config startup-config
\end{verbatim}

\section{Настройка протокола PAgP}
\subsection{Настройте PAgP на S1\_ФАМИЛИЯ и S3}

\begin{verbatim}
config t
interface range f0/3-4
channel-group 1 mode desirable  // S1
channel-group 1 mode auto  // S3
no shutdown
\end{verbatim}

\subsection{Проверьте конфигурации на портах}

\begin{verbatim}
show run interface f0/3
show interfaces f0/3 switchport
\end{verbatim}

\subsection{Убедитесь, что порты объединены}

\begin{verbatim}
show etherchannel summary
\end{verbatim}

\textbf{Что означают флаги «SU» и «P» в сводных данных по Ethernet?}\\
Флаг P указывает, что порты объединены в порт-канал.\\
Флаг S указывает, что порт-канал является EtherChannel уровня 2.\\
Флаг U указывает, что EtherChannel используется.\\

\subsection{Настройте транковые порты}

\textbf{S1 S3}

\begin{verbatim}
config t
interface port-channel 1
switchport mode trunk
switchport trunk native vlan 99
\end{verbatim}

\subsection{Убедитесь в том, что порты настроены в качестве транковых}

Выполните команды \verb|show running-startup| идентификатор-интерфейса
на S1\_ФАМИЛИЯ и S3.
Какие команды включены в список для интерфейсов F0/3 и F0/4
на обоих коммутаторах?

\begin{verbatim}
switchport trunk native vlan 99
switchport mode trunk
\end{verbatim}

Сравните результаты с текущей конфигурацией для интерфейса Po1.
Запишите наблюдения.

Команды, относящиеся к конфигурации таранков, одинаковы.
Когда команды таранков были применены к EtherChannel,
команды также повлияли на отдельные ссылки в пакете.

\begin{verbatim}
!
interface Port-channel1
 switchport trunk native vlan 99
 switchport mode trunk
!
interface FastEthernet0/1
!
interface FastEthernet0/2
!
interface FastEthernet0/3
 switchport trunk native vlan 99
 switchport mode trunk
 channel-group 1 mode desirable
!
\end{verbatim}

Выполните команды \verb|show interfaces trunk| и \verb|show spanning-tree|
на S1\_ФАМИЛИЯ и S3.
Какой транковый порт включен в список?
Какая используется сеть native VLAN?
Какой вывод можно сделать на основе выходных данных?

Указанный таранковый порт - Po1.
Собственная VLAN - 99.
После объединения ссылок в некоторых командах отображается только
агрегированный интерфейс.

Какие значения стоимости и приоритета порта
для агрегированного канала отображены
в выходных данных команды show spanning-tree?

Стоимость порта для Po1 равна 12, а приоритет порта равен 128.

\begin{verbatim}
S1_BONDAR#show interfaces trunk
Port        Mode         Encapsulation  Status        Native vlan
Po1         on           802.1q         trunking      99
Port        Vlans allowed on trunk
Po1         1-1005
Port        Vlans allowed and active in management domain
Po1         1,13,99
Port        Vlans in spanning tree forwarding state and not pruned
Po1         1,13,99
S1_BONDAR#
\end{verbatim}

\begin{verbatim}
S1_BONDAR#show spanning-tree 
VLAN0001
  Spanning tree enabled protocol ieee
  Root ID    Priority    32769
             Address     0002.176D.07E9
             Cost        12
             Port        27(Port-channel1)
             Hello Time  2 sec  Max Age 20 sec  Forward Delay 15 sec
  Bridge ID  Priority    32769  (priority 32768 sys-id-ext 1)
             Address     0060.7031.8697
             Hello Time  2 sec  Max Age 20 sec  Forward Delay 15 sec
             Aging Time  20
Interface        Role Sts Cost      Prio.Nbr Type
---------------- ---- --- --------- -------- --------------------------------
Fa0/1            Desg FWD 19        128.1    P2p
Fa0/2            Desg FWD 19        128.2    P2p
Po1              Root FWD 12        128.27   Shr
VLAN0013
  Spanning tree enabled protocol ieee
  Root ID    Priority    32781
             Address     0002.176D.07E9
             Cost        12
             Port        27(Port-channel1)
             Hello Time  2 sec  Max Age 20 sec  Forward Delay 15 sec
  Bridge ID  Priority    32781  (priority 32768 sys-id-ext 13)
             Address     0060.7031.8697
             Hello Time  2 sec  Max Age 20 sec  Forward Delay 15 sec
             Aging Time  20
Interface        Role Sts Cost      Prio.Nbr Type
---------------- ---- --- --------- -------- --------------------------------
Fa0/6            Desg FWD 19        128.6    P2p
Po1              Root FWD 12        128.27   Shr
VLAN0099
  Spanning tree enabled protocol ieee
  Root ID    Priority    32867
             Address     0002.176D.07E9
             Cost        12
             Port        27(Port-channel1)
             Hello Time  2 sec  Max Age 20 sec  Forward Delay 15 sec
  Bridge ID  Priority    32867  (priority 32768 sys-id-ext 99)
             Address     0060.7031.8697
             Hello Time  2 sec  Max Age 20 sec  Forward Delay 15 sec
             Aging Time  20
Interface        Role Sts Cost      Prio.Nbr Type
---------------- ---- --- --------- -------- --------------------------------
Po1              Root FWD 12        128.27   Shr
S1_BONDAR#
\end{verbatim}

\begin{verbatim}
S3#show spanning-tree 
VLAN0001
  Spanning tree enabled protocol ieee
  Root ID    Priority    32769
             Address     0002.176D.07E9
             This bridge is the root
             Hello Time  2 sec  Max Age 20 sec  Forward Delay 15 sec
  Bridge ID  Priority    32769  (priority 32768 sys-id-ext 1)
             Address     0002.176D.07E9
             Hello Time  2 sec  Max Age 20 sec  Forward Delay 15 sec
             Aging Time  20
Interface        Role Sts Cost      Prio.Nbr Type
---------------- ---- --- --------- -------- --------------------------------
Fa0/1            Desg FWD 19        128.1    P2p
Fa0/2            Desg FWD 19        128.2    P2p
Po1              Desg FWD 12        128.27   Shr
VLAN0013
  Spanning tree enabled protocol ieee
  Root ID    Priority    32781
             Address     0002.176D.07E9
             This bridge is the root
             Hello Time  2 sec  Max Age 20 sec  Forward Delay 15 sec

  Bridge ID  Priority    32781  (priority 32768 sys-id-ext 13)
             Address     0002.176D.07E9
             Hello Time  2 sec  Max Age 20 sec  Forward Delay 15 sec
             Aging Time  20
Interface        Role Sts Cost      Prio.Nbr Type
---------------- ---- --- --------- -------- --------------------------------
Fa0/18           Desg FWD 19        128.18   P2p
Po1              Desg FWD 12        128.27   Shr
VLAN0099
  Spanning tree enabled protocol ieee
  Root ID    Priority    32867
             Address     0002.176D.07E9
             This bridge is the root
             Hello Time  2 sec  Max Age 20 sec  Forward Delay 15 sec
  Bridge ID  Priority    32867  (priority 32768 sys-id-ext 99)
             Address     0002.176D.07E9
             Hello Time  2 sec  Max Age 20 sec  Forward Delay 15 sec
             Aging Time  20
Interface        Role Sts Cost      Prio.Nbr Type
---------------- ---- --- --------- -------- --------------------------------
Po1              Desg FWD 12        128.27   Shr
S3#
\end{verbatim}

\section{Настройка протокола LACP}

Протокол LACP является открытым протоколом агрегирования каналов,
разработанным на базе стандарта IEEE.
В части 3 необходимо выполнить настройку канала между S1\_ФАМИЛИЯ и S2
и канала между S2 и S3 с помощью протокола LACP.
Кроме того, отдельные каналы необходимо настроить в качестве транковых
и указать native vlan, прежде чем они будут объединены в каналы EtherChannel.

\subsection{Настройте LACP между S1\_ФАМИЛИЯ и S2}

Настройте канал между S2 и S3 как Po3,
используя LACP как протокол агрегирования каналов.
Канал на S1\_ФАМЛИЛИЯ должен быть в режиме active,
а канал на S2 – в режиме passive.

\begin{verbatim}
interface range f0/1-2
switchport mode trunk
switchport trunk native vlan 99
channel-group 2 mode active  // S1
channel-group 2 mode passive  // S2
no shutdown
\end{verbatim}

\subsection{Убедитесь, что порты объединены}

Какой протокол использует Po2 для агрегирования каналов?
Какие порты агрегируются для образования Po2?
Запишите команду, используемую для проверки.

Po2 использует LACP. F0/1 и F0/2 агрегируются с образованием Po2.

\begin{verbatim}
S1_BONDAR#show etherchannel summary
Flags:  D - down        P - in port-channel
        I - stand-alone s - suspended
        H - Hot-standby (LACP only)
        R - Layer3      S - Layer2
        U - in use      f - failed to allocate aggregator
        u - unsuitable for bundling
        w - waiting to be aggregated
        d - default port
Number of channel-groups in use: 2
Number of aggregators:           2
Group  Port-channel  Protocol    Ports
------+-------------+-----------+----------------------------------------------
1      Po1(SU)           PAgP   Fa0/3(P) Fa0/4(D) 
2      Po2(SU)           LACP   Fa0/1(P) Fa0/2(P) 
S1_BONDAR#
\end{verbatim}

\begin{verbatim}
S2#show etherchannel summary
Flags:  D - down        P - in port-channel
        I - stand-alone s - suspended
        H - Hot-standby (LACP only)
        R - Layer3      S - Layer2
        U - in use      f - failed to allocate aggregator
        u - unsuitable for bundling
        w - waiting to be aggregated
        d - default port
Number of channel-groups in use: 1
Number of aggregators:           1
Group  Port-channel  Protocol    Ports
------+-------------+-----------+----------------------------------------------
2      Po2(SU)           LACP   Fa0/1(P) Fa0/2(P) 
S2#
\end{verbatim}

\subsection{Настройте LACP между S2 и S3}

Аналогично настройте канал между S2 и S3 как Po3,
используя LACP как протокол агрегирования каналов.

\begin{verbatim}
interface range f0/3-4  // S2
interface range f0/1-2  // S3
switchport mode trunk
switchport trunk native vlan 99
channel-group 3 mode active  // S2
channel-group 3 mode passive  // S3
no shutdown
\end{verbatim}

b. Убедитесь в том, что канал EtherChannel образован.

\begin{verbatim}
S2#show etherchannel summary
Flags:  D - down        P - in port-channel
        I - stand-alone s - suspended
        H - Hot-standby (LACP only)
        R - Layer3      S - Layer2
        U - in use      f - failed to allocate aggregator
        u - unsuitable for bundling
        w - waiting to be aggregated
        d - default port
Number of channel-groups in use: 2
Number of aggregators:           2

Group  Port-channel  Protocol    Ports
------+-------------+-----------+----------------------------------------------

2      Po2(SU)           LACP   Fa0/1(P) Fa0/2(P) 
3      Po3(SU)           LACP   Fa0/3(P) Fa0/4(P) 
S2#
\end{verbatim}

\begin{verbatim}
S3#show etherchannel summary
Flags:  D - down        P - in port-channel
        I - stand-alone s - suspended
        H - Hot-standby (LACP only)
        R - Layer3      S - Layer2
        U - in use      f - failed to allocate aggregator
        u - unsuitable for bundling
        w - waiting to be aggregated
        d - default port
Number of channel-groups in use: 2
Number of aggregators:           2
Group  Port-channel  Protocol    Ports
------+-------------+-----------+----------------------------------------------
1      Po1(SU)           PAgP   Fa0/3(P) Fa0/4(D) 
3      Po3(SU)           LACP   Fa0/1(P) Fa0/2(P) 
S3#
\end{verbatim}

\subsection{Проверьте наличие сквозного соединения}

Убедитесь в том, что все устройства могут передавать друг другу эхо-запросы
в пределах одной сети VLAN.
Если нет, устраните неполадки,
чтобы установить связь между конечными устройствами.

192.168.99.11
192.168.99.12
192.168.99.13

192.168.13.1
192.168.13.2
192.168.13.3

Да, между компьютерами. Да, между коммутаторами.
Между компьютером и коммутатором не пингуется.

