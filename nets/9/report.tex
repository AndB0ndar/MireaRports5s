\section{Выполнение практической работы}

\section{Настройка устройств и проверка подключения}

\subsection{Настройте маршрутизатор}

\begin{verbatim}
enable
config terminal
hostname R1_BONDAR
no ip domain lookup
security passwords min-length 10
enable secret cisco12345
line console 0
password ciscoconpass
exec-timeout 5 0
login
logging synchronous
exit
line vty 0 4
password ciscovtypass
exec-timeout 5 0
login
logging synchronous
exit
service password-encryption
banner motd # You must be authorizeded! #
interface g0/0/0
ip address 192.168.0.1 255.255.255.0
no shutdown
interface g0/0/1
ip address 192.168.1.1 255.255.255.0
no shutdown
end
clock set 20:00:00 18 Mar 2024
copy running-config startup-config
\end{verbatim}

\textbf{Укажите способы усиления защиты паролей,
кроме установки минимальной длины.}

Используйте в паролях заглавные буквы, цифры и специальные символы.

\textbf{Что произойдет, если перезагрузить маршрутизатор до того,
как будет выполнена команда \texttt{copy running-config startup-config}?}

Содержимое запущенной конфигурации будет удалено.
В этой лабораторной работе маршрутизатор не будет иметь конфигурации запуска.
После перезагрузки пользователя спросят,
хочет ли он войти в диалоговое окно начальной настройки.

\subsection{Проверьте подключение к сети}

Из командной строки компьютера PC-B отправьте эхо-запрос на компьютер PC-A.

\begin{image}
	\includegrph{Screenshot from 2024-03-18 19-55-11}
	\caption{Эхо-запрос от PC-B к PC-A}
\end{image}

\textbf{Успешно ли выполнена проверка связи?}

Да

\textbf{Какой тип удаленного доступа будет использоваться для получения
доступа к маршрутизатору R1\_ФАМИЛИЯ после завершения этого набора команд?}

Telnet

Подключитесь к маршрутизатору R1\_ФАМИЛИЯ
от компьютера PC-A с помощью службы Telnet.

\begin{image}
	\includegrph{Screenshot from 2024-03-18 19-55-11}
	\caption{Подключение по Telnet}
\end{image}

\textbf{Почему использование протокола Telnet считается угрозой безопасности?}

Сеанс Telnet можно увидеть в виде открытого текста. Он не зашифрован.
Пароли можно легко увидеть с помощью анализатора пакетов.

\subsection{Настройте маршрутизатор для доступа по протоколу SSH}
Активируйте подключения SSH и создайте пользователя
(username --- ваша фамилия на английском языке,
доменное имя маршрутизатора --- CCNA-lab.com)
в локальной базе данных маршрутизатора. Длина ключа шифрования --- 1024 бит.
Не забудьте записать пароль, чтобы не забыть его при повторном подключении.

\begin{verbatim}
ip domain-name CCNA-lab.com
username andrey privilege 15 secret andreypass1
line vty 0 4
transport input ssh
login local
exit
crypto key generate rsa
exit
\end{verbatim}

Подключитесь к маршрутизатору R1\_ФАМИЛИЯ
от компьютера PC-A по протоколу SSH.

\begin{image}
	\includegrph{Screenshot from 2024-03-18 20-08-11}
	\caption{Подключение по SSH}
\end{image}

\textbf{Удаленный доступ был настроен успешно?}

Да

\section{Отображение сведений о маршрутизаторе}

\subsection{Установите SSH-подключение к R1\_ФАМИЛИЯ}

На компьютере PC-B создайте сеанс SSH с маршрутизатором R1\_ФАМИЛИЯ
по IP-адресу 192.168.0.1 и войдите в систему, используя имя пользователя
(ваша фамилия на английском языке) и пароль,
который вы придумали самостоятельно.

\subsection{Получите основные данные об аппаратном и программном обеспечении}

Используйте команду \texttt{show version},
чтобы ответить на вопросы о маршрутизаторе.

\textbf{Как называется образ IOS,
под управлением которой работает маршрутизатор?}

Версия изображения может отличаться, но ответы должны быть примерно такими:
isr4300-universalk9.16.06.04.SPA.bin.

\textbf{Какой объем энергонезависимого ОЗУ (NVRAM) имеет маршрутизатор?}

4194304K

\textbf{Каким объемом флеш-памяти обладает маршрутизатор?}

3207167K

\begin{image}
	\includegrph{Screenshot from 2024-03-18 20-13-32}
	\caption{Вывод команды show version}
\end{image}

Зачастую команды \texttt{show} могут выводить несколько экранов данных.
Фильтрация выходных данных позволяет пользователю отображать
лишь нужные разделы выходных данных.
Чтобы включить команду фильтрации, после команды \texttt{show}
введите прямую черту (|),
после которой следует ввести параметр и выражение фильтрации.
Чтобы отобразить все строки выходных данных,
которые содержат выражение фильтрации,
можно согласовать выходные данные с оператором фильтрации
с помощью ключевого слова include.
Настройте фильтрацию для команды\texttt{show version} 
и используйте команду \texttt{show version | include register},
чтобы ответить на следующий вопрос.

\textbf{Какому процессу загрузки последует маршрутизатор
при следующей перезагрузке?}

Configuration register is 0x2102

\subsection{Отобразите загрузочную конфигурацию}

\begin{verbatim}
show startup-config
\end{verbatim}

\begin{image}
	\includegrph{Screenshot from 2024-03-18 20-23-40}
	\caption{Вывод команды show startup-config}
\end{image}

\textbf{Как пароли представлены в выходных данных?}

\begin{verbatim}
show startup-config | begin vty
\end{verbatim}

\texttt{НЕ РАБОТАЕТ}

\textbf{Что происходит в результате выполнения этой команды?}

\subsection{Отобразите таблицу маршрутизации на маршрутизаторе}

\begin{verbatim}
show ip route
\end{verbatim}

\begin{image}
	\includegrph{Screenshot from 2024-03-18 20-27-06}
	\caption{Вывод команды show ip route}
\end{image}

\textbf{Какой код используется в таблице маршрутизации
для обозначения сети с прямым подключением?}

Буква C обозначает подсеть с прямым подключением.
Буква L обозначает локальный интерфейс.

\textbf{Сколько записей маршрутов закодированы с символом «C»
в таблице маршрутизации?}

2

\subsection{Отобразите на маршрутизаторе сводный список интерфейсов}

\begin{verbatim}
show ip interface brief
\end{verbatim}

\begin{image}
	\includegrph{Screenshot from 2024-03-18 20-29-47}
	\caption{Вывод команды show ip interface brief}
\end{image}

\textbf{Какая команда позволяет изменить состояние
портов Gigabit Ethernet с DOWN на UP?}

no shutdown

\section{Настройка протокола IPv6 и проверка подключения}
\subsection{Назначьте IPv6-адреса интерфейсу G0/0 маршрутизатора R1\_ФАМИЛИЯ
и включите IPv6-маршрутизацию}

Назначьте интерфейсу G0/0 глобальный индивидуальный
IPv6-адрес --- 2001:db8:acad:a::1/64,
в дополнение к индивидуальному адресу на интерфейсе назначьте локальный адрес
канала (linklocal) --- fe80::1. Включите IPv6-маршрутизацию.

\begin{verbatim}
configure terminal
interface g0/0
ipv6 address 2001:db8:acad:a::1/64
ipv6 address fe80::1 link-local
no shutdown
exit
ipv6 unicast-routing
exit
\end{verbatim}

Проверьте параметры IPv6 на маршрутизаторе R1\_ФАМИЛИЯ.

\begin{verbatim}
show ipv6 interface brief
\end{verbatim}

\begin{image}
	\includegrph{Screenshot from 2024-03-18 20-36-25}
	\caption{Вывод команды show ipv6 interface brief}
\end{image}

\textbf{Если интерфейсу G0/1 не назначен IPv6-адрес,
то почему он отображается как [up/up] (ВКЛ/ВКЛ)?}

Статус [up/up] отражает статус интерфейса уровня 1 и уровня 2
и не зависит от статуса уровня 3.

На компьютере PC-B выполните команду для отображения настроек IPv6.

\begin{verbatim}
ipconfig
\end{verbatim}

\textbf{Какой IPv6-адрес назначен компьютеру PC-B?}

Link-local IPv6 Address.........: FE80::2D0:D3FF:FEEA:441C

\textbf{Какой шлюз по умолчанию назначен компьютеру PC-B?}

FE80::1 и 192.168.0.1

От компьютера PC-B отправьте эхо-запрос на локальный адрес
канала шлюза по умолчанию маршрутизатора R1\_ФАМИЛИЯ.

\begin{image}
	\includegrph{Screenshot from 2024-03-18 21-02-41}
	\caption{Эхо-запрос от PC-B к R1 по ipv4}
\end{image}

\textbf{Была ли проверка успешной?}

Да

От компьютера PC-B отправьте эхо-запрос на индивидуальный IPv6-адрес
маршрутизатора R1\_ФАМИЛИЯ 2001:db8:acad:a::1.

\begin{image}
	\includegrph{Screenshot from 2024-03-18 20-53-30}
	\caption{Эхо-запрос от PC-B к R1 по ipv6}
\end{image}


\textbf{Была ли проверка успешной?}

Да

