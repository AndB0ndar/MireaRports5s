\section{Выполнение практической работы}

\section{Настройка устройств и проверка подключения}

\subsection{Настройте маршрутизатор}

\begin{verbatim}
enable
config terminal
hostname R1_BONDAR
no ip domain lookup
security passwords min-length 10
enable secret cisco12345
line console 0
password ciscoconpass
exec-timeout 5 0
login
logging synchronous
exit
line vty 0 4
password ciscovtypass
exec-timeout 5 0
login
logging synchronous
exit
service password-encryption
banner motd # You must be authorizeded! #
interface g0/0/0
ip address 192.168.0.1 255.255.255.0
no shutdown
interface g0/0/1
ip address 192.168.1.1 255.255.255.0
no shutdown
end
clock set 20:00:00 18 Mar 2024
copy running-config startup-config
\end{verbatim}

\textbf{Укажите способы усиления защиты паролей,
кроме установки минимальной длины.}

Используйте в паролях заглавные буквы, цифры и специальные символы.

\textbf{Что произойдет, если перезагрузить маршрутизатор до того,
как будет выполнена команда \texttt{copy running-config startup-config}?}

Содержимое запущенной конфигурации будет удалено.
В этой лабораторной работе маршрутизатор не будет иметь конфигурации запуска.
После перезагрузки пользователя спросят,
хочет ли он войти в диалоговое окно начальной настройки.

\subsection{Проверьте подключение к сети}

Из командной строки компьютера PC-B отправьте эхо-запрос на компьютер PC-A.

\begin{image}
	\includegrph{Screenshot from 2024-03-18 19-55-11}
	\caption{Эхо-запрос от PC-B к PC-A}
\end{image}

\textbf{Успешно ли выполнена проверка связи?}

Да

\textbf{Какой тип удаленного доступа будет использоваться для получения
доступа к маршрутизатору R1\_ФАМИЛИЯ после завершения этого набора команд?}

Telnet

Подключитесь к маршрутизатору R1\_ФАМИЛИЯ
от компьютера PC-A с помощью службы Telnet.

\begin{image}
	\includegrph{Screenshot from 2024-03-18 19-55-11}
	\caption{Подключение по Telnet}
\end{image}

\textbf{Почему использование протокола Telnet считается угрозой безопасности?}

Сеанс Telnet можно увидеть в виде открытого текста. Он не зашифрован.
Пароли можно легко увидеть с помощью анализатора пакетов.

\subsection{Настройте маршрутизатор для доступа по протоколу SSH}
Активируйте подключения SSH и создайте пользователя
(username --- ваша фамилия на английском языке,
доменное имя маршрутизатора --- CCNA-lab.com)
в локальной базе данных маршрутизатора. Длина ключа шифрования --- 1024 бит.
Не забудьте записать пароль, чтобы не забыть его при повторном подключении.

\begin{verbatim}
ip domain-name CCNA-lab.com
username andrey privilege 15 secret andreypass1
line vty 0 4
transport input ssh
login local
exit
crypto key generate rsa
exit
\end{verbatim}

Подключитесь к маршрутизатору R1\_ФАМИЛИЯ
от компьютера PC-A по протоколу SSH.

\begin{image}
	\includegrph{Screenshot from 2024-03-18 20-08-11}
	\caption{Подключение по SSH}
\end{image}

\textbf{Удаленный доступ был настроен успешно?}

Да

\section{Отображение сведений о маршрутизаторе}

\subsection{Установите SSH-подключение к R1\_ФАМИЛИЯ}

На компьютере PC-B создайте сеанс SSH с маршрутизатором R1\_ФАМИЛИЯ
по IP-адресу 192.168.0.1 и войдите в систему, используя имя пользователя
(ваша фамилия на английском языке) и пароль,
который вы придумали самостоятельно.

\subsection{Получите основные данные об аппаратном и программном обеспечении}

Используйте команду \texttt{show version},
чтобы ответить на вопросы о маршрутизаторе.

\textbf{Как называется образ IOS,
под управлением которой работает маршрутизатор?}

Версия изображения может отличаться, но ответы должны быть примерно такими:
isr4300-universalk9.16.06.04.SPA.bin.

\textbf{Какой объем энергонезависимого ОЗУ (NVRAM) имеет маршрутизатор?}

4194304K

\textbf{Каким объемом флеш-памяти обладает маршрутизатор?}

3207167K

\begin{image}
	\includegrph{Screenshot from 2024-03-18 20-13-32}
	\caption{Вывод команды show version}
\end{image}

Зачастую команды \texttt{show} могут выводить несколько экранов данных.
Фильтрация выходных данных позволяет пользователю отображать
лишь нужные разделы выходных данных.
Чтобы включить команду фильтрации, после команды \texttt{show}
введите прямую черту (|),
после которой следует ввести параметр и выражение фильтрации.
Чтобы отобразить все строки выходных данных,
которые содержат выражение фильтрации,
можно согласовать выходные данные с оператором фильтрации
с помощью ключевого слова include.
Настройте фильтрацию для команды\texttt{show version} 
и используйте команду \texttt{show version | include register},
чтобы ответить на следующий вопрос.

\textbf{Какому процессу загрузки последует маршрутизатор
при следующей перезагрузке?}

Configuration register is 0x2102

\subsection{Отобразите загрузочную конфигурацию}

\begin{verbatim}
show startup-config
\end{verbatim}

\begin{image}
	\includegrph{Screenshot from 2024-03-18 20-23-40}
	\caption{Вывод команды show startup-config}
\end{image}

\textbf{Как пароли представлены в выходных данных?}

\begin{verbatim}
show startup-config | begin vty
\end{verbatim}

\texttt{НЕ РАБОТАЕТ}

\textbf{Что происходит в результате выполнения этой команды?}

\subsection{Отобразите таблицу маршрутизации на маршрутизаторе}

\begin{verbatim}
show ip route
\end{verbatim}

\begin{image}
	\includegrph{Screenshot from 2024-03-18 20-27-06}
	\caption{Вывод команды show ip route}
\end{image}

\textbf{Какой код используется в таблице маршрутизации
для обозначения сети с прямым подключением?}

Буква C обозначает подсеть с прямым подключением.
Буква L обозначает локальный интерфейс.

\textbf{Сколько записей маршрутов закодированы с символом «C»
в таблице маршрутизации?}

2

\subsection{Отобразите на маршрутизаторе сводный список интерфейсов}

\begin{verbatim}
show ip interface brief
\end{verbatim}

\begin{image}
	\includegrph{Screenshot from 2024-03-18 20-29-47}
	\caption{Вывод команды show ip interface brief}
\end{image}

\textbf{Какая команда позволяет изменить состояние
портов Gigabit Ethernet с DOWN на UP?}

no shutdown

\section{Настройка протокола IPv6 и проверка подключения}
\subsection{Назначьте IPv6-адреса интерфейсу G0/0 маршрутизатора R1\_ФАМИЛИЯ
и включите IPv6-маршрутизацию}

Назначьте интерфейсу G0/0 глобальный индивидуальный
IPv6-адрес --- 2001:db8:acad:a::1/64,
в дополнение к индивидуальному адресу на интерфейсе назначьте локальный адрес
канала (linklocal) --- fe80::1. Включите IPv6-маршрутизацию.

\begin{verbatim}
configure terminal
interface g0/0
ipv6 address 2001:db8:acad:a::1/64
ipv6 address fe80::1 link-local
no shutdown
exit
ipv6 unicast-routing
exit
\end{verbatim}

Проверьте параметры IPv6 на маршрутизаторе R1\_ФАМИЛИЯ.

\begin{verbatim}
show ipv6 interface brief
\end{verbatim}

\begin{image}
	\includegrph{Screenshot from 2024-03-18 20-36-25}
	\caption{Вывод команды show ipv6 interface brief}
\end{image}

\textbf{Если интерфейсу G0/1 не назначен IPv6-адрес,
то почему он отображается как [up/up] (ВКЛ/ВКЛ)?}

Статус [up/up] отражает статус интерфейса уровня 1 и уровня 2
и не зависит от статуса уровня 3.

На компьютере PC-B выполните команду для отображения настроек IPv6.

\begin{verbatim}
ipconfig
\end{verbatim}

\textbf{Какой IPv6-адрес назначен компьютеру PC-B?}

Link-local IPv6 Address.........: FE80::2D0:D3FF:FEEA:441C

\textbf{Какой шлюз по умолчанию назначен компьютеру PC-B?}

FE80::1 и 192.168.0.1

От компьютера PC-B отправьте эхо-запрос на локальный адрес
канала шлюза по умолчанию маршрутизатора R1\_ФАМИЛИЯ.

\begin{image}
	\includegrph{Screenshot from 2024-03-18 21-02-41}
	\caption{Эхо-запрос от PC-B к R1 по ipv4}
\end{image}

\textbf{Была ли проверка успешной?}

Да

От компьютера PC-B отправьте эхо-запрос на индивидуальный IPv6-адрес
маршрутизатора R1\_ФАМИЛИЯ 2001:db8:acad:a::1.

\begin{image}
	\includegrph{Screenshot from 2024-03-18 20-53-30}
	\caption{Эхо-запрос от PC-B к R1 по ipv6}
\end{image}


\textbf{Была ли проверка успешной?}

Да

\section{Ответы на контрольные вопросы}

\subsection{Дайте определение понятию “маршрутизация”. Какими 
	способами маршрутизатор получает сведения об удаленных 
	сетях?}

Когда маршрутизатор получает IP-пакет на одном интерфейсе, он 
определяет, какой интерфейс следует использовать для пересылки пакета до 
места назначения. Это называется маршрутизация.\par
Удаленные сети --- это сети, которые напрямую не подключены к 
маршрутизатору. Маршрутизатор получает сведения об удаленных сетях 
двумя способами:

\begin{itemize}
	\item Статические маршруты --- добавляется в таблицу маршрутизации, 
		когда маршрут настраивается вручную.
	\item Протоколы динамической маршрутизации --- добавляется в таблицу 
		маршрутизации, когда протоколы маршрутизации динамически узнают 
		о удаленной сети. Протоколы динамической маршрутизации включают 
		Enhanced Interior Gateway Routing Protocol (EIGRP), Open Shortest Path 
		First (OSPF) и другие.
\end{itemize}

\subsection{Что означает понятие “поиск наилучшего совпадения” 
относительно маршрутизатора? Для чего служат статические 
маршруты?}

Наилучшим совпадением является маршрут в таблице маршрутизации, 
в котором максимальное число крайних левых битов совпадает с IPv4-
адресом назначения пакета. Маршрут с самым большим числом 
эквивалентных крайних левых битов (самое длинное совпадение) всегда 
является предпочтительным.\par
Статические маршруты --- добавляется в таблицу маршрутизации, 
когда маршрут настраивается вручную. С помощью таких маршрутов вы 
можете пустить часть трафика через прокси-сервер, настроить связность с 
инфраструктурой и многое другое.

\subsection{Опишите процесс пересылки пакетов маршрутизатором. Что 
произойдет, если в таблице маршрутизации нет соответствия 
между IP-адресом назначения и префиксом?}

Процесс пересылки пакетов, маршрутизатором:
Кадр канального уровня с инкапсулированным IP-пакетом поступает 
на входной интерфейс.\par
Маршрутизатор проверяет IP-адрес назначения в заголовке пакета и 
обращается к своей таблице IP-маршрутизации.
Маршрутизатор находит самый длинный совпадающий префикс в 
таблице маршрутизации.\par
Маршрутизатор инкапсулирует пакет во кадр канального уровня 
выходного интерфейса и пересылает его из него. Назначением может 
быть устройство, подключенное к сети, или маршрутизатор 
следующего перехода.\par
Однако если нет соответствующей записи маршрута, пакет 
отбрасывается.\par
Если в таблице маршрутизации нет соответствия между IP-адресом 
назначения и префиксом, и если маршрут по умолчанию отсутствует, пакет 
будет отброшен.\par

\subsection{Дайте характеристику механизмам пересылки пакетов. 
Опишите все возможные источники получения маршрутов в 
таблице маршрутизации.}

Маршрутизаторы поддерживают три механизма пересылки пакетов:

\begin{itemize}
	\item \textbf{Процессорная коммутация (Process switching)}\\
		Устаревший механизм пересылки пакетов, все еще доступный на 
		маршрутизаторах Cisco. Когда пакет прибывает на интерфейс, он 
		пересылается на уровень управления,
		де ЦП сопоставляет адрес назначения 
		с записью в таблице маршрутизации, а затем определяет выходной 
		интерфейс и пересылает пакет. Механизм процессорной коммутации 
		работает очень медленно и редко реализуется в современных сетях.
		Сравните данный механизм с механизмом быстрой коммутации.
	\item \textbf{Быстрая коммутация (Fast switching)}\\
		Быстрое переключение использует кэш быстрой коммутации для 
		хранения информации следующего перехода. Когда пакет прибывает на 
		интерфейс, он пересылается на уровень управления,
		где ЦП ищет совпадение 
		в кэше быстрой коммутации. Если совпадение не найдено, пакет проходит 
		программную коммутацию и пересылается на выходной интерфейс. 
		Информация о трафике для пакетов также хранится в кэше быстрой 
		коммутации. Если на интерфейс прибывает другой пакет,
		адресованный тому 
		же назначению, то из кэш-памяти повторно используется информация о 
		следующем переходе без вмешательства ЦП.
	\item \textbf{Cisco Express Forwarding (CEF)}
		CEF является самым новым и используемым по умолчанию 
		механизмом пересылки пакетов Cisco IOS. Как и быстрая коммутация, CEF 
		создает 24-портовую базу данных переадресации (FIB)
		и таблицу смежности. 
		Однако записи таблицы инициированы не пакетами, как при быстрой 
		коммутации, а изменениями. Таким образом,
		по завершении сходимости сети 
		в базе данных FIB и таблице смежности содержится вся информация, 
		необходимая маршрутизатору при пересылке пакета.
\end{itemize}

Источники:

\begin{itemize}
	\item непосредственно подключенные сети;
	\item статические маршруты;
	\item протоколов динамической маршрутизации.
\end{itemize}

\subsection{В каких случаях целесообразно настроить статический 
	маршрут? Дайте определение понятию "<административное расстояние">}

Статическая маршрутизация используется в трех ситуациях:
Это обеспечивает простоту обслуживания таблиц маршрутизации 
в небольших сетях, рост которых не ожидается.
Использование единого маршрута по умолчанию для 
представления пути к любой сети, которая не имеет более точного 
соответствия с другим маршрутом в таблице маршрутизации. 
Маршруты по умолчанию используются для отправки трафика к 
любому целевому адресу за пределами следующего вышестоящего 
маршрутизатора.\par
Маршрутизация к тупиковым сетям и от них. Тупиковая сеть 
представляет собой сеть, доступ к которой осуществляется через 
один маршрут, и маршрутизатор имеет только одно соседнее 
устройство.\par
Административное расстояние (AD) --- величина выражает 
надежность маршрута. Чем ниже значение AD, тем выше надежность. 

\subsection{В каких случаях целесообразно настроить динамическую 
	маршрутизацию? Дайте определение понятию "<метрика маршрута">}

Протоколы динамической маршрутизации обычно используются в 
следующих сценариях:

\begin{itemize}
	\item В сетях, состоящих из более чем нескольких маршрутизаторов;
	\item Когда изменение топологии сети требует от сети автоматического 
		определения другого пути;
	\item Для масштабируемости. По мере роста сети протокол 
		динамической маршрутизации автоматически узнает о новых сетях.
\end{itemize}

Метрика --- это числовое значение, используемое для измерения 
расстояния до заданной сети. Наиболее оптимальным путем к сети является 
путь с наименьшей метрикой.

\subsection{Проведите краткую сравнительную характеристику 
статической и динамической маршрутизации на основе 
нескольких критериев. Какие бывают протоколы динамической 
маршрутизации (опишите категории и приведите примеры)?}

\begin{image}
	\includegrph{table1}
\end{image}

Протоколы динамической маршрутизации классифицируются обычно 
на два типа: внутридоменные (IGP) и междуобластные (EGP).\par
Внутридоменные протоколы динамической маршрутизации (Interior 
Gateway Protocol, IGP) используются для обмена информацией о 
маршрутах внутри одной автономной системы (AS). Примеры 
внутридоменных протоколов:

\begin{itemize}
	\item RIP (Routing Information Protocol) - старый простой протокол, 
		работающий на основе метрики числа прыжков.
	\item OSPF (Open Shortest Path First) - протокол, который определяет 
		маршруты по наименьшей стоимости в сети на основе Dijkstra's 
		algorithm.
	\item EIGRP (Enhanced Interior Gateway Routing Protocol) - 
		проприетарный протокол Cisco, комбинирующий преимущества 
		протоколов векторного и состояния канала.
	\item Междуобластные протоколы динамической маршрутизации (Exterior 
		Gateway Protocol, EGP) используются для обмена информацией о 
		маршрутах между различными автономными системами (AS). Пример 
		междуобластных протоколов:
	\item BGP (Border Gateway Protocol) - протокол, используемый на 
		границах автономных систем для обмена информацией о 
		маршрутах и выбора оптимального маршрута на основе различных 
		атрибутов.
\end{itemize}

\subsection{Для чего нужны протоколы динамической маршрутизации? 
Какие компоненты включают в себя протоколы динамической 
маршрутизации?}

Протоколы динамической маршрутизации позволяют 
маршрутизаторам совместно использовать сведения о надежности и 
состоянии удаленных сетей. Протоколы динамической маршрутизации 
выполняют ряд операций, включая обнаружение сетей и ведение таблиц 
маршрутизации.\par
Важными преимуществами протоколов динамической маршрутизации 
являются возможность выбора наилучшего пути и возможность 
автоматического обнаружения нового наилучшего пути при изменении 
топологии.\par
Протоколы динамической маршрутизации включают в себя следующие 
компоненты:

\begin{itemize}
	\item Структуры данных --- протоколы маршрутизации обычно 
		используют для своих операций таблицы или базы данных. Данная 
		информация хранится в ОЗУ.
	\item Сообщения протокола маршрутизации --- протоколы 
		маршрутизации используют различные типы сообщений для 
		обнаружения соседних маршрутизаторов, обмена информацией о 
		маршрутах и выполнения других задач, связанных с получением 
		точной информации о сети.
	\item Алгоритм --- алгоритм представляет собой определенный список 
		действий, используемых для выполнения задачи. Протоколы 
		маршрутизации используют алгоритмы, упрощающие обмен данных 
		маршрутизации и определение оптимального пути.
\end{itemize}

\subsection{Как вычисляется метрика для протоколов RIP, OSPF и EIGRP? 
	Как работает распределение нагрузки при использовании 
	динамической маршрутизации?}

\begin{image}
	\includegrph{table2}
\end{image}

Если маршрутизатор располагает двумя или более путями к пункту 
назначения с метриками равной стоимости, он отправляет пакеты по обоим 
путям. Это называется распределением нагрузки в соответствии с равной 
стоимостью. Таблица маршрутизации содержит одну сеть назначения, но 
несколько выходных интерфейсов --- по одному для каждого пути с равной 
стоимостью. Маршрутизатор пересылает пакеты через несколько выходных 
интерфейсов, указанных в таблице маршрутизации.
При правильной конфигурации распределение нагрузки может 
повысить эффективность и производительность сети.

\subsection{Опишите назначение кодов C, L и S в таблице маршрутизации. 
	В каких случаях используется протокол BGP?}

Коды:

\begin{itemize}
	\item L --- указывает адрес, назначенный интерфейсу маршрутизатора. 
		Данный код позволяет маршрутизатору быстро определить, что 
		полученный пакет предназначен для интерфейса, а не для пересылки.
	\item C --- определяет сеть с прямым подключением.
	\item S --- определяет статический маршрут, созданный для 
		достижения конкретной сети.
\end{itemize}

Протокол граничного шлюза (BGP) используется для связи между 
сетями интернет-провайдеров. Протокол BGP также обеспечивает обмен 
данными маршрутизации между интернет-провайдерами и их крупными 
частными клиентами.

