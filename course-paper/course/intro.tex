\chapter*{ВВЕДЕНИЕ}
\addcontentsline{toc}{chapter}{ВВЕДЕНИЕ}

В современном мире, особенно в условиях быстрого развития информационных
технологий, эффективное управление временем и ресурсами играет ключевую роль.
Одним из актуальных аспектов этого управления является организация очередности
в различных сферах деятельности, включая образование.
Студенты, преподаватели и сотрудники учебных заведений часто сталкиваются
с необходимостью эффективной организации очередности, например,
при сдаче работ, консультациях или других административных процедурах.\par
Существующие методы управления очередностью,
такие как ручное назначение времени или фиксированные расписания,
часто не обеспечивают оптимального результата. Они могут быть неудобными,
недостаточно гибкими и могут приводить к неэффективному использованию времени
как для преподавателей, так и для студентов.
Разработка мобильного приложения "Очередь"
представляет собой инновационный подход к решению этой проблемы,
предоставляя удобный инструмент для организации
и управления очередностью через мобильные устройства.\par
Такое приложение будет способствовать оптимизации процесса
организации очередности, уменьшению времени ожидания
и повышению эффективности работы как для преподавателей,
так и для студентов. Приложение "Очередь" будет актуальным
и востребованным инструментом, отвечающим современным требованиям управления
временем и ресурсами в образовательных учреждениях.\par
Цель данной курсовой работы --- разработка программного комплекса
для организации очереди, включающего мобильное приложение
на платформе Android и серверную часть
на основе Ktor с использованием языка программирования Kotlin.
Мобильное приложение должно предоставить пользователям интуитивно понятный
интерфейс для записи в очередь, мониторинга текущего статуса
и получения уведомлений о продвижении. Серверная часть будет обеспечивать
взаимодействие с клиентами, управление данными и их безопасное хранение.\par
Для реализации серверной части выбрано использование фреймворка Ktor,
который позволяет создавать легковесные и производительные серверные
приложения на Kotlin. В качестве системы управления базами данных (СУБД)
выбрана PostgreSQL, известная своей надежностью,
производительностью и поддержкой сложных запросов.\par
Введение включает обзор проблематики и актуальности темы,
обоснование выбора технологий и инструментов для разработки,
а также описание структуры и содержания работы.
В первой главе проводится анализ состояния исследуемой области,
рассматриваются существующие решения и выявляются основные требования
к программному комплексу.
Вторая глава посвящена проектированию мобильного приложения,
включая определение его функциональных возможностей и архитектуры.
В третьей главе описывается процесс разработки мобильного приложения,
включающий реализацию ключевых компонентов и функционала.
Заключение подводит итоги работы
и намечает возможные направления дальнейшего развития проекта.\par
Таким образом, данная курсовая работа направлена на создание комплексного
решения для организации очередей, которое может быть адаптировано
под различные условия и требования,
обеспечивая при этом высокую эффективность и удобство использования.

