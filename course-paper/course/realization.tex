\Chapter{РАЗРАБОТКА МОБИЛЬНОГО ПРИЛОЖЕНИЯ}

\Section{Разработка клиентской части мобильного приложения}

В этой главе будет рассмотрено создание клиентской части мобильного
приложения "'Очередь"', включая написание ApiService и RetrofitInstanse
для взаимодействия с сервером, создание адаптеров для активити,
написание синглтона для хранения токена пользователя,
создание класса приложения (App) и разработка активити.

\Subsection{Написание ApiService и RetrofitInstanse}

Для взаимодействия с сервером были разработаны
классы ApiService и RetrofitInstance.
ApiService содержит методы для отправки запросов к серверу,
а RetrofitInstance инициализирует Retrofit и создает экземпляр ApiService
(Приложение~\ref{lst:retrofit}).

\Subsection{Написание адаптеров}

В этой главе рассматривается разработка адаптеров для отображения списков
очередей и элементов очереди в мобильном приложении "'Очередь"'.
Адаптеры позволяют связывать данные с элементами пользовательского интерфейса,
обеспечивая удобное взаимодействие пользователя с приложением.\par
Созданные адаптеры обеспечивают динамическое обновление и отображение данных
в списках, что делает взаимодействие с приложением удобным
и интуитивно понятным для пользователей.

\Subsubsection{QueueListAdapter}
Адаптер \texttt{QueueListAdapter} используется
для отображения списка очередей в приложении.
Он обеспечивает фильтрацию списка, обработку нажатий на элементы
и удаление очередей, озданных текущим пользователем
(листнинг~\ref{lst:adapter:queue:list}).

\Subsubsection{QueueItemAdapter}
Адаптер \texttt{QueueItemAdapter} используется
для отображения списка пользователей в очереди.
Он обеспечивает фильтрацию списка и отображение информации
о каждом пользователе (листнинг~\ref{lst:adapter:queue:item}).

\Subsection{Написание синглтона для хранения токена пользователя}

Для управления токеном пользователя
в приложении используется синглтон \texttt{AuthManager}
(Приложение~\ref{lst:auth:manager}).
Синглтон обеспечивает централизованное хранение
и доступ к токену аутентификации,
который используется для выполнения защищённых запросов к серверу.\par
Использование синглтона \texttt{AuthManager}
позволяет упростить управление токеном
и обеспечивает централизованный доступ к нему из различных частей приложения.
Это особенно полезно при выполнении сетевых запросов,
требующих аутентификации пользователя.

\Subsection{Создание класса приложения (App)}

Класс приложения (App) отвечает за инициализацию глобальных настроек
и компонентов, необходимых для работы мобильного приложения "'Очередь"'.
Он управляет темами (светлой и тёмной)
и сохраняет предпочтения пользователя.\par
Ниже приведён код класса \texttt{App},
который наследуется от класса \texttt{Application}.
Он содержит методы для установки темы приложения
и сохранения предпочтений пользователя (Приложение~\ref{lst:app}).\par
Класс \texttt{App} выполняет важную роль в управлении настройками приложения
и обеспечении удобства использования,
позволяя пользователям выбирать предпочитаемую тему оформления.

\Subsection{Разработка активити}

В этой главе рассматривается разработка основных активити
для мобильного приложения "'Очередь"'
в соответствии с проектированием пользовательского интерфейса,
представленным ранее (см. \ref{sc:designing:ui}).
Мы создадим следующие активити:
вход, регистрации, домашнее окно со списком очередей пользователя,
профиль, добавление очереди, очередь со списком пользователей в очереди.\par
Каждое активити отвечает за свою функциональность
и обеспечивает пользователю удобный интерфейс для работы
с приложением "'Очередь"'. В следующей главе будет рассмотрена разработка
серверной части и взаимодействие с API.

\Subsubsection{Активити входа (LoginActivity)}
Активити входа предоставляет пользователю возможность войти
в приложение, введя свои учетные данные (Приложение~\ref{lst:activity:login}).
В этом активити используются текстовые поля для ввода логина и пароля,
а также кнопка для отправки данных на сервер для аутентификации.
В случае успешного входа пользователя перенаправляют на домашнее окно.
Также предусмотрена возможность перехода на активити регистрации.

\Subsubsection{Активити регистрации (RegisterActivity)}
Активити регистрации позволяет пользователю создать новую учетную запись
(Приложение~\ref{lst:activity:register}).
Пользователь вводит необходимые данные, такие как логин, имя, фамилия и пароль,
после чего отправляет эти данные на сервер для создания учетной записи.
В случае успешной регистрации пользователя перенаправляют на активити входа
для авторизации.

\Subsubsection{Домашнее окно (HomeActivity)}
Домашнее окно отображает список очередей, созданных пользователем,
и предоставляет доступ к другим функциям приложения
(Приложение~\ref{lst:activity:home}).
Здесь реализована возможность добавления новой очереди
через соответствующую кнопку. Список очередей представлен в виде RecyclerView
с адаптером, который обновляет данные при добавлении или удалении очередей.

\Subsubsection{Активити добавления очереди (AddQueueActivity)}
Активити добавления очереди позволяет пользователю создать новую очередь
(Приложение~\ref{lst:activity:queue:add}).
Пользователь вводит название очереди и описание,
после чего отправляет эти данные на сервер для создания новой очереди.
В случае успешного создания очереди пользователь возвращается на домашнее окно,
где новый элемент отображается в списке.

\Subsubsection{Активити профиля (ProfileActivity)}
Активити профиля предоставляет пользователю доступ к его профилю
и возможность изменения настроек (Приложение~\ref{lst:activity:profile}).
Пользователь может просматривать свои данные,
такие как логин, имя и фамилия, а также выполнить выход из системы.
При выходе выполняется сброс токена авторизации,
и пользователя перенаправляют на активити входа.

\Subsubsection{Активити очереди (QueueActivity)}
Активити очереди отображает список пользователей, находящихся в очереди
(Приложение~\ref{lst:activity:queue}).
Здесь используется RecyclerView для отображения списка элементов очереди
с соответствующим адаптером.
Данные загружаются с сервера и обновляются в реальном времени,
предоставляя актуальную информацию о текущем состоянии очереди.

\Subsection{Создание ресурсов приложения}

В этой главе мы рассмотрим создание ресурсов приложения,
таких как строки, цвета и размеры, которые используются
для стилизации пользовательского интерфейса и управления текстом,
цветами и размерами элементов.\par
Создание и использование ресурсов позволяет сделать приложение более гибким
и легко настраиваемым, что упрощает поддержку
и развитие приложения в дальнейшем.

\Subsubsection{Строковые ресурсы}
Строковые ресурсы используются для хранения текстовых значений,
таких как названия кнопок, подписей и сообщений об ошибках.
Они обеспечивают легкость перевода приложения на различные языки
и облегчают обновление текста во всем приложении.
(Листнинг~\ref{lst:res:strings}).

\Subsubsection{Цветовые ресурсы}
Цветовые ресурсы используются для хранения цветов,
которые используются в пользовательском интерфейсе приложения.
Они обеспечивают единообразие в цветовой схеме приложения
и упрощают изменение цветовых палитр.
(Листнинг~\ref{lst:res:colors}).

\Subsubsection{Ресурсы размеров}
Ресурсы размеров используются для хранения размеров,
таких как высота, ширина, отступы и размеры шрифтов.
Они позволяют легко изменять размеры элементов интерфейса во всем приложении.
(Листнинг~\ref{lst:res:dimens}).

\Subsubsection{Темы}
Темы используются для определения стиля и внешнего вида элементов интерфейса
(Листнинг~\ref{lst:res:themes}).


\Section{Разработка серверной части мобильного приложения}

В данной главе рассмотрим разработку серверной части мобильного приложения
"'Очередь"' с использованием фреймворка Ktor и базы данных PostgreSQL.
Будем описывать основные компоненты, такие как маршрутизация,
конфигурация приложения, подключение к базе данных,
модели данных и обработчики запросов.

\Subsection{Маршрутизация}

Маршрутизация в приложении осуществляется с использованием фреймворка Ktor.
В этом разделе описывается конфигурация маршрутов
для обработки запросов от клиентской части приложения
(Приложение~\ref{lst:routing}).

\Subsubsection{Общий маршрут}
В общем маршруте определен обработчик для корневого пути,
который возвращает текст "'Worked!"'.
Также включены обработчики авторизации
и маршруты для работы с группами и очередями.

\Subsubsection{Обработчик авторизации}
Обработчик авторизации определяет маршруты для входа,
регистрации и получения профиля пользователя.

\Subsubsection{Обработчик групп}
Обработчик групп определяет маршруты для работы с группами пользователей.

\Subsubsection{Обработчик очередей}
Обработчик очередей определяет маршруты для работы с очередями пользователей.

\Subsection{Конфигурация приложения}\label{sc:server:config}

В файле \texttt{application.conf} определяются настройки приложения,
такие как порт, модули и параметры подключения к базе данных PostgreSQL.\par
Здесь определены параметры порта, модули приложения
и настройки подключения к базе данных (Приложение~\ref{lst:config}).

\Subsection{Работа с базой данных}

Для работы с базой данных в приложении используется язык программирования Kotlin в сочетании с фреймворком Ktor. В этой главе описывается подключение к базе данных и сервисы для работы с данными.

\Subsubsection{Подключение к базе данных}
Для подключения к базе данных используется JDBC.
Определены функции для настройки и установки соединения с PostgreSQL
(Приложение~\ref{lst:db:connect}).
Параметры подключения (URL, имя пользователя, пароль)
хранятся в конфигурационном файле (см. \ref{sc:server:config}).\par

\Subsubsection{Сервисы}
Сервисы предоставляют уровень абстракции для работы с данными в базе данных.
Ниже приведены основные сервисы, используемые в приложении.

\textbf{AuthorizationService:} Сервис для работы
с авторизацией пользователей. Включает методы для входа
и регистрации пользователей, а также получения профиля пользователя
(Приложение~\ref{lst:db:service}).

\textbf{ConnectionService:}
Сервис для управления связями пользователей с очередями.
Обеспечивает создание, удаление и получение связей между пользователями
и очередями.

\textbf{GroupService:}
Сервис для работы с группами пользователей.
Предоставляет методы для получения списка всех групп
и информации о конкретной группе.

\textbf{QueuePosService:}
Сервис для работы с позициями пользователей в очередях.
Включает методы для добавления, удаления
и получения позиций пользователей в очередях.

\textbf{QueueService:}
Сервис для управления очередями. Обеспечивает создание, удаление
и получение информации о очередях.


\Subsection{Модели данных}

Модели данных представляют собой классы, используемые для передачи
и обработки информации между клиентской и серверной частями приложения.\par
Ниже приведены основные модели данных, используемые в приложении.

\Subsubsection{Модели авторизации}
Модели, связанные с процессом авторизации пользователей
(Листнинг~\ref{lst:models:auth}).

\begin{itemize}
    \item \texttt{LoginReceiveRemote}: Модель для передачи данных
		при входе пользователя. Включает в себя поля для логина и пароля.
    \item \texttt{RegisterReceiveRemote}: Модель для передачи данных
		при регистрации нового пользователя.
		Включает в себя поля для логина, имени, фамилии,
		названия группы и пароля.
    \item \texttt{TokenResponseRemote}: Модель для передачи
		токена авторизации. Включает в себя поле для токена.
\end{itemize}

\Subsubsection{Модель очереди}
Модель, связанная с созданием и управлением очередями.
\texttt{QueueReceiveRemote}: Модель для передачи данных
при создании новой очереди. Включает в себя поля для названия очереди,
названия группы и описания
(Листнинг~\ref{lst:models:queue}).

\Subsection{Обработчики (Handlers)}

Обработчики (Handlers) представляют функции,
которые обрабатывают запросы от клиентской части приложения
и взаимодействуют с базой данных для выполнения соответствующих операций.
В данной главе представлены основные обработчики, отвечающие за авторизацию,
управление группами и очередями.

\Subsubsection{Обработчик авторизации}
Обработчик \texttt{authorizationHandler} отвечает за обработку запросов,
связанных с авторизацией пользователей
(Листнинг~\ref{lst:handler:auth}).

\begin{itemize}
    \item \texttt{POST /login}: Обрабатывает запрос на вход пользователя.
		Проверяет переданные учетные данные и возвращает токен авторизации
		в случае успеха.
    \item \texttt{POST /register}: Обрабатывает запрос на регистрацию нового
		пользователя. Проверяет переданные данные,
		создает нового пользователя и возвращает токен авторизации.
    \item \texttt{GET /\{token\}/profile}: Получает профиль пользователя
		по токену авторизации.
\end{itemize}

\Subsubsection{Обработчик групп}
Обработчик \texttt{groupHandler} отвечает за работу с группами пользователей
(Листнинг~\ref{lst:handler:group}).

\begin{itemize}
    \item \texttt{GET /group/all}: Возвращает список всех групп.
    \item \texttt{GET /group/my}: Возвращает группу текущего пользователя.
\end{itemize}

\Subsubsection{Обработчик очередей}
Обработчик \texttt{queueHandler} отвечает за управление очередями
(Листнинг~\ref{lst:handler:queue}).

\begin{itemize}
    \item \texttt{GET /queues/all}: Возвращает список всех доступных очередей
		для текущего пользователя.
    \item \texttt{POST /queues/create}: Создает новую очередь.
    \item \texttt{DELETE /queues/delete/\{id\}}:
		Удаляет очередь по идентификатору.
    \item \texttt{POST /queues/connect/\{id\}}:
		Подключает пользователя к очереди.
\end{itemize}

\Subsection{Настройка Docker Compose для PostgreSQL}

Для запуска базы данных PostgreSQL в контейнере Docker используется файл
конфигурации Docker Compose (Листнинг~\ref{lst:docker-compose}).\par
В этом файле определен один сервис с именем \texttt{postgres},
который использует официальный образ PostgreSQL (\texttt{postgres:latest}).
Сервис настроен на прослушивание порта \texttt{6000} на хосте
и порта \texttt{5432} в контейнере, где PostgreSQL обычно работает.\par
Переменные окружения \texttt{POSTGRES\_DB}, \texttt{POSTGRES\_USER}
и \texttt{POSTGRES\_PASSWORD} определяют базу данных, имя пользователя
и пароль для доступа к базе данных PostgreSQL соответственно.\par
Кроме того, в опции \texttt{volumes} указано монтирование локальной директории 
\texttt{pg\_data} внутрь контейнера по пути \texttt{/var/lib/postgresql/data}. 
Это позволяет сохранять данные PostgreSQL между перезапусками контейнера.\par
Таким образом, данный файл Docker Compose настраивает и запускает контейнер
с PostgreSQL с определенными параметрами.

\Section{Тестирование разработанного приложения}

Для обеспечения качества и корректной работы разработанного мобильного
приложения необходимо провести тестирование.
В данной главе рассмотрены различные аспекты тестирования приложения,
а также представлены скриншоты работающего приложения.

\Subsection{Тестирование пользовательского интерфейса}

Тестирование пользовательского интерфейса (UI) включает
в себя несколько важных аспектов:

\begin{itemize}
	\item Оценивается, насколько интуитивно понятным
		и удобным для пользователя является интерфейс.
	\item Проверяется, насколько интерфейс соответствует установленным
		стандартам и рекомендациям по дизайну.
	\item Оценивается визуальная привлекательность интерфейса.
\end{itemize}

Для наглядного представления работы приложения прикладываются
скриншоты, демонстрирующие различные экраны и функциональность.
Эти скриншоты помогают лучше понять, как выглядит и функционирует приложение.

На рисунке~\ref{fig:registration} представлен экран,
на котором пользователь может зарегистрироваться в приложении,
введя свои данные.

\begin{image}
	\includegrph[scale=0.3]{Screenshot from 2024-06-01 15-16-45}
    \caption{Экран регистрации}
    \label{fig:registration}
\end{image}

\clearpage

На рисунке~\ref{fig:login} представлен экран входа в приложение,
где пользователь может ввести свои учетные данные для доступа
к функциональности приложения.

\begin{image}
	\includegrph[scale=0.27]{Screenshot from 2024-06-01 15-14-19}
    \caption{Экран входа в приложение}
    \label{fig:login}
\end{image}

На рисунке~\ref{fig:home} представлен главный экран приложения,
на котором отображаются очереди, связанные с пользователем.

\begin{image}
	\includegrph[scale=0.27]{Screenshot from 2024-06-01 15-14-31}
    \caption{Главный экран с очередями пользователя}
    \label{fig:home}
\end{image}

\clearpage

На рисунке~\ref{fig:home:history} представлен главный экран приложения,
на котором отображается история очереди пользователя.

\begin{image}
	\includegrph[scale=0.3]{Screenshot from 2024-06-01 15-15-21}
    \caption{Главный экран с историей очереди}
    \label{fig:home:history}
\end{image}

На рисунке~\ref{fig:profile} представлен экран профиля пользователя,
где пользователь может просмотреть и изменить свои личные данные.

\begin{image}
	\includegrph[scale=0.3]{Screenshot from 2024-06-01 15-14-35}
    \caption{Экран профиля пользователя}
    \label{fig:profile}
\end{image}

На рисунке~\ref{fig:queue:create} представлен экран создания новой очереди,
где пользователь может указать необходимые параметры для создания очереди.

\begin{image}
	\includegrph[scale=0.3]{Screenshot from 2024-06-01 15-14-39}
    \caption{Экран создания новой очереди}
    \label{fig:queue:create}
\end{image}

На рисунке~\ref{fig:queue} представлен экран с пользователями в очереди,
где пользователь может видеть других участников очереди.

\begin{image}
	\includegrph[scale=0.3]{Screenshot from 2024-06-01 15-14-43}
    \caption{Экран с пользователями в очереди}
    \label{fig:queue}
\end{image}

На скриншотах представлены основные экраны и функциональность приложения,
которые могут быть протестированы пользователем для убедительности
в его работоспособности.

