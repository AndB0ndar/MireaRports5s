\Chapter{АНАЛИЗ СОСТОЯНИЯ ИССЛЕДУЕМОЙ ОБЛАСТИ}

\Section{Описание предметной области мобильного приложения}

Предметная область мобильного приложения "'Очередь"' связана с организацией
и управлением очередями в различных сферах деятельности, в частности,
в учебных заведениях. Основной целью приложения является предоставление
эффективного инструмента для управления очередями студентов перед различными
событиями, такими как консультации, экзамены, прием документов и прочее.

Сравнительный анализ аналогов разрабатываемого мобильного приложения "'Очередь"'
позволяет выявить и оценить существующие решения в данной области
и определить их преимущества и недостатки.
Ниже представлен анализ нескольких аналогичных приложений:

\Subsection{Qminder:} Приложение Qminder предназначено
для управления очередями в различных сферах,
включая образование (рис.~\ref{fig:an:qminder}).
Оно предлагает возможность виртуального ожидания
в очереди через мобильное устройство,
уведомления о статусе очереди и т.д.
Однако некоторые пользователи отмечают сложность в настройке
и использовании интерфейса.

\begin{image}
	\includegrph[scale=0.3]{qminder}
	\caption{Приложение Qminder}
	\label{fig:an:qminder}
\end{image}

\Subsection{WaitWhile:} WaitWhile также позволяет организовывать
очереди и уведомлять клиентов о статусе (рис.~\ref{fig:an:waitwhile}).
Пользователи отмечают простоту использования
и настраиваемые опции уведомлений.
Однако интеграция с учебными заведениями может быть ограничена.

\begin{image}
	\includegrph[scale=0.45]{waitwhile}
	\caption{Приложение WaitWhile}
	\label{fig:an:waitwhile}
\end{image}

В сравнении с вышеупомянутыми приложениями,
разрабатываемое приложение "'Очередь"' будет ориентировано
на специфические потребности учебных заведений,
обеспечивая простоту использования, гибкость настройки
и оптимизацию процесса управления очередностью как для преподавателей,
так и для студентов.\par
Приложение предоставляет следующие функциональные возможности:

\begin{enumerate}
    \item Создание очереди: преподаватели или администраторы
			могут создавать новые очереди, устанавливать параметры,
			такие как время работы и максимальное количество студентов
			в очереди.
    \item Удаление своей очереди: пользователи могут удалить созданную
		ими очередь в случае необходимости.
    \item Подключение к очереди: студенты могут подключиться
		к существующей очереди, введя ее индентификатор.
    \item Добавление в очередь: студенты могут добавить себя в выбранную
		очередь, выбрав доступное время и оставив контактные данные.
    \item Удаление из очереди: студенты могут удалить себя из очереди
		в случае изменения планов или ненадобности.
\end{enumerate}

\Section{Функциональные истории}

Пользовательские истории являются важным инструментом
при разработке программного обеспечения, в том числе и мобильных приложений.
Они помогают команде разработчиков и заказчику точно определить требования
к продукту и обеспечить его соответствие ожиданиям пользователей.

\begin{itemize}
	\item \textbf{Как преподаватель}, я хочу создавать новые очереди
		для приема студентов, чтобы организовать процесс консультаций.
	\item \textbf{Как студент}, я хочу просматривать доступные очереди
		и записываться в них, чтобы получить помощь от преподавателей.
	\item \textbf{Как владельца очереди}, я хочу иметь возможность управлять
		пользователями и их правами доступа, чтобы обеспечить безопасность
		и эффективность системы.
	\item \textbf{Как преподаватель}, я хочу получать уведомления
		о новых запросах на консультацию, чтобы своевременно отвечать на них.
	\item \textbf{Как студент}, я хочу видеть свое место в очереди
		и примерное время ожидания, чтобы планировать свое время.
	\item \textbf{Как владельц очереди}, я хочу иметь возможность настраивать
		параметры каждой очереди, такие как максимальное количество студентов,
		время работы и т.д.,
		чтобы учитывать особенности каждого учебного заведения.
	\item \textbf{Как преподаватель}, я хочу иметь возможность отмечать
		студентов, которые были обслужены,
		чтобы поддерживать актуальную информацию о состоянии очереди.
	\item \textbf{Как студент}, я хочу отменять свою запись в очередь,
		если у меня изменятся планы,
		чтобы освободить место для других студентов.
	\item \textbf{Как владельц очереди}, я хочу иметь доступ к отчетам
		и статистике о работе приложения,
		чтобы анализировать его эффективность и делать необходимые улучшения.
	\item \textbf{Как студент}, я хочу иметь возможность оставить обратную
		связь о качестве обслуживания,
		чтобы помочь улучшить процесс консультаций и организацию очередей.
\end{itemize}

Каждая из этих пользовательских историй определяет конкретные требования
к функциональности мобильного приложения "'Очередь"'
и помогает сформулировать его основные возможности.

\Section{Обоснование выбора инструментальных средств
для разработки мобильного приложения}

Для разработки мобильного приложения "'Очередь"' были выбраны следующие
инструментальные средства и технологии:
Kotlin, Android Studio, Ktor и PostgreSQL.
Ниже приведено обоснование выбора каждого из этих средств.

\Subsection{Kotlin}

Kotlin был выбран в качестве основного языка программирования
для разработки мобильного приложения по следующим причинам:

\begin{itemize}
    \item \textbf{Совместимость с Java:} Kotlin полностью совместим с Java,
		что позволяет использовать существующие библиотеки
		и фреймворки Java без необходимости переписывать их на новый язык.
    \item \textbf{Простота и лаконичность:} Kotlin предлагает более лаконичный
		и читаемый синтаксис по сравнению с Java,
		что сокращает количество кода и уменьшает вероятность ошибок.
    \item \textbf{Безопасность:} Kotlin имеет встроенные механизмы
		для предотвращения распространенных ошибок,
		таких как NullPointerException, что повышает надежность кода.
    \item \textbf{Поддержка Android:} Kotlin официально поддерживается Google
		для разработки приложений под Android,
		что гарантирует регулярные обновления
		и хорошую интеграцию с Android Studio.
\end{itemize}

\Subsection{Android Studio}

Android Studio был выбран в качестве основной среды разработки (IDE)
для создания мобильного приложения по следующим причинам:

\begin{itemize}
    \item \textbf{Интеграция с инструментами:} Android Studio предлагает
		тесную интеграцию с Gradle, системой сборки,
		которая упрощает управление зависимостями
		и автоматизацию сборки проекта.
    \item \textbf{Отладка и тестирование:} Встроенные инструменты для отладки
		и тестирования позволяют разработчикам быстро находить
		и исправлять ошибки, а также проводить юнит-тестирование
		и UI-тестирование.
    \item \textbf{Эмуляторы:} Android Studio предоставляет эмуляторы различных
		устройств, что позволяет тестировать приложение на разных версиях
		Android и конфигурациях устройств.
    \item \textbf{Инструменты дизайна:} Удобные инструменты
		для разработки пользовательского интерфейса (UI),
		такие как Layout Editor, упрощают создание
		и визуализацию макетов экранов.
\end{itemize}

\Subsection{Ktor}

Ktor был выбран для разработки серверной части приложения
по следующим причинам:

\begin{itemize}
    \item \textbf{Легковесность:} Ktor является легковесным фреймворком,
		что обеспечивает высокую производительность
		и быстрое время отклика сервера.
    \item \textbf{Модульность:} Ktor предлагает модульную архитектуру,
		которая позволяет легко добавлять или удалять функциональность
		в зависимости от потребностей проекта.
    \item \textbf{Гибкость:} Ktor предоставляет гибкие инструменты
		для обработки запросов и управления маршрутизацией,
		что упрощает создание RESTful API.
    \item \textbf{Поддержка Kotlin:} Ktor написан на Kotlin и интегрируется
		с ним, что обеспечивает единообразие в разработке клиентской
		и серверной части приложения.
\end{itemize}

\Subsection{PostgreSQL}

В качестве системы управления базами данных (СУБД)
была выбрана PostgreSQL по следующим причинам:

\begin{itemize}
    \item \textbf{Надежность:} PostgreSQL известна своей надежностью
		и устойчивостью к сбоям, что является критически важным
		для хранения данных пользователей и управления очередями.
    \item \textbf{Производительность:} PostgreSQL обеспечивает высокую
		производительность и масштабируемость,
		позволяя обрабатывать большое количество запросов и данных.
    \item \textbf{Расширяемость:} PostgreSQL поддерживает расширяемость
		за счет использования пользовательских функций и типов данных,
		что позволяет адаптировать базу данных
		под специфические требования приложения.
    \item \textbf{Сообщество и поддержка:} PostgreSQL имеет большое сообщество
		и хорошую документацию, что облегчает поиск решений
		и внедрение новых функций.
\end{itemize}

Таким образом, выбор инструментальных средств
для разработки мобильного приложения "'Очередь"' основан на их надежности,
производительности, удобстве использования
и поддержке современных стандартов разработки.

